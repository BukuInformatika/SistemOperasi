\section{Sistem Operasi}
	\subsection{Definisi}
		\begin{enumerate}
			\item Sistem operasi (OS) adalah sistem perangkat lunak yang mengelola hardware dan sumber daya perangkat lunak dengan menyediakan pelayanan secara umum untuk program komputer.
			\item Sistem operasi berbagi waktu menjadwalkan tugas untuk penggunaan sistem yang efisien dan mungkin juga termasuk perangkat lunak akuntansi untuk alokasi biaya waktu prosesor, penyimpanan massal, pencetakan, dan sumber daya lainnya.
			\item Untuk fungsi perangkat keras seperti input dan output dan alokasi memori, sistem operasi bertindak sebagai perantara antara program dan perangkat keras komputer, meskipun kode aplikasi biasanya dijalankan langsung oleh perangkat keras dan sering membuat panggilan sistem ke fungsi OS atau terganggu oleh saya t. Sistem operasi banyak ditemukan pada perangkat komputer,telepon seluler dan konsol permainan video ke server web dan superkomputer.
			\item Sistem operasi desktop yang dominan adalah Microsoft Windows dengan pangsa pasar sekitar 82,74%. macOS oleh Apple Inc. berada di tempat kedua (13,23%), dan varietas Linux secara kolektif berada di tempat ketiga (1,57%). Di sektor seluler (gabungan ponsel dan tablet), penggunaan pada tahun 2017 adalah hingga 70% dari Google Android dan menurut data kuartal ketiga 2016, Android pada smartphone dominan dengan 87,5 persen dan tingkat pertumbuhan 10,3 persen per tahun, diikuti oleh Apple iOS dengan 12,1 persen dan penurunan per tahun di pangsa pasar 5,2 persen, sementara jumlah sistem operasi lainnya hanya 0,3 persen. Distribusi Linux dominan di sektor server dan superkomputer. Kelas khusus lainnya dari sistem operasi, seperti embedded dan sistem real-time, ada untuk banyak aplikasi.
		\end{enumerate}