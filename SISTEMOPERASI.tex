\section{Sistem Operasi}
	\subsection{Definisi}
		\begin{enumerate}
			\item Sistem operasi (OS) adalah sistem perangkat lunak yang mengelola hardware dan sumber daya perangkat lunak dengan menyediakan pelayanan secara umum untuk program komputer.
			\item Sistem operasi berbagi waktu menjadwalkan tugas untuk penggunaan sistem yang efisien dan mungkin juga termasuk perangkat lunak akuntansi untuk alokasi biaya waktu prosesor, penyimpanan massal, pencetakan, dan sumber daya lainnya.
			\item Untuk fungsi perangkat keras seperti input dan output dan alokasi memori, sistem operasi bertindak sebagai perantara antara program dan perangkat keras komputer, meskipun kode aplikasi biasanya dijalankan langsung oleh perangkat keras dan sering membuat panggilan sistem ke fungsi OS atau terganggu oleh saya t. Sistem operasi banyak ditemukan pada perangkat komputer,telepon seluler dan konsol permainan video ke server web dan superkomputer.
			\item Sistem operasi desktop yang dominan adalah Microsoft Windows dengan pangsa pasar sekitar 82,74%. macOS oleh Apple Inc. berada di tempat kedua (13,23%), dan varietas Linux secara kolektif berada di tempat ketiga (1,57%). Di sektor seluler (gabungan ponsel dan tablet), penggunaan pada tahun 2017 adalah hingga 70% dari Google Android dan menurut data kuartal ketiga 2016, Android pada smartphone dominan dengan 87,5 persen dan tingkat pertumbuhan 10,3 persen per tahun, diikuti oleh Apple iOS dengan 12,1 persen dan penurunan per tahun di pangsa pasar 5,2 persen, sementara jumlah sistem operasi lainnya hanya 0,3 persen. Distribusi Linux dominan di sektor server dan superkomputer. Kelas khusus lainnya dari sistem operasi, seperti embedded dan sistem real-time, ada untuk banyak aplikasi.
		\end{enumerate}

\subsection{Jenis sistem operasi}
\subsubsection{Single dan multi-tasking}
\begin{enumerate}
\item Sistem satu tugas hanya dapat menjalankan satu program dalam satu waktu, sementara sistem operasi multi-tasking memungkinkan lebih dari satu program berjalan dalam konkurensi. Ini dicapai dengan time-sharing, di mana waktu prosesor yang tersedia dibagi antara beberapa proses. Proses ini masing-masing terganggu berulang kali dalam irisan waktu oleh subsistem penjadwalan tugas dari sistem operasi. Multi-tasking dapat dicirikan dalam tipe preemptif dan kooperatif. Dalam preemptive multitasking, sistem operasi memotong waktu CPU dan mendedikasikan slot untuk masing-masing program. Sistem operasi mirip Unix, seperti Solaris dan Linux — serta non-Unix-like, seperti AmigaOS — mendukung multitasking preemptif. Multitasking kooperatif dicapai dengan mengandalkan pada setiap proses untuk menyediakan waktu untuk proses lain dengan cara yang ditentukan. Versi 16-bit Microsoft Windows menggunakan multi-tasking kooperatif. Versi 32-bit dari Windows NT dan Win9x, menggunakan preemptive multi-tasking.
\end{enumerate}
\subsection{Sejarah}
\begin{enumerate}
\item Komputer awal dibangun untuk melakukan serangkaian tugas tunggal, seperti kalkulator. Fitur-fitur sistem operasi dasar dikembangkan pada tahun 1950-an, seperti fungsi monitor penduduk yang dapat secara otomatis menjalankan program yang berbeda secara berurutan untuk mempercepat pemrosesan. Sistem operasi tidak ada dalam bentuk modern dan lebih kompleks hingga awal 1960-an. Fitur perangkat keras ditambahkan, yang memungkinkan penggunaan pustaka runtime, interupsi, dan pemrosesan paralel. Ketika komputer pribadi menjadi populer pada tahun 1980-an, sistem operasi dibuat untuk mereka yang serupa dalam konsep untuk yang digunakan pada komputer yang lebih besar.
\item Pada 1940-an, sistem digital elektronik paling awal tidak memiliki sistem operasi. Sistem elektronik saat ini diprogram pada deretan switch mekanis atau oleh kabel jumper pada papan steker. Ini adalah sistem tujuan khusus yang, misalnya, menghasilkan tabel balistik untuk militer atau mengontrol pencetakan cek gaji dari data pada kartu kertas berlubang. Setelah komputer tujuan umum yang dapat diprogram diciptakan, bahasa mesin (terdiri dari string digit biner 0 dan 1 pada pita kertas berlubang) diperkenalkan yang mempercepat proses pemrograman (Stern, 1981). [Kutipan lengkap diperlukan]
OS / 360 digunakan pada sebagian besar komputer mainframe IBM mulai tahun 1966, termasuk komputer yang digunakan oleh program Apollo.
\itemPada awal 1950-an, komputer hanya dapat menjalankan satu program dalam satu waktu. Setiap pengguna memiliki satu-satunya penggunaan komputer untuk jangka waktu terbatas dan akan tiba pada waktu yang dijadwalkan dengan program dan data pada kartu kertas berlubang atau pita berlubang. Program akan dimuat ke dalam mesin, dan mesin akan diatur untuk bekerja sampai program selesai atau crash. Program umumnya dapat di-debug melalui panel depan menggunakan saklar beralih dan lampu panel. Dikatakan bahwa Alan Turing adalah seorang ahli dalam mesin Manchester Mark 1 awal ini, dan dia sudah mendapatkan konsep primitif dari sistem operasi dari prinsip-prinsip mesin Turing universal.
\item Kemudian mesin datang dengan perpustakaan program, yang akan dikaitkan dengan program pengguna untuk membantu dalam operasi seperti input dan output dan menghasilkan kode komputer dari kode simbolik yang dapat dibaca manusia. Ini adalah awal dari sistem operasi modern. Namun, mesin masih menjalankan pekerjaan tunggal pada suatu waktu. Di Cambridge University di Inggris, antrian pekerjaan pada suatu waktu adalah garis pencucian (garis pakaian) dari mana kaset digantung dengan pakaian warna berbeda untuk menunjukkan prioritas pekerjaan. 
\end{enumerate}
