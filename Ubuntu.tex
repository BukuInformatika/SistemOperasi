\section {Ubuntu}
\subsection {Definisi}
\begin ({enumerance}
\item Ubuntu sistem operasi open source untuk komputer. Ini adalah distribusi Linux berdasarkan arsitektur Debian. Ini biasanya berjalan di komputer pribadi, dan juga populer di server jaringan, biasanya menjalankan varian Ubuntu Server, dengan fitur kelas perusahaan. Ubuntu beroperasi pada arsitektur yang paling terkenal, seperti Intel, AMD, dan mesin berbasis ARM. Ubuntu juga tersedia untuk tablet dan smartphone, dengan edisi Ubuntu Touch.
Ubuntu diterbitkan oleh Canonical Ltd, yang menawarkan dukungan komersial. Hal ini didasarkan pada perangkat lunak bebas dan dinamakan sesuai dengan filosofi Afrika Selatan dari ubuntu (secara harfiah, 'ke-manusia-an'), yang menurut Canonical Ltd. dapat diterjemahkan secara bebas sebagai "kemanusiaan untuk orang lain" atau "Saya adalah saya karena siapa kami semua adalah".
Ubuntu adalah sistem operasi paling populer yang berjalan di lingkungan yang dihosting, yang disebut "awan", karena ini adalah distribusi server Linux yang paling populer.
Pengembangan Ubuntu dipimpin oleh Canonical Ltd. yang berbasis di Inggris, sebuah perusahaan yang didirikan oleh pengusaha Afrika Selatan, Mark Shuttleworth. Canonical menghasilkan pendapatan melalui penjualan dukungan teknis dan layanan lain yang terkait dengan Ubuntu. Proyek Ubuntu secara terbuka berkomitmen pada prinsip pengembangan perangkat lunak sumber terbuka; orang didorong untuk menggunakan perangkat lunak bebas, mempelajari cara kerjanya, meningkatkan, dan mendistribusikannya.