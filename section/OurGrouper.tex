\section {Tkinter}

\subsection{Definisi}
Modul Tkinter atau Tk interface adalah Python interface standar ke toolkit Tk GUI dari Scriptics yang sebelumnya dikembangkan oleh Sun Labs. Baik Tk dan Tkinter tersedia di sebagian besar platform Unix, begitu juga pada sistem Windows dan Macintosh. Dimulai dengan rilis 8.0, Tk menawarkan tampilan dan nuansa asli pada semua platform. Tkinter terdiri dari sejumlah modul. Antarmuka Tk terletak di modul biner bernama _tkinter (ini tkinter di versi sebelumnya). Modul ini berisi antarmuka tingkat rendah ke Tk, dan tidak boleh digunakan langsung oleh pemrogram aplikasi. Ini biasanya adalah shared library (atau DLL), tetapi mungkin dalam beberapa kasus terhubung secara statis dengan penerjemah Python. Selain modul antarmuka Tk, Tkinter menyertakan sejumlah modul Python. Dua modul yang paling penting adalah modul Tkinter itu sendiri, dan modul yang disebut Tkconstants. Yang pertama secara otomatis mengimpor yang terakhir, jadi untuk menggunakan Tkinter, semua yang perlu Anda lakukan adalah mengimpor satu modul: 

\begin {verbatim}
	import Tkinter
\end {verbatim}
	
Atau, lebih sering:

\begin {verbatim} 
Form Tkinter import *
\end {verbatim}

\begin {table}
\centering
\begin {tabular}{|l|l|}
	\hline
	Widget & Deskripsi\\
	\hline
	Button & Tombol sederhana, digunakan untuk menjalankan perintah atau operasi lainnya\\
	\hline
	Canvas & Grafis terstruktur. Widget ini dapat digunakan untuk menggambar grafik dan plot, membuat editor grafis, dan mengimplementasikan widget khusus.\\
	\hline
	Checkbutton & Merupakan variabel yang dapat memiliki dua nilai berbeda. Mengklik tombol matikan antara nilai-nilai.\\
	\hline
	Entry & \\
	\hline
	Frame	 & \\
	\hline
	Label & \\
	\hline
	Listbox & \\
	\hline
	Menu & \\
	\hline
	Menubutton & \\
	\hline
	Message & \\
	\hline
	Radiobutton & \\
	\hline
	Scale & \\
	\hline
	Scrollbar & \\
	\hline
	Text & \\
	\hline
	Toplevel & \\
	
\end {tabular}
\end {table}