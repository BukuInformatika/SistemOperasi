% Kelompok : 5
% Kelas : D4 TI 1A
% Anggota : 
% 1. Harun Ar - Rasyid 	1174027
% 2. Choirul Anam 		1174004
% 3. M.Tomy N.M.		1174031
% 4. Izza			1174013

Artikel ini mengenai PnP

\section{Definisi}
Plug and Play merupakan salah satu fasilitas yang tersedia dalam PC yang memiliki tujuan untuk memudahkan pemakai dalam melkakukan instalasi. Dengan adanya fasilitas ini, instaler ketika akan memasang hardware seperti mouse, keyboard, webcam maka tidak usah lagi susah susah untuk memilih driver. Karena dengan adanya fasilitas ini, sistem operasi otomatis akan langsung ke konfigursinya yang paling tepat. Contoh sistem yang sering digunakan dengan fasilitas ini adalah USB dan Fireware. 
\section{Pengenalan}
Direktorat Angkutan Luar Angkasa Laboratorium Penelitian Angkatan Udara (AFRL / RV) telah mengembangkan suatu pendekatan untuk pembuatan satelit secara cepat. Pendekatan ini, disebut sebagai ruang avionik plug-and-play (SPA), menggabungkan modularitas, standardisasi, dan antarmuka cerdas. Sistem adalah pengaturan perangkat SPA, masing-masing dirancang agar terlihat seperti kotak hitam dengan antarmuka umum. Standar seperti itu telah dicoba sebelumnya.