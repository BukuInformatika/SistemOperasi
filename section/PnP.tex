% Kelompok : 5
% Kelas : D4 TI 1A
% Anggota : 
% 1. Harun Ar - Rasyid 	1174027
% 2. Choirul Anam 		1174004
% 3. M.Tomy N.M.		1174031
% 4. Izza			1174013

Artikel ini mengenai PnP

\section{Definisi}
Plug and Play merupakan salah satu fasilitas yang tersedia dalam PC yang memiliki tujuan untuk memudahkan pemakai dalam melkakukan instalasi. Dengan adanya fasilitas ini, instaler ketika akan memasang hardware seperti mouse, keyboard, webcam maka tidak usah lagi susah susah untuk memilih driver. Karena dengan adanya fasilitas ini, sistem operasi otomatis akan langsung ke konfigursinya yang paling tepat. Contoh sistem yang sering digunakan dengan fasilitas ini adalah USB dan Fireware. 
\section{Pengenalan}
Direktorat Angkutan Luar Angkasa Laboratorium Penelitian Angkatan Udara (AFRL / RV) telah mengembangkan suatu pendekatan untuk pembuatan satelit secara cepat. Pendekatan ini, disebut sebagai ruang avionik plug-and-play (SPA), menggabungkan modularitas, standardisasi, dan antarmuka cerdas. Sistem adalah pengaturan perangkat SPA, masing-masing dirancang agar terlihat seperti kotak hitam dengan antarmuka umum. Standar seperti itu telah dicoba sebelumnya.
Apa yang membedakan SPA adalah bahwa setiap kotak hitam menggambarkan dirinya sendiri (melalui lembar data elektronik tertanam), dan jaringan perangkat ini mengatur dirinya sendiri untuk membentuk sistem. Dengan demikian, sejumlah perangkat black box SPA dapat diambil dari inventaris, dengan cepat dirakit, diintegrasikan, dan diuji dengan menggabungkan komponen, mengkonfigurasinya, dan melatihnya melalui pendekatan uji virtual yang disebut sebagai uji memotong. Kemampuan SPA ini telah ditunjukkan melalui pengembangan Plug and Play Satellite (PnPSat) [1].
PnPSat adalah upaya pertama untuk membuat seluruh sistem aerospace dari komponen plug-and-play (SPA). Ini adalah satelit kecil (180 kg) yang dirancang, dikembangkan, dan dievaluasi sebagai arsitektur satelit taktis prospektif untuk digunakan dalam misi dukungan taktis. PnPSat menggunakan sejumlah fitur baru, termasuk panel yang dibuat sebelumnya dengan grid pegboard 5 cm x-y untuk komponen pemasangan.
Sementara pendekatan SPA untuk plug-and-play cukup menjanjikan, konsep ini akan tetap sedikit lebih dari sekadar teknologi tanpa paparan langsung ke calon pengembang dan pengguna. Sama seperti sistem operasi yang membutuhkan aplikasi agar berguna, SPA mensyaratkan keberadaan komponen SPA untuk membuat sistem SPA (alias satelit). 
Memulai banyak proyek satelit pada skala PnPSat, bagaimanapun, akan menjadi proposisi yang mahal. Sementara AFRL sedang mempertimbangkan pengadaan yang melibatkan SPA [4], upaya ini tentu terbatas dalam ruang lingkup untuk memfokuskan sumber daya hanya pada beberapa penyedia. Untuk mempromosikan jangkauan yang terjangkau dan untuk berkecambah dalam pembuatan komponen plug-and-play, AFRL telah menjelajahi integrasi SPA dengan CubeSats, karena platform dengan harga lebih rendah ini lebih mudah diakses oleh berbagai macam pengguna.
CubeSats, didefinisikan sebagai sangat kecil (10 × 10 × 10n cm
volume dan massa 1-3kg, di mana n adalah antara 1 dan 3)
pesawat ruang angkasa [5] telah menerima banyak sekali
perhatian baru-baru ini (penilaian informal kami miliki
mengungkapkan lebih dari 150 kelompok memiliki beberapa proyek penelitian,
terbaru atau berkelanjutan). Kami merasakan banyak minat baru-baru ini
berasal dari pengembangan yang sederhana namun efektif
dispenser, dikenal sebagai “Orbital Poly-Picosatelit
Dispenser ”(PPOD). PPOD, dengan enkapsulasi penuh
beberapa Cubesats yang lebih kecil, memungkinkan seluruh satelit berada
diperlakukan sebagai kotak hitam, menyederhanakan integrasi mereka
dengan kendaraan peluncuran. Berpegang pada Cubesat
spesifikasi amplop menjamin kepatuhan terhadap
PPOD. PPOD memisahkan (untuk urutan pertama) kebutuhan
untuk pengembang Cubesat untuk menyibukkan diri dengan
seluk-beluk integrasi peluncuran dan, sebaliknya, batas
perlunya penyedia peluncuran untuk berpikir banyak tentang
satelit yang mungkin ada di PPODs.
Sementara Cubesats adalah salah satu kelas kendaraan ruang angkasa yang paling sederhana, kebanyakan dari mereka, seperti rekan-rekan mereka yang lebih besar (yaitu, pesawat ruang angkasa tradisional) dibangun dengan susah payah, seperti jam tangan Swiss. Meskipun ada lusinan upaya pengembangan independen, komponen individu Cubesats tertentu, untuk sebagian besar, belum dapat dipertukarkan. Ide memperluas plug-and-play seperti SPA ke Cubesats tampaknya proposisi yang menarik, karena pertukaran komponen antara pembangunan Cubesat yang berbeda kemungkinan akan menghasilkan ekonomi yang signifikan dalam upaya dan pengurangan waktu yang diperlukan untuk membuat Cubesats. Namun, implementasi SPA belum dioptimalkan sebelumnya untuk kompatibilitas standar CubeSat. Penggabungan SPA dan CubeSats memberikan tantangan yang menarik, yang menjadi fokus dari penelitian baru-baru ini, yang hasilnya dijelaskan dalam makalah ini.
Itu