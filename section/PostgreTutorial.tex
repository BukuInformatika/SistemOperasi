\section{Cara Menghubungkan Python Dengan PostgreSQL}

\subsection{Definisi PostgreSQL}
\cite{momjian2001postgresql}PostgreSQL adalah database server dengan basis open source yang paling maju. PostgreSQL sendiri sudah dipakai oleh banyak kalangan di luar negeri karena faktor keamanan yang cukup terjamin. Seperti database server lainnya, PostgreSQL memakai bahasa pada umumnya yaitu standard SQL tetapi beberapa syntax diubah untuk kepentingan keamanan.Dalam PostgreSQL dapat menyimpan titik geografis dari aplikasi Sistem Informasi Geografis yang membuat bahasa ini popular dalam kalangan programmer Sistem Informasi Geografis. Selain titik geografis, PostgreSQL dapat menyimpan objek geometric diantaranya titik, garis, dan area. Selain itu, PostgreSQL dapat beroperasi seperti layaknya bahasa lainnya seperti manipulasi data atau membuat table. Hal tersebut memungkinkan PostgreSQL menjadi bahasa sql terbaik untuk digunakan pada Sistem Informasi Geografis karena sifatnya yang Spatially-enabled. Untuk mengunduh PostgreSQL dapat langsung dicari dan diunduh di website resminya yaitu di www.postgresql.org, cara instalasinya dapat terbilang cukup rumit karena untuk instalasi memakan banyak waktu dan tidak direkomendasikan untuk pemula, diantaranya harus memasuki regedit pada windows. 

\subsection{Integrasi Python Dengan PostgreSQL} 
Integrasi Python dengan PostgreSQL bisa dibilang sama dengan integrasi Python dengan bahasa lainnya.  Karena kita tinggal mendownload hal yang dibutuhkan dan membuat kode untuk dikoneksikan ke database. Salah satunya juga yaitu mengkoneksikan PostgreSQL dimana kita harus instalasi PostgreSQL itu sendiri. Berikut hal yang perlu disiapkan sebelum mengkoneksikan : 
\begin{itemize}
\item Connector Python to PostgreSQL : Fungsi connector yaitu penghubung antara python dengan postgresqlnya sendiri. 
\item Python (Semua Versi) : Sebelum mendownload piranti yang lain, pastikan python sudah terinstall di komputer
\item PostgreSQL DB : Sebagai media yang menyimpan data, aplikasi ini harus sudah terinstal juga di computer anda.
\end{itemize}



\subsection{Insert Data Di PostgreSQL}
Insert data pada umumnya memiliki sistem yang sama, bahasa yang sama. di PostgreSQL sendiri sama seperti bahasa yang umum yaitu MySQL yang memiliki syntax Insert seperti berikut
\begin{verbatim}
insert into db_postgre(kol1, kol2, kol3) values('kol1', 'kol2', 'kolek');
\end{verbatim}
Tetapi PostgreSQL memiliki aturan ketat mengenai cara insert database. Jadi tidak sembarang teknik menginsert dapat digunakan dalam metode insert data ke postgres.
