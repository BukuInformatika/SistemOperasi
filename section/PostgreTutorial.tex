\section{Cara Menghubungkan Python Dengan PostgreSQL}

\subsection{Definisi PostgreSQL}
\cite{momjian2001postgresql}PostgreSQL adalah database server dengan basis open source yang paling maju. PostgreSQL sendiri sudah dipakai oleh banyak kalangan di luar negeri karena faktor keamanan yang cukup terjamin. Seperti database server lainnya, PostgreSQL memakai bahasa pada umumnya yaitu standard SQL tetapi beberapa syntax diubah untuk kepentingan keamanan.Dalam PostgreSQL dapat menyimpan titik geografis dari aplikasi Sistem Informasi Geografis yang membuat bahasa ini popular dalam kalangan programmer Sistem Informasi Geografis. Selain titik geografis, PostgreSQL dapat menyimpan objek geometric diantaranya titik, garis, dan area. Selain itu, PostgreSQL dapat beroperasi seperti layaknya bahasa lainnya seperti manipulasi data atau membuat table. Hal tersebut memungkinkan PostgreSQL menjadi bahasa sql terbaik untuk digunakan pada Sistem Informasi Geografis karena sifatnya yang Spatially-enabled. Untuk mengunduh PostgreSQL dapat langsung dicari dan diunduh di website resminya yaitu di www.postgresql.org, cara instalasinya dapat terbilang cukup rumit karena untuk instalasi memakan banyak waktu dan tidak direkomendasikan untuk pemula, diantaranya harus memasuki regedit pada windows. 

\subsection{Integrasi Python Dengan PostgreSQL} 
Integrasi Python dengan PostgreSQL bisa dibilang sama dengan integrasi Python dengan bahasa lainnya.  Karena kita tinggal mendownload hal yang dibutuhkan dan membuat kode untuk dikoneksikan ke database. Salah satunya juga yaitu mengkoneksikan PostgreSQL dimana kita harus instalasi PostgreSQL itu sendiri. Berikut hal yang perlu disiapkan sebelum mengkoneksikan : 
\begin{itemize}
\item Connector Python to PostgreSQL : Fungsi connector yaitu penghubung antara python dengan postgresqlnya sendiri. 
\item Python (Semua Versi) : Sebelum mendownload piranti yang lain, pastikan python sudah terinstall di komputer
\item PostgreSQL DB : Sebagai media yang menyimpan data, aplikasi ini harus sudah terinstal juga di computer anda.
\end{itemize}



\subsection{Insert Data Di PostgreSQL}
Insert data pada umumnya memiliki sistem yang sama, bahasa yang sama. di PostgreSQL sendiri sama seperti bahasa yang umum yaitu MySQL yang memiliki syntax Insert seperti berikut
\begin{verbatim}
insert into db_postgre(kol1, kol2, kol3) values('kol1', 'kol2', 'kolek');
\end{verbatim}
Tetapi PostgreSQL memiliki aturan ketat mengenai cara insert database. Jadi tidak sembarang teknik menginsert dapat digunakan dalam metode insert data ke postgres.

\subsection{Pengujian}
Menguji POSTGRESQL pada platform UNIX sebagai berikut: 
beberapa kompilasi ini membutuhkan gcc
\begin {itemize} 
	\item aix - IBM pada AIX 3.2.5 atau 4.x
	\item alpha - DEC Alpha AXP pada Digital Unix 2.0, 3.2, 4.0
	\item BSD44_derived - OS yang berasal dari 4.4-lite BSD (NetBSD, FreeBSD)
	\item bsdi - BSD / OS 2.x, 3.x, 4.x
	\item dgux - DG / UX 5.4R4.11
	\item hpux - HP PA-RISC pada HP-UX 9. *, 10. *
	\item i386_solaris - i386
	\item Irix5 - SGI MIPS
	\item MIPS pada IRIX 5.3
	\item linux - Intel i86 Alpha SPARC PPC M68k
	\item sco - SCO 3.2v5
	\item Unixware
	\item sparc_solaris - SUN SPARC pada Solaris 2.4, 2.5, 2.5.1
	\item sunos4 - SUN SPARC pada SunOS 4.1.3
	\item svr4 - Intel x86 pada Intel SVR4 dan MIPS
	\item ultrix4 - DEC MIPS di Ultrix 4.4
\end {itemize}

Port selain sistem operasi Unix yang tersedia dan memungkinkan. 
Untuk mengkompilasi libpq C library, psql, dan interface lain dan binari untuk dijalankan pada MS Platform Windows. Dalam hal ini, klien berjalan di MS Windows, dan berkomunikasi melalui TCP / IP ke server yang berjalan di salah satu platform Unix kami yang didukung.Sebuah file win31.mak termasuk dalam distribusi untuk membuat libpq librq Win32 dan psql. 
Server database sekarang bekerja pada Windows NT menggunakan pustaka porting Cygnus Unix / NT

\subsection {Bahasa Pemrograman yang tersedia}
Bahasa yang tersedia untuk berkomunikasi dengan POSTGRESQL ada
 \begin {itemize}
	\item C atau libpq
	\item C ++ atau libpq ++
	\item Embedded C atau ecpg
	\item Java atau jdbc
	\item Perl atau perl5
	\item ODBC atau odbc
	\item Python atau PyGreSQL
	\item TCL atau libpgtcl
	\item C Easy API atau libpgeasy
	\item Embedded HTML atau PHP
\end {itemize}

\subsection{Instalasi} 
\begin {itemize}
	\item Dapatkan POSTGRESQL
	\item Membuat POSTGRESQL user
	\item Konfigurasi
	\item Kompilasi
	\item Instalasi
	\item Inisialisasi
	\item Memulai Server
	\item Membuat Database
\end {itemize}

\subsubsection{Akses selain host asli}
Cara untuk mencegah host lain mengakses
database POSTGRESQL, Secara default, POSTGRESQL hanya mengizinkan koneksi dari komputer lokal menggunakan soket domain Unix. Mesin lain tidak akan dapat terhubung kecuali Anda menambahkan flag -i ke postmaster, dan mengaktifkan hostbased otentikasi. Ini akan memungkinkan TCP / IP koneksi. Perangkat lunak POSTGRESQL didistribusikan dalam beberapa format:
\begin {itemize}
	\item File -Tar-gzip dengan ekstensi file .tar.gz
	\item File pra-paket dengan ekstensi file .rpm
	\item Format lain yang sudah dikemas
	\item CD ROM
\end {itemize}
Karena begitu banyak format yang ada, apendiks ini hanya akan membahas langkah-langkah umum yang perlu dipasang POSTGRESQL. Setiap distribusi dilengkapi dengan file INSTALL atau README dengan lebih spesifik instruksi.

