%Kelompok Memory Allocation
%Arjun Yuda Firwanda
%Dezha Aidil Martha
%Dwi Septiani Tsaniyah
%Muh.Rifky Prananda
%Yusuf Al-Qordhawi

Memory Allocation

\section {Memory Allocation}

\subsection {Pengertian Memori}

Memori adalah pusat dari operasi computer. Memori sebagai tempat penyimpanan dan dapat menyimpan file atau program dimana file tersebut tergantung ruang atau space penyimpannya di dalam komputer. Memori sabagai tempat penyimpanan yang harus di dijaga dengan sebaik mungkin agar terhindar dari virus atau file yang dapat merusak program atau fie yang disimpan.
<<<<<<< HEAD

\subsection {Fungsi Allocation Memori}


Manajemen memori merupakan salah satu bagian terpenting pada sistem operasi. Memori sangat perlu dikelola dengan sebaik-baiknya agar :

1.	Utility CPU akan meningkat.
2.	Data dan instruksi akan diakses dengan cepat oleh CPU.
3.	Tercapai efisiensitas dalam pemakaian memori.
4.	Transfer data memori utama ke/dari CPU lebih efisien.
5.	Mengelola informasi. 
6.	Mengalokasikan memori ke proses yang diperlukan. 
7.	Mendealokasikan memori dari proses yang telah selesai. 
8.	Mengelola swapping atau pagging antara memori utama dan disk.


=======
Jenis memori ada 2 yaitu RAM dan ROM, RAM adalah Random Akses Memori yang isi penyimpanannya dapat diakses dalam waktu yang tetap. ROM adalah Read Only Memori adalah penyimpanan data yang permanen. sedangkan jenis-jenis memori eksternal ada 2 yaitu berdasarkan karakteristik bahan dan berdasarkan jenis akses data. Tentunya hal tersebut tergantung space dan ruang penyimpanan yang ada.
>>>>>>> 2c895bec8ef7e450dd16745cb5647695be320d69

\subsection {Manajemen Proses}

A. Proses
Proses adalah sebuah program yang akan dieksekusi atau yang sedang belangsung. Sedangkan program adalah kumpulan instruksi yang ada di dalam program dengan bahasa yang mudah dimengerti system operasi. Proses berisi instruksi dan data. Program counter dan semua register pemroses, dan stack berisi data sementara seperti parameter rutin, alamat pengiriman dan variabel-variabel lokal.Sistem operasi menjalankan semua proses di sistem operasi dan mengalokasikan sumber daya ke dalam proses-proses yang sesuai dengan kebijaksanaan untuk memenuhi salah satu sasaran sistemnya itu sendiri. Abstraksi proses yaitu merupakan  hal yang mendasar dalam manajemen proses-prosesnya yang bejalan di sistem. Sistem operasi dalam mengelola proses ini dapat menjalankan operasi-operasi terhadap suatu proses.

B. Istilah yang terkait dengan Proses
Istilah Proses dapat dibagi menjadi dua, yaitu :

1. Multi Programming (Multitasking)
Multi Programming atau Multitasking adalah suatu istilah yang merujuk kepada metode banyak pekerjaan yang disebut proses, diolah menggunakan sumber daya CPU yang sama-sama bekerja.

2. Multiprocessing
Multiprocessing adalah Kemampuan komputer menjalani tugas beberapa proses secara bersamaan dengan menggunakan teknologi berbasis multiprocessor.
Contoh : Rendering Video, Compress File, dan sebagainya.


C. Tipe-tipe Manajemen Proses

A. Berdasarkan keberadaan swapping :

1. Manajemen tanpa swapping.
Manajemen memori tanpa pemindahan citra proses antara memori utama dan disk selama ekseskusi.

2. Manajemen dengan swapping.
Manajemen memori dengan pemindahan citra proses antara memori utama dan disk selama ekseskusi.

B. Manajemen Memori Berdasarkan Alokasi Memori

Terdapat dua cara menempatkan informasi ke dalam memori kerja

