%Kelompok Memory Allocation
%Arjun Yuda Firwanda
%Dezha Aidil Martha
%Dwi Septiani Tsaniyah
%Muh.Rifky Prananda
%Yusuf Al-Qordhawi

Memory Allocation

\section {Memory Allocation}

\subsection {Pengertian Memori}

Memori adalah pusat dari operasi computer. Memori sebagai tempat penyimpanan dan dapat menyimpan file atau program dimana file tersebut tergantung ruang atau space penyimpannya di dalam komputer. Memori sabagai tempat penyimpanan yang harus di dijaga dengan sebaik mungkin agar terhindar dari virus atau file yang dapat merusak program atau fie yang disimpan.

\subsection {Fungsi Allocation Memori}


Manajemen memori merupakan salah satu bagian terpenting pada sistem operasi. Memori sangat perlu dikelola dengan sebaik-baiknya agar :

1.	Utility CPU akan meningkat.
2.	Data dan instruksi akan diakses dengan cepat oleh CPU.
3.	Tercapai efisiensitas dalam pemakaian memori.
4.	Transfer data memori utama ke/dari CPU lebih efisien.
5.	Mengelola informasi. 
6.	Mengalokasikan memori ke proses yang diperlukan. 
7.	Mendealokasikan memori dari proses yang telah selesai. 
8.	Mengelola swapping atau pagging antara memori utama dan disk.


