%Kelompok Memory Allocation
%Arjun Yuda Firwanda
%Dezha Aidil Martha
%Dwi Septiani Tsaniyah
%Muh.Rifky Prananda
%Yusuf Al-Qordhawi

Memory Allocation

\section {Memory Allocation}

\subsection {Pengertian Memori}

Memori adalah pusat dari operasi computer. Memori sebagai tempat penyimpanan dan dapat menyimpan file atau program dimana file tersebut tergantung ruang atau space penyimpannya di dalam komputer. Memori sabagai tempat penyimpanan yang harus di dijaga dengan sebaik mungkin agar terhindar dari virus atau file yang dapat merusak program atau fie yang disimpan.
Jenis memori ada 2 yaitu RAM dan ROM, RAM adalah Random Akses Memori yang isi penyimpanannya dapat diakses dalam waktu yang tetap. ROM adalah Read Only Memori adalah penyimpanan data yang permanen. sedangkan jenis-jenis memori eksternal ada 2 yaitu berdasarkan karakteristik bahan dan berdasarkan jenis akses data. Tentunya hal tersebut tergantung space dan ruang penyimpanan yang ada.