%Nama Kelompok: Sistem_Operasi_Deadlock
%Kelas: D4 TI 1B
%Alit Fajar Kurniawan(1174057) 
%Muhammad Iqbal Panggabean(1174063)
%Muhammad Afra Faris(1174041)
%Khadijah Hasanah Puteri Harahap(1174044)

\section {Insert Database MongoDB}

\subsection {MongoDB}}
\subsubsection {Pengertian MongoDB}}
MongoDB (dari " humongous ") adalah platform lintas platform untuk sistem database yang berorientasi  dokumen. 
MongoDB diklasifikasikan sebagai database NoSQL dengan menghindari struktur basis data relasional berbasis tabel tradisional yang mendukung JSON sebagai dokumen dinamis (MongoDB menyebutnya dalam format BSON), membuatnya lebih mudah dan lebih cepat untuk mengintegrasikan beberapa data aplikasi. 
Dirilis di bawah Lisensi Publik Umum GNU Affero dan kombinasi lisensi Apache, MongoDB adalah perangkat lunak bebas dan sumber terbuka.

\subsubsection {Sejarah MongoDB}
Perusahaan ini dibangun pada Oktober 2007 oleh sebuah perusahaan New York City, 10gen (sekarang MongoDB Inc.) sebagai bagian dari platform yang direncanakan sebagai produk layanan.
MongoDB telah diadopsi sebagai perangkat lunak backend melalui banyak situs web dan layanan, termasuk Craigslist, eBay, Square, SourceForge, dan The New York Times. MongoDB adalah sistem basis data NoSQL yang paling populer. 
Perusahaan telah mengembangkan model pengembangan open source pada tahun 2009, dengan 10gen menawarkan dukungan komersial dan layanan lainnya.

\subsubsection {Menggunakan MongoDB pada Mongolab}
Langkah - Langkahnya Sebagai berikut :

1. Login ke Account MongoLab.

2. Klik Create New
3. Tentukan penyedia layanan cloud yang ada dipilihan beserta region servernya

4. Pada menu Plan pilih tab Single Node untuk memilih kapasitas dan harga penyimpanannya mulai dari Gratis sampai Menengah.Sedangkan Replica Set Cluster untuk kapasistas dan harga penyimpanan dari Menengah sampai Kelas atas.(Pilihanku free pada tab single Node)

5. Pada Database isikan nama Databasenya. (nama databaseku akademik).

6. Klik Tombol Create New MongoDB Deployment.

7. Double klik database yang sudah di buat tadi .

8. Pada menu collections terdapat add collections untuk menambah tabel.

9. Isikan nama tabel pada dialog box yang muncul.setelah itu klik Create. (nama tabelku matkul).

10. Muncul tampilan baru dimana terdapat tombol add docummnet yang digunakan untuk mengisi data pada tabel yang sudah kita buat tadi.

11. Isikan data dimana untuk membedakan antara data dengan nama kolom tabel adalah dengan mengunakan symbol titik dua ":".
12. Setelah selesai klik save atau bisa save and go back untuk menyimpan data lalu kembali kehalaman sebelumnya.

13. Pada menu DISPLAY MODE terdapat dua pilihan yaitu list dan tabel. 

untuk tampilan tabel perlu di atur untuk tempilan tabelnya dengan menklik kata edit teble view.

14. Terdapat panduan berisi contoh mengatur tampilan tabelnya.

\subsubsection {kelebihan mongodb}

1. Kerangka untuk MongodDB MySQL berasal dari simplistream dan format media adalah format JSON.

2. Replikasi, data cadangan real-time adalah fitur yang sangat penting. MongoDB cocok untuk rilis berita atau blog, tetapi masih tidak cocok untuk digunakan dengan sistem informasi keuangan karena MongoDB tidak mendukung transaksi SQL.

3. Auto-curtail adalah fitur yang memuntahkan database besar dalam beberapa bagian untuk meningkatkan kinerja basis data. Sangat penting untuk menggunakan diri sendiri ketika pers memiliki database dalam jutaan baris, yang membantu membagi tirai.

4.MongoDB mencoba yang berikut: C, C ++, C #, Erlang, Haskell, Java, JavaScript, .NET (C # F #, PowerShell), Lippe, Perl, PHP, Python, Ruby, dan Scala

5.Cross-platform, dapat digunakan di Windows, Linux, OS X dan Solaris

6.CRUD Viewer (Create, Read, Work, Remove) terasa banyak cahaya

7.Map / Reduce, kami akan sangat membantu dalam melakukan operasi agregasi. Di mana semua entri dari koleksi dan outputnya juga akan menjadi koleksi. Anda dapat menggunakan MySQL untuk meminta pertanyaan oleh GROUP BY

GridFS, menggunakan spesifikasi untuk menyimpan data yang sangat besar

\subsubsection {Kekurangan Maria DB}
1.	Yang pertama kali kekurangan Maria DB adalah MongoDB harus diinstal pada server, dan jika Anda menggunakan PHP, Anda juga harus me-refresh server Anda. Driver mongoDB Anda dapat digunakan oleh PHP.

2.	Dan yang kedua adalah MARIA DB Ini belum didukung oleh hosting, tetapi dapat ditipu menggunakan MongoQQ (gratis hingga 16 MB gratis)

