%Kelompok5OS
%Dika Sukma Pradana
%Iqbal Hambali
%Evietania Sujadi
%Liyana Majdah Rahma


\section{VPhython}
\subsection{Definisi}
		\begin{enumerate}
		\item VPython adalah bahasa pemrograman Python ditambah modul grafis 3D yang disebut Visual. VPython memungkinkan pengguna untuk membuat objek seperti bola dan kerucut dalam ruang 3D dan menampilkan objek-objek ini di windows. Ini membuatnya lebih mudah untuk membuat visualisasi sederhana, memungkinkan pemrogram untuk lebih fokus pada aspek komputasi dari program mereka. Kesederhanaan VPython telah menjadikannya alat untuk ilustrasi fisika sederhana, terutama dalam pengaturan pendidikan.
		\end{enumerate}
\subsection {A. Mengenal VPhython}
\secion{VPhython}
		\begin{enumerate}
			\item VPython adalah bahasa pemrograman Python ditambah modul grafis 3D yang disebut Visual. VPython memungkinkan pengguna untuk membuat objek seperti bola dan kerucut dalam ruang 3D dan menampilkan objek-objek ini di jendela. Hal ini mempermudah pembuatan visualisasi sederhana, memungkinkan pemrogram untuk lebih fokus pada aspek komputasi dari program mereka. Kesederhanaan VPython telah menjadikannya alat untuk ilustrasi fisika sederhana, terutama dalam lingkungan pendidikan.Vpython merupakan lingkungan pemrograman yang memungkinkan bagi  pemula untuk menulis program yang menghasilkan navigasi animasi 3D real-time. 
		\end{enumerate}
\subsection{Kegunaan Vphython}
		\begin{enumerate}
	\item VPhython adalah alat render sederhana untuk objek 3D dan grafik. Penggunaan utamanya dalam pendidikan, tetapi juga digunakan dalam pengaturan komersial atau penelitian. VPython pertama kali digunakan dalam mata kuliah pengantar fisika di Carnegie Mellon dan kemudian menyebar ke universitas lain dan akhirnya sekolah menengah atas, terutama yang berhubungan dengan Materi & Kurikulum Interaksi.
	\item Perkembangan terkait karena David Scherer dan Bruce Sherwood adalah GlowScript, yang memungkinkan untuk menulis dan menjalankan program VPython di browser, termasuk di perangkat seluler, berkat kompilasi RapydScript Python-ke-JavaScript, yang dibuat oleh Alexander Tsepkov. Program dapat ditulis, dijalankan, dan disimpan di glowscript.org, dan kode dikompilasi-ke-JavaScript dapat diekspor dan disematkan di halaman web seseorang. John Coady telah membuat versi ivisual untuk digunakan dalam IPython, sekarang lingkungan Jupyter, menggunakan pustaka grafis GlowScript WebGL untuk merender output 3D dalam notebook IPython / Jupyter. Rhett Allain di blog Wired-nya menunjukkan contoh penggunaan Trinket untuk menyematkan kode sumber VPython yang dapat diedit dan eksekusi 3D di halaman webnya sendiri.
		\end{enumerate}
\subsection{Kelebihan VPhython}

	\begin{enumerate}
		\item Tidak ada langkah kompilasi dan tautan (tautan) sehingga perubahan kecepatan dalam sistem yang dibuat ditambah aplikasi.
		\item Tidak ada deklarasi tipe data yang rumit sehingga program menjadi lebih sederhana, lebih pendek, dan lebih fleksibel.
		\item Manajemen memori otomatis adalah kumpulan sampah memori sehingga dapat menghindari enkripsi kode.
		\item Jenis data dan tingkat operasi yang tinggi adalah kecepatan dari sistem yang dibuat menggunakan jenis objek yang ada.
	\end{enumerate}

\subsection{Kekurangan VPhython}

	\begin {enumerate}
		\item Beberapa tugas berada di luar jangkauan python, seperti bahasa pemrograman dinamis lainnya, python tidak secepat atau seefisien statis, tidak seperti bahasa pemrograman kompilasi bahasa C yang serupa.
		\item Karena python adalah seorang interpreter, python bukanlah alat terbaik untuk memperkenalkan komponen kinerja yang penting.
		\item Python tidak dapat digunakan sebagai dasar untuk mengimplementasikan bahasa pemrograman untuk beberapa komponen, tetapi berfungsi dengan baik ketika bagian depan skrip mencarinya.
		\item Python memberikan efisiensi dan fleksibilitas dari tradeoff dengan tidak memberikannya secara luas. Python mempersiapkan bahasa pemrograman optimasi untuk kegunaan, bersama dengan alat yang diperlukan untuk mengintegrasikannya dengan bahasa pemrograman lainnya.
	\end {enumerate}

	\subsection{Variabel dan Tipe Data Pada vphython}
	
	\begin{enumerate}
		\item Variabel adalah memori yang dicadangkan untuk menyimpan nilai-nilai. Artinya dengan variabel, kita dapat menyimpan nilai dalam program yang nantinya akan kita tampilkan. Namun variabel hanya bersifat sementara. Yaitu hanya menyimpan ketika program dijalankan. Jika sudah ditutup, maka nilai didalam variabel tersebut akan terhapus otomatis.
		 Jenis Variabel Pada Python tidak serumit bahasa pemrograman lain. Karena pada bahasa pemrograman python, jenis variabel tidak ditentukan dengan tipe data yang ada. variabel pada python hanya kita buat dengan tulisan sesuai dengan selera kita. Jadi tidak perlu memusingkan diri dengan masalah variabel.
		\item Tipe Data adalah jenis nilai yang dapat ditampung variabel, dan memiliki jenis-jenis tertentu. Artinya tipe data merupakan tipe / jenis dari suatu value, yang nantinya akan disimpan ke dalam variabel.Dari pengertian ini, kita bisa mengetahui bahwa variabel dan tipe data adalah hal yang bersangkutan. Untuk bahasa pemrograman lain memang benar. Tapi untuk python itu tidak sepenuhnya benar.  
		 Tipe Data di Python memiliki berbagai jenis. Antara lain huruf, karakter, angka, boolean dan list (kelompok). tipe data tersebut nantinya disimpan ke dalam variabel di program python sesuai dengan jenisnya. 
	\end{enumerate}
	

