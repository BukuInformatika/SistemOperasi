\section{Cara Insert Database Mysql}

	\subsection{Database Mysql}}
	Basis data merupakan istilah yang mengacu kepada kumpulan data yang saling berhubungan satu sama lain, dan perangkat lunak harus mengacu pada sistem manajemen basis data DBMS. Jika konteksnya jelas, banyak administrator dan programer menggunakan istilah basis data untuk kedua makna, dan juga  merupakan sekumpulan data yang membentuk file yang saling berhubungan dengan format tertentu untuk membentuk data atau informasi baru. Atau database adalah kumpulan data yang saling berhubungan satu sama lain yang diatur berdasarkan skema atau struktur tertentu. Di komputer, basis data disimpan di perangkat penyimpanan perangkat keras, dan dengan perangkat lunak tertentu yang dimanipulasi oleh minat atau minat tertentu. 
	Hubungan atau hubungan data biasanya ditunjukkan oleh kunci dari setiap file. MySQL adalah sistem manajemen basis data perangkat lunak atau perangkat lunak SQL atau DBMS Multithread dan multi user. mysql sebenarnya merupakan turunan dari salah satu konsep utama dalam database untuk seleksi atau seleksi dan entri data yang memungkinkan operasi data dilakukan dengan mudah dan otomatis. 
	Michael Widenius merupakan seseorang yang menciptakan mysql pada tahun 1979, seorang programmer komputer swedia yang mengembangkan sistem database sederhana yang disebut unireg yang menggunakan koneksi mesin database isam tingkat rendah dengan pengindeksan. mysql adalah salah satu jenis server basis data yang paling populer. mysql menggunakan bahasa sql untuk mengakses database-nya. lisensi mysql adalah lisensi foss exception dan ada juga versi komersial. 
