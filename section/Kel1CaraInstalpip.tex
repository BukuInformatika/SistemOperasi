%kelompok 1 Sistem Operasi (Proses Os)
%Kelas D4 TI 1B
%Adam Noer Hidayatullah 1174097
%Ichsan Hizman
%Teddy
%Nisrina Aulia
%Irvan Rizkiansyah 1174043

\section{PIP}

    \subsection{PIP}
	PIP adalah singkatan dari pip installs python.sebuah tool yang memudahkan programmer untuk menginstall library-library  atau disebut juga sebuah package manager untuk Python

	\subsection{cara install pip di windows}
	Sebelum melakukan peng-install-an pip, di wajibkan untuk menginstall python terlebih dahulu.
		\begin{enumerate}
			\item download pip langsung ke website resminya di \url{<https://pip.pypa.io/en/latest/installing/>}
			\item letakan file get-pip.py ke direktori yang mudah di temukan
			\item Buka CMD  dan masuk ke direktori file get-pip.py yang tadi sudah deletakan direktory.
			\item Pada saat di  CMD langsung ketikkan python get-pip.py
			\item tunggu proses nya 
			\item Setelah menginstall, kini saatnya mengsetting Environment Variables supaya mudah dan dapat  diakses lewat CMD.
			\item Masuk ke Control Panel - System And Security - System - dvanced System Settings
			\item Lalu setelah muncul windows baru,klik Environment variables, terhadap  sesi System variables pilih path - edit dan tambahkan 
				  variables value ;C:/Python27/Scripts
		\end{enumerate}

	\subsection{cara upgrade pip}
		\begin{item}
			\item python -m pip install -U pip
		\end{itemize}
		
