%kelompok 1 Sistem Operasi (Proses Os)
%Kelas D4 TI 1B
%Adam Noer Hidayatullah 1174097
%Ichsan Hizman
%Teddy
%Nisrina Aulia
%Irvan Rizkiansyah 1174043

\section{PIP}

    \subsection{PIP}
	PIP adalah singkatan dari pip installs python.sebuah tool yang memudahkan programmer untuk menginstall library-library  atau disebut juga sebuah package manager untuk Python

	\subsection{cara install pip di windows}
	Sebelum melakukan peng-install-an pip, di wajibkan untuk menginstall python terlebih dahulu. Karena tanpa python tidak akan bisa mengeksekusi installer pip.
		\begin{enumerate}
			\item download pip langsung ke website resminya di \url{<https://pip.pypa.io/en/latest/installing/>}
			\item letakan file get-pip.py ke direktori yang mudah di temukan
			\item Buka CMD dan masuk ke direktori file get-pip.py yang tadi sudah diletakkan direktori.
			\item Pada saat di CMD langsung ketikkan python get-pip.py
			\item tunggu proses nya 
			\item Setelah menginstall, kini saatnya mensetting Environment Variables supaya mudah dan dapat menjalankan pip lewat CMD tanpa harus masuk ke dalam folder hasil instalasi pip.
			\item Masuk ke dalam Control Panel - System And Security - System - advanced System Settings
			\item Lalu setelah muncul windows baru, klik pada Environment variables, terhadap sesi System variables pilih path lalu klik edit dan tambahkan 
				  variables value ;C:/Python27/Scripts
		\end{enumerate}

	\subsection{cara upgrade pip}
		\begin{item}
			\item python -m pip install -U pip
		\end{itemize}
		
	
	Dirangkum dari makalah \cite{feautrierpip}
	Dirangkum dari makalah \cite{nation2011qutip}
	Dirangkum dari artikel \cite{jewett2016moltemplate}
