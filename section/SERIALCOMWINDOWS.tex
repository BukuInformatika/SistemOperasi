\section{Serial Com Windows}
	\subsection{Membuka Port}
		\begin{enumerate} 
			Dokumentasi SDK Platform menyatakan bahwa ketika membuka port komunikasi, panggilan ke CreateFile memiliki persyaratan berikut:
				\item 1. fdwShareMode harus nol. Port komunikasi tidak dapat dibagikan dengan cara yang sama seperti file yang dibagikan. Aplikasi yang menggunakan TAPI dapat menggunakan fungsi TAPI untuk memfasilitasi berbagi sumber daya antar aplikasi. Untuk aplikasi yang tidak menggunakan TAPI, penanganan warisan atau duplikasi diperlukan untuk berbagi port komunikasi. Berurusan dengan duplikat berada di luar cakupan artikel ini; silakan merujuk ke dokumentasi Platform SDK untuk informasi lebih lanjut.
				\item 2. fdwCreate harus menentukan bendera OPEN_EXISTING.
				\item 3. hTemplateFile parameter harus NULL.
			Satu hal yang perlu diperhatikan tentang nama port adalah bahwa mereka secara tradisional telah COM1, COM2, COM3, atau COM4. Windows API tidak menyediakan mekanisme apa pun untuk menentukan port apa yang ada pada sistem. Beberapa sistem bahkan memiliki lebih banyak port daripada maksimum tradisional empat. Vendor perangkat keras dan pembuat perangkat serial-driver bebas memberi nama port apa pun yang mereka sukai. Untuk alasan ini, yang terbaik adalah pengguna memiliki kemampuan untuk menentukan nama port yang ingin mereka gunakan. Jika port tidak ada, kesalahan akan terjadi (ERROR_FILE_NOT_FOUND) setelah mencoba membuka port, dan pengguna harus diberitahu bahwa port tidak tersedia.
		\end{enumerate}
		\begin{verbatim}
			HANDLE hComm;
			hComm = CreateFile( gszPort,  
                    GENERIC_READ | GENERIC_WRITE, 
                    0, 
                    0, 
                    OPEN_EXISTING,
                    FILE_FLAG_OVERLAPPED,
                    0);
			if (hComm == INVALID_HANDLE_VALUE)
				// error opening port; abort
		\end{verbatim}
	\subsection{Membaca dan menulis}
		\begin{enumerate}
			\item Membaca dari dan menulis ke port komunikasi di Windows sangat mirip dengan file input / output (I / O) di Windows. Bahkan, fungsi yang melengkapi I / O file adalah fungsi yang sama yang digunakan untuk serial I / O. I / O dapat dilakukan dengan salah satu dari dua cara: tumpang tindih atau tidak tumpang tindih. Dokumentasi SDK Platform menggunakan istilah asinkron dan sinkron untuk mengkonotasikan jenis operasi I / O ini. Artikel ini, bagaimanapun, menggunakan istilah yang tumpang tindih dan tidak terabaikan.
			\item Nonoverlapped I / O akrab bagi kebanyakan pengembang karena ini adalah bentuk tradisional I / O, di mana operasi diminta dan diasumsikan lengkap ketika fungsi kembali. Dalam kasus I / O yang tumpang tindih, sistem dapat kembali ke pemanggil segera bahkan ketika operasi tidak selesai dan akan memberi sinyal kepada pemanggil ketika operasi selesai. Program ini dapat menggunakan waktu antara permintaan I / O dan penyelesaiannya untuk melakukan beberapa pekerjaan latar belakang.
				\subsubsection{Bacaan}
					\begin{enumerate}
						\item Fungsi ReadFile menerbitkan operasi baca. ReadFileEx juga mengeluarkan operasi baca, tetapi karena tidak tersedia pada Windows 95, itu tidak tercakup dalam artikel ini. Berikut adalah potongan kode yang merinci cara mempublikasikan permintaan baca. Perhatikan bahwa fungsi memanggil fungsi untuk memproses data jika ReadFile mengembalikan TRUE. Ini adalah fungsi yang sama yang disebut jika operasi menjadi tumpang tindih. Perhatikan flag fWaitingOnRead yang didefinisikan oleh kode; ini menunjukkan apakah operasi baca tumpang tindih atau tidak. Ini digunakan untuk mencegah penciptaan operasi baca baru jika mereka luar biasa.
				
				\begin{verbatim}
				\item DWORD dwRead;
					BOOL fWaitingOnRead = FALSE;
					OVERLAPPED osReader = {0};

						// Create the overlapped event. Must be closed before exiting
						// to avoid a handle leak.
						osReader.hEvent = CreateEvent(NULL, TRUE, FALSE, NULL);

						if (osReader.hEvent == NULL)
						// Error creating overlapped event; abort.

						if (!fWaitingOnRead) {
						// Issue read operation.
						if (!ReadFile(hComm, lpBuf, READ_BUF_SIZE, &dwRead, &osReader)) {
						if (GetLastError() != ERROR_IO_PENDING)     // read not delayed?
						// Error in communications; report it.
					else
						fWaitingOnRead = TRUE;
				}
					else {    
						// read completed immediately
						HandleASuccessfulRead(lpBuf, dwRead);
			}
		}
				\end{verbatim}
			
		\end{enumerate}
			\subsubsection{Penulisan}
				\begin{enumerate}
					\item Pengarsipan data dari port komunikasi sangat mirip dengan membaca karena menggunakan banyak API yang sama. Cuplikan kode di bawah ini menunjukkan cara menghapus dan menunggu operasi tulis selesai.
				\end{enumerate}
