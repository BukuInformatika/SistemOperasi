\section{Sistem Keamanan Linux}
\subsection{SELinux}
	SELinux \textit{(Security Enhaced Linux)} yang mana merupakan salah satu peningkatan keamanan dari sebuah sistem operasi berbasiskan linux, keamanan yang dimaksud disini yaitu untuk membedakan antara user root dan juga user yang sifatnya terbatas atau memiliki hak akses masing-masing. Aplikasi mendasar dari SELinux ini adalah layanan FTP \textit{(File Transfer Protocol)} dan HTTP \textit{(Hyper Text Transfer Protocol)}.
SELinux ini memiliki 3 mode yaitu :
\begin{enumerate}
\item Enforcing, merupakan pengaturan keamanan yang paling ketat
\item Permissive, merupakan pengaturan keamanan yang longgar
\item Disabled, merupakan pengaturan untuk memayikan SELinux
\end{enumerate}

pada linux terdapat SELinux, di windows pun sebenarnya ada yang seperti itu namun dengan nama yang berbeda yaitu \textit{User Account Control} atau UAC berfungsi untuk menjalankan aplikasi atau membuat, mengedit dan menghapus program yang penting.
