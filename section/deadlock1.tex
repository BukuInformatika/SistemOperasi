%Nama Kelompok: Sistem_Operasi_Deadlock
%Kelas: D4 TI 1B
%Alit Fajar Kurniawan 
%Muhammad Iqbal Panggabean
%Muhammad Afra Faris
%Khadijah Khasanah Puteri Harahap


\subsection {Deadlock}
\subsubsection {Pengertian Deadlock}
	Pada kesempatan ini saya akan menjelaskan tentang definisi Deadlock, Deadlock merupakan suatu kondisi dimana 2 proses atau lebih, saling menunggu proses untuk melepaskan sumber daya atau resources yang sedang digunakan. Mudahnya, ada proses A yang memperlukan suatu resources, tetapi resources tersebut sedang digunakan oleh proses lain. Untuk lebih paham tentang definisi deadlock dan cara mengatasinya, anda dapat membandingkan dengan situasi berikut. yang pertama, Dalam kehidupan, tentunya anda membutuhkan pekerjaan. Untuk memperoleh pekerjaan, anda harus mempunyai pengalaman, untuk mempunyai pengalaman anda harus bekerja.
	
\subsection {Masalah Deadlock dan Metode Penanganan Deadlock}
\subsubsection {Masalah Deadlock}
	Deadlock merupakan dampak pengaruh dari sinkronisasi, yaitu dimana satu variabel yang digunakan oleh dua proses yang berbeda. Deadlock selalu tidak terlepas dari yang namanya sumber daya, karena hampir secara keseluruhan merupakan masalah mengenai sebuah sumber daya yang digunakan secara bersamaan. Sebuah Kelompok Proses yang diblok atau diblokir, dimana setiap proses memegang sebuah resource dan kemudian menunggu resource lain dari proses yang berada didalam proses yang sedang diBlok tersebut, biasanya dari semua proses-proses atau resource yang non preemptive.
	
