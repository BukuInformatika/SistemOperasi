\section{Sistem Operasi Open Source}
\subsection{Definisi}
	Sistem Operasi Open Source yaitu sebuah sistem operasi yang source code dapat dibuka bebas oleh pengembangnya sehingga dapat dipelajari, diubah, dikembangkan, dan disebarluaskan lebih lanjut oleh setiap orang.

\subsection{Sejarah Sistem Operasi Open Source}
	Open source pertama kali digagas oleh  Eric S. Raymond, Christine Peterson, Todd Anderson, Larry Agustin, Jon Hall, dan Sam Ockman, yang dipimpin langsung oleh Richard Stallman pada tahun 1998. ini lah awal dari terbentuknya sistem operasi linux yang kita kenal saat ini.
\subsection{macam-macam sistem operasi open source}
	\begin{itemize}
		\item UNIX
			UNIX merupakan awal dari sebuah sistem operasi linux, UNIX bermula dari sebuah project Multics (Multiplexed Information and Computing Service) pada tahun 1965. Namun, UNIX ini sudah jarang digunakan saat ini dikarenakan sistem operasi ini sangat rumit untuk pemula.
		\item BSD (Berkeley Software Distribution)
			Sistem operasi ini hampir mirip dengan UNIX akan tetapi free BSD ini  bukan merupakan turunan dari UNIX melainkan sistem operasi yang dikembangkan oleh Berkeley Software Distribution.
		\item GNU Linux
			Linux merupakan sistem operasi yang banyak digunakan saat ini selain dari windows dan mac os, namun untuk di indonesia sendiri penggunaan  masih sedikit dikarenakan penggunaan tidak se user friendly windows ataupun mac os.
\end{itemize}
\subsection{Kelebihan dan Kekurangan Open Source}
\begin{enumerate}
\item Kelebihan
\begin{itemize}
\item User atau pengguna memiliki kebebasan dalam pengembangan sistem
\item tidak melanggar hak cipta/legal
\item kesalahan Bugs atau error lebih cepat ditangani atau diperbaiki
\item karena free/atau gratis sehigga tidak akan ada versi bajakan pada sistem operasi open source
\item kulitas sistem operasi lebih terjamin karena banyak orang mengevaluasi dan quality control
\item sistem operasi lebih stabil dan mudah digunakan
\item lebih aman karena lebih tahan dari serangan virus
\end{itemize}

\item Kekurangan
\begin{itemize}
\item kurang Sumber Daya Manusia (SDM) yang dapat memanfaatkan open source ini
\item interface kurang user friendly, terkadang tidak mudah dipahami oleh orang kebanyakan orang
\item kompatibilitas hardware rendah, ini merupakan kelemahan dari opensource kompatibilitas pada hardware dan perangkat peripheral.karena hal itu maka pengembang harus menyesuaikan sistem dan aplikasi yang bisa digunakan
\end{itemize}

\end{enumerate}

\section{UNIX}
\subsection{pengertian}
	Unix merupakan sistem operasi yang digunakan sebaga sistem operasi baku pada berbagai jenis komputer terutama pada komputer mini baik sebagai workstation ataupun server.
\subsection{sejarah}
	Unix adalah sebuah sistem operasi komputer yang dikembangkan oleh AT\&T Bell Labs pada tahun 1960 dan 1970-an. Pada tahun 1960, Massachusetts Institute of Technology, AT\&T Bell Labs, and General Electric bekerja dalam sebuah sistem operasi eksprimental yang disebut Multics (Multiplexed Information and Computing Service).
	Di Indonesia Unix digunakan sebagai Server aplikasi, produk yang beredar di pasaran antara lain IBM AIX, HP UX, Sun Solaris.
\subsection{jenis-jenis Unix}

\begin{enumerate}
\item A/UX
\item Domain/X
\item Darwin
\item CTIX
\item Distrix
\item UniCOS
\item DG/UX
\item Digital UNIX
\item Ultrix
\item CLIX
\item Dynix
\item SINIX
\item IRIX
\item SunOS
\item Solaris
\item Eunice
\item Uniplus+
\item BSD UNIX
\item BSD/I
\item OSF/1
\item GNU/Linux
\item GNU/Hurd
\item FreeBSD
\item NetBSD
\item OpenBSD
\item NextStep
\item Minix
\item Mach
\item UNIX System V
\item QNX
\end{enumerate} 

\section{BSD}
\subsection{Pengertian}
	BSD atau Berkeley Software Distribution, merupakan sistem operasi tingkat lanjut namun bukan turunan dari UNIX melainkan sistem operasi yang dikembangkan oleh Berkeley Software Distribution.
\subsection{sejarah}
	Pada tahun 70an Ken Thompson memperkenalkan sistem operasi Unix di University of California di Berkeley. Lalu di tahun 1978 mahasiswa tersebut memulai pembuatan custom UNIX release, dan di tahun 1980 Berkeley menandatangani kontrak kerjasama dengan DOD (Departmen of Defense) untuk masalah penggunaan TCP/IP pada BSD yang menghasilkan \textit{standard operating system} untuk komputer di departmen tersebut, yang dikenal sebagai Net2 namun sistem operasi ini hampir mirip dengan kode AT\&T sehingga banyak orang yang salah paham akan hal itu. Pada tahun 1982 BSD kembali dimana Bill Jolith pertama kali mengumumkan mengenai keinginannya membuat BSD versi free atau yang kita kenal sekarang dengan nama FreeBSD.

\section{GNU Linux}
\subsection{pengertian}
	Linux merupakan tiruan \textit{(clone)} dari sistem operasi unix, namun tidak terdapat kode program unix yang di tiru sendiri yaitu  menulis ulang berdasarkan standar POSIX berupa True - multitasking, Virtual memory, shared libraries, demand - loading, proper memory management dan multi user.
\subsection{sejarah}
	Linux pertama kali dibuat oleh LINUS TORVALDS, di universitas helsinki, finlandia	dimana berawal dari proyek hobi. Proyek hobi ini mendapatkan inspirasi dari MINIX yang mana merupakan sistem UNIX kecil yang dikembangkan oleh Andy Tanenbaum. Linux versi 0.01 dikerjakan sekitar bulan Agustus tahun 1991, yang diikuti pada tanggal 5 oktober 1991 diumumkannya LINUX versi resmi yaitu versi 0.02 yang dapat menjalankan bash (GNU Bourne Again Shell) dan GCC(GNU C COmpiler).