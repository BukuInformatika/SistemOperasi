\section{Sistem Operasi Open Source}
\subsection{Definisi}
	Sistem Operasi Open Source yaitu sebuah sistem operasi yang source code dapat dibuka bebas oleh pengembangnya sehingga dapat dipelajari, diubah, dikembangkan, dan disebarluaskan lebih lanjut oleh setiap orang.

\subsection{Sejarah Sistem Operasi Open Source}
	Open source pertama kali digagas oleh  Eric S. Raymond, Christine Peterson, Todd Anderson, Larry Agustin, Jon Hall, dan Sam Ockman, yang dipimpin langsung oleh Richard Stallman pada tahun 1998. ini lah awal dari terbentuknya sistem operasi linux yang kita kenal saat ini.
\subsection{macam-macam sistem operasi open source}
	\begin{itemize}
		\item UNIX
			UNIX merupakan awal dari sebuah sistem operasi linux, UNIX bermula dari sebuah project Multics (Multiplexed Information and Computing Service) pada tahun 1965. Namun, UNIX ini sudah jarang digunakan saat ini dikarenakan sistem operasi ini sangat rumit untuk pemula.
		\item BSD (Berkeley Software Distribution)
			Sistem operasi ini hampir mirip dengan UNIX akan tetapi free BSD ini  bukan merupakan turunan dari UNIX melainkan sistem operasi yang dikembangkan oleh Berkeley Software Distribution.
		\item GNU Linux
			Linux merupakan sistem operasi yang banyak digunakan saat ini selain dari windows dan mac os, namun untuk di indonesia sendiri penggunaan  masih sedikit dikarenakan penggunaan tidak se user friendly windows ataupun mac os.
\end{itemize}
\subsection{Kelebihan dan Kekurangan Open Source}
\begin{enumerate}
\item Kelebihan
\begin{itemize}
\item User atau pengguna memiliki kebebasan dalam pengembangan sistem
\item tidak melanggar hak cipta/legal
\item kesalahan Bugs atau error lebih cepat ditangani atau diperbaiki
\item karena free/atau gratis sehigga tidak akan ada versi bajakan pada sistem operasi open source
\item kulitas sistem operasi lebih terjamin karena banyak orang mengevaluasi dan quality control
\item sistem operasi lebih stabil dan mudah digunakan
\item lebih aman karena lebih tahan dari serangan virus
\end{itemize}

\item Kekurangan
\begin{itemize}
\item kurang Sumber Daya Manusia (SDM) yang dapat memanfaatkan open surce ini
\item interface kurang user friendly, terkadang tidak mudah dipahami oleh orang kebanyakan orang
\item kompatibilitas hardware rendah, ini merupakan kelemahan dari opensource kompatibilitas pada hardware dan perangkat peripheral.karena hal itu maka pengembang harus menyesuaikan sistem dan aplikasi yang bisa digunakan
\end{itemize}

\end{enumerate}

