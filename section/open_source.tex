\section{Sistem Operasi Open Source}
\subsection{Definisi}
	Sistem Operasi Open Source yaitu sebuah sistem operasi yang source code dapat dibuka bebas oleh pengembangnya sehingga dapat dipelajari, diubah, dikembangkan, dan disebarluaskan lebih lanjut oleh setiap orang.

\subsection{Sejarah Sistem Operasi Open Source}
	Open source pertama kali digagas oleh  Eric S. Raymond, Christine Peterson, Todd Anderson, Larry Agustin, Jon Hall, dan Sam Ockman, yang dipimpin langsung oleh Richard Stallman pada tahun 1998. ini lah awal dari terbentuknya sistem operasi linux yang kita kenal saat ini.
\subsection{macam-macam sistem operasi open source}
	\begin{itemize}
		\item UNIX
			UNIX merupakan awal dari sebuah sistem operasi linux, UNIX bermula dari sebuah project Multics (Multiplexed Information and Computing Service) pada tahun 1965. Namun, UNIX ini sudah jarang digunakan saat ini dikarenakan sistem operasi ini sangat rumit untuk pemula.
		\item BSD (Berkeley Software Distribution)
			Sistem operasi ini hampir mirip dengan UNIX akan tetapi free BSD ini  bukan merupakan turunan dari UNIX melainkan sistem operasi yang dikembangkan oleh Berkeley Software Distribution.
		\item GNU Linux
			Linux merupakan sistem operasi yang banyak digunakan saat ini selain dari windows dan mac os, namun untuk di indonesia sendiri penggunaan  masih sedikit dikarenakan penggunaan tidak se user friendly windows ataupun mac os.
\end{itemize}