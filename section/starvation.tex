\section{Starvation}
Perkembangan sistem komputer mendatang adalah menuju ke sistem multi-processing, multiprogramming, terdistribusi dan paralel yang mengharuskan adanya proses-proses yang berjalan bersama dalam waktu yang bersamaan. Hal demikian merupakan masalah yang perlu perhatian dari perancang sistem operasi. Kondisi dimana pada saat yang bersamaan terdapat lebih dari satu proses disebut dengan kongkurensi (proses-proses yang kongkuren). Dan dalam kongruensi ini pasti ada masalah yang salah satunya adalah STARVATION.
Kondisi dimana pada saat yang bersamaan terdapat lebih dari satu proses disebut dengan kongkurensi (proses-proses yang kongkuren). Dan dalam kongruensi ini pasti ada masalah yang salah satunya adalah STARVATION. Starvation adalah kondisi yang biasanya terjadi setelah deadlock. Proses yang kekurangan resource (karena terjadi deadlock) tidak akan pernah mendapat resource yang dibutuhkan sehingga mengalami starvation (kelaparan).Ilustrasi starvation dengan deadlock seperti pada gambar di bawah ini.

