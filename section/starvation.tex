\section{Starvation} 
Suatu kondisi yang biasanya terjadi setelah deadlock. Progres deadlock yang terjadi dapat mengakibatkan kekurangan resource, maka  yang akan terjadi deadlock tidak akan pernah mendapat resource yang dibutuhkan
sehingga dapat mengakibatkan suatu kejadian, yaitu �starvation� atau kelaparan. Namun, starvation juga bisa terjadi tanpa deadlock.
Hal ini ketika terdapat kesalahan dalam sistem sehingga terjadi ketimpangan dalam pembagian resouce. Suatu proses selalu mendapatkan resource, sedangkan proses yang lain tidak pernah mendapatkannya
tarvation adalah kondisi yang biasanya terjadi setelah deadlock. Proses yang kekurangan resource (karena terjadi deadlock) tidak akan pernah mendapat resource yang dibutuhkan sehingga mengalami starvation (kelaparan). Namun, starvation juga bisa terjadi tanpa deadlock. Hal ini ketika terdapat kesalahan dalam sistem sehingga terjadi ketimpangan dalam pembagian resouce. Satu proses selalu mendapat resource, sedangkan proses yang lain tidak pernah mendapatkannya

\section{Algoritma Starvation}
Starvation terjadi pada proses proses penjadwalan yang menggunakan prinsip �gproses yang paling cepat diselesaikan didahulukan�h, seperti pada Shortest Job First atau yang biasa di singkat SJF dan Penjadwalan Prioritas.
Logikanya, Misalkan saya mempunyai banyak sekali kebutuhan, saya akan memilihnya mana yang didahulukan berdasarkan sesuatu.
