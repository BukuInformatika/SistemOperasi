% Kelompok : 5
% Kelas : D4 TI 1A
% Anggota : 
% 1. Harun Ar - Rasyid 	1174027
% 2. Choirul Anam 		1174004
% 3. M.Tomy N.M.		1174031
% 4. Izza				1174013
% 5.Putra				


Artikel ini mengenai segmentation

  \begin{figure}[ht]
  \centerline{\includegraphics[width=1\textwidth]{..figures/segmentation.jpg}}
  \caption{Contoh segmentation}
  \label{segmentation}
  \end{figure}

\section{Pengertian Segmentation}
ini adalah contoh segmentation \ref{segmentation}
Segmentasi adalah teknik manajemen memori di mana setiap pekerjaan dibagi menjadi beberapa segmen dengan ukuran yang berbeda, satu untuk setiap modul yang berisi potongan-potongan yang melakukan fungsi terkait. Setiap segmen sebenarnya merupakan ruang alamat logis yang berbeda dari program. Ketika sebuah proses akan dieksekusi, segmentasinya yang sesuai akan dimuat ke dalam memori yang tidak bersebelahan meskipun setiap segmen dimuat ke dalam blok memori yang berdekatan dengan segmen lainnya.
Segmentasi manajemen memori bekerja sangat mirip dengan paging tetapi segmen di sini adalah variabel-panjang di mana seperti dalam halaman paging adalah ukuran tetap.
Segmen program berisi fungsi utama program, fungsi utilitas, struktur data, dan sebagainya. Sistem operasi memelihara tabel peta segmen untuk setiap proses dan daftar blok memori bebas bersama dengan nomor segmen, ukurannya dan lokasi memori yang sesuai dalam memori utama. Untuk setiap segmen, tabel menyimpan alamat awal segmen dan panjang segmen.

Segmen biasanya cocok atau sesuai dengan divisi alami dari suatu program seperti rutinitas individu atau tabel data sehingga segmentasi umumnya lebih terlihat oleh programmer daripada paging saja.  Segmen yang berbeda dapat dibuat untuk modul program yang berbeda, atau untuk kelas yang berbeda dari penggunaan memori seperti kode dan segmen data. 