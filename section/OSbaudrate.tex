%Nama Kelompok: Sistem_Operasi_BaudRate
%Kelas: D4 TI 1B
%Anggota : 
%Kevin Natanael Nainggolan(1174059) 
%Luthfi Muhammad Nabil(1174035)
%Salwaa Tania(1174047)
%Surya Pandu Prananda(1174036)

\section{Konsep Baud Rate}
Komunikasi secara berturut-turut sudah tidak asing lagi di era teknologi ini, salah satunya dikarenakan jumlah penghantar yang digunakan bisa lebih efektif daripada melakukannya secara sejajar. Mengapa demikian? Karena kata Berturut-turut berarti mengirim satu bit data dan selanjutnya yang diikuti oleh bit-bit data yang lain pada jalur yang sama. Karena itulah kita dapat meringkas penggunaan kabel. Dikarenakan jalur yang dilalui bersamaan, maka kecepatan komunikasi berturut-turut tidak secepat kecepatan komunikasi sejajar. Komunikasi sejajar, dapat mengirim data secara bersamaan melalui beberapa jalur. Namun, untuk proses secara keseluruhan, sistem komunikasi berturut-turut memenuhi berbagai aplikasi microcontroler. Selain itu, sistem komunikasi berturut-turut sering digunakan pada modem, USB, RS-232, dan teman-temannya.
