% Nama Kelompok : Kelompok 3
% Kelas : D4 TI 1A
% 1. Kadek Diva Krishna Murti (1174006)
% 2. Niko
% 3. Rizal Rony Sitorus
% 4. Jeremia
% 5. Sri Rahayu (1174015)

\section{Jenis-Jenis Fragmentasi}
\subsection{Fragmentasi Internal}
Berdasarkan buku yang berjudul Sistem Operasi \cite{pangera2005sistem} fragmentasi internal adalah tidak efisiennya utilitas memori utama di mana sembarang program sekecil apapun tidak bergantung, akan menempati seluruh partisi. Keadaan ini mengakibatkan pemborosan ruang yang bersifat internal terhadap partisi dengan kenyataan di mana blok data yang dimuatkan berukuran lebih kecil dari partisi.


\subsection{Fragmentasi Eksternal}
Berdasarkan buku yang berjudul Sistem Operasi \cite{pangera2005sistem} fragmentasi eksternal adalah keadaan di mana suatu proses pada awalnya berjalan dan bekerja dengan baik, namun ketika mengarah ke situasi yang di mana terdapat banyak lubang - lubang kecil di dalam sebuah memori sehingga semakin lama memori tersebut semakin terfragmentasi dan utilisasi memori pun kinerjanya menjadi menurun sehubungan dengan kenyataan bahwa memori bersifat eksternal terhadap semua partisi menjadi sangat terfragmentasi. Keadaan ini merupakan kebalikan dari fragmentasi ini.

Fragmentasi eksternal terjadi pada saat di mana jumlah keseluruhan memori kosong yang tersedia memang mencukupi untuk menampung permintaan tempat dari proses, tetapi tidak dapat dialokasikan karena letaknya tidak berkesinambungan atau tidak berurutan atau terpecah menjadi beberapa bagian kecil sehingga proses tidak dapat masuk. 
Umumnya, ini terjadi ketika kita menggunakan sistem partisi banyak dinamis. Pada sistem partisi banyak dinamis, seperti yang diungkapkan sebelumnya, sistem terbagi menjadi blok-blok yang besarnya tidak tetap.
 %%%%%%%% Kadek Diva Krishna Murti %%%%%%%%%%