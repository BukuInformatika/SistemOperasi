
\documentclass{wileySix}
\usepackage{w-bookps}

% \usepackage{mathptmx}

\usepackage{graphicx}
\usepackage{enumitem}

\usepackage{listings}

\usepackage{color}

\definecolor{codegreen}{rgb}{0,0.6,0}
\definecolor{codegray}{rgb}{0.5,0.5,0.5}
\definecolor{codepurple}{rgb}{0.58,0,0.82}
\definecolor{backcolour}{rgb}{0.95,0.95,0.92}

\lstdefinestyle{mystyle}{
    backgroundcolor=\color{backcolour},
    commentstyle=\color{codegreen},
    keywordstyle=\color{magenta},
    numberstyle=\tiny\color{codegray},
    stringstyle=\color{codepurple},
    basicstyle=\footnotesize,
    breakatwhitespace=false,
    breaklines=true,
    captionpos=b,
    keepspaces=true,
    numbers=left,
    numbersep=5pt,
    showspaces=false,
    showstringspaces=false,
    showtabs=false,
    tabsize=2,
    language=sh
}

\lstset{style=mystyle}

\setcounter{secnumdepth}{3}

\setcounter{tocdepth}{2}

\UseRawInputEncoding


\begin{document}

\booktitle{Sistem Operasi}
\subtitle{Semua Tentang Sistem Operasi}

\author{Rolly Maulana Awangga}

\halftitlepage
\titlepage


\offprintinfo{Sistem Operasi, pre-release}{Rolly Maulana Awangga}


\begin{copyrightpage}{2018}
Web Service / Rolly Maulana Awangga
\end{copyrightpage}


\dedication{For my family}

\begin{contributors}
\input{info/contributors}
\end{contributors}

\begin{foreword}
\input{info/foreword}
\end{foreword}



\begin{preface}

Buku ini ditujukan kepada mahasiswa yang ingin memahami tentang sistem operasi.



\prefaceauthor{R. M. Awangga}

\where{Bandung, Jawa Barat\\
Februari, 2019}
\end{preface}





\begin{acknowledgments}
\input{info/acknowledgments}
\end{acknowledgments}



\begin{acronyms}
\input{info/acronyms}
\end{acronyms}



\begin{glossary}
\input{info/glossary}
\end{glossary}



\begin{symbols}
\input{info/symbols}
\end{symbols}



\begin{introduction}
\input{info/introduction}
\end{introduction}


\contentsinbrief %optional
\tableofcontents
\listoffigures %optional
\listoftables  %optional

%%%%%%%%%
%%Content
%%%%%%%%%

\part[Pengenalan Sistem Operasi]
{Pengenalan\\ Sistem Operasi}

%\chapter[Contoh]
%{Contoh\\ Latex}
%tugas 1 :
Memory
https://www.tutorialspoint.com/operating_system/os_memory_management.htm
1A :
1. Process Address Space
2. Memory Allocation
3. Fragmentation
4. Paging
5. Segmentation


Pre requisite :
1. Tanda baca
2. Huruf kapital
3. SPOK

Parameter :
1. itemize dan enumerate yang benar (10)
2. gambar dan referensi disebutkan dalam kalimat dengan benar (10)
3. penggunaan section subsection subsubsection yang benar (10)
4. Penggunaan tabel atau verbatim atau equation (10)
5. commit sehari(min 50 kata) per anggota kelompok selama 6 hari (60)

Nilai akhir X persentasi plagiarisme = Nilai tugas

Sebelum memulai pekerjaan update terlbih dahulu :
git remote add upstream git@github.com:BukuInformatika/WebService.git
git fetch upstream
git pull upstream master

jika ada error maka minus 5 setiap kali pull request

\chapter[Sistem Operasi]
{OS\\ Sistem Operasi}
\section{Sistem Operasi}
	\subsection{Definisi}
		\begin{enumerate}
			Sistem operasi (OS) adalah sistem perangkat lunak yang mengelola hardware dan sumber daya perangkat lunak dengan menyediakan pelayanan secara umum untuk program komputer.
			Sistem operasi berbagi waktu menjadwalkan tugas untuk penggunaan sistem yang efisien dan mungkin juga termasuk perangkat lunak akuntansi untuk alokasi biaya waktu prosesor, penyimpanan massal, pencetakan, dan sumber daya lainnya.
			Untuk fungsi perangkat keras seperti input dan output dan alokasi memori, sistem operasi bertindak sebagai perantara antara program dan perangkat keras komputer, meskipun kode aplikasi biasanya dijalankan langsung oleh perangkat keras dan sering membuat panggilan sistem ke fungsi OS atau terganggu oleh saya t. Sistem operasi banyak ditemukan pada perangkat komputer,telepon seluler dan konsol permainan video ke server web dan superkomputer.
			Sistem operasi desktop yang dominan adalah Microsoft Windows dengan pangsa pasar sekitar 82,74%. macOS oleh Apple Inc. berada di tempat kedua (13,23%), dan varietas Linux secara kolektif berada di tempat ketiga (1,57%). Di sektor seluler (gabungan ponsel dan tablet), penggunaan pada tahun 2017 adalah hingga 70% dari Google Android dan menurut data kuartal ketiga 2016, Android pada smartphone dominan dengan 87,5 persen dan tingkat pertumbuhan 10,3 persen per tahun, diikuti oleh Apple iOS dengan 12,1 persen dan penurunan per tahun di pangsa pasar 5,2 persen, sementara jumlah sistem operasi lainnya hanya 0,3 persen. Distribusi Linux dominan di sektor server dan superkomputer. Kelas khusus lainnya dari sistem operasi, seperti embedded dan sistem real-time, ada untuk banyak aplikasi.
		\end{enumerate}
	\begin{figure}[ht]
		\centerline{\includegraphics[width=1\textwidth]{figures/OS.JPG}}
		\caption{Sistem Operasi}
		\label{OS}
	\end{figure}

\subsection{Jenis sistem operasi}
\subsubsection{Single dan multi-tasking}
	\begin{enumerate}
		Sistem satu tugas hanya dapat menjalankan satu program dalam satu waktu, sementara sistem operasi multi-tasking memungkinkan lebih dari satu program berjalan dalam konkurensi. Ini dicapai dengan time-sharing, di mana waktu prosesor yang tersedia dibagi antara beberapa proses. Proses ini masing-masing terganggu berulang kali dalam irisan waktu oleh subsistem penjadwalan tugas dari sistem operasi. Multi-tasking dapat dicirikan dalam tipe preemptif dan kooperatif. Dalam preemptive multitasking, sistem operasi memotong waktu CPU dan mendedikasikan slot untuk masing-masing program. Sistem operasi mirip Unix, seperti Solaris dan Linux — serta non-Unix-like, seperti AmigaOS — mendukung multitasking preemptif. Multitasking kooperatif dicapai dengan mengandalkan pada setiap proses untuk menyediakan waktu untuk proses lain dengan cara yang ditentukan. Versi 16-bit Microsoft Windows menggunakan multi-tasking kooperatif. Versi 32-bit dari Windows NT dan Win9x, menggunakan preemptive multi-tasking.
	\end{enumerate}
\subsection{Single dan multi-user}
	\begin{enumerate}
		\item 1. Sistem operasi pengguna tunggal tidak memiliki fasilitas untuk membedakan pengguna, tetapi dapat memungkinkan beberapa program berjalan bersama-sama. Sistem operasi multi-pengguna memperluas konsep dasar multi-tasking dengan fasilitas yang mengidentifikasi proses dan sumber daya, seperti ruang disk, milik beberapa pengguna, dan sistem memungkinkan banyak pengguna untuk berinteraksi dengan sistem pada saat yang bersamaan. 
		\item 2. Sistem operasi berbagi waktu menjadwalkan tugas untuk penggunaan sistem yang efisien dan mungkin juga termasuk perangkat lunak akuntansi untuk alokasi biaya waktu prosesor, penyimpanan massal, pencetakan, dan sumber daya lainnya untuk banyak pengguna.
	\end{enumerate}
\subsection{Pendistribusian}
	\begin{enumerate}
		Sistem operasi terdistribusi mengelola sekelompok komputer yang berbeda dan membuatnya tampak sebagai komputer tunggal. Pengembangan jaringan komputer yang dapat dihubungkan dan berkomunikasi satu sama lain menimbulkan komputasi terdistribusi. Komputasi terdistribusi dilakukan pada lebih dari satu mesin. Ketika komputer dalam kelompok bekerja dalam kerja sama, mereka membentuk sistem terdistribusi.
	\end{enumerate}
\subsection{Sejarah}
	\begin{enumerate}
		Komputer awal dibangun untuk melakukan serangkaian tugas tunggal, seperti kalkulator. Fitur-fitur sistem operasi dasar dikembangkan pada tahun 1950-an, seperti fungsi monitor penduduk yang dapat secara otomatis menjalankan program yang berbeda secara berurutan untuk mempercepat pemrosesan. Sistem operasi tidak ada dalam bentuk modern dan lebih kompleks hingga awal 1960-an. Fitur perangkat keras ditambahkan, yang memungkinkan penggunaan pustaka runtime, interupsi, dan pemrosesan paralel. Ketika komputer pribadi menjadi populer pada tahun 1980-an, sistem operasi dibuat untuk mereka yang serupa dalam konsep untuk yang digunakan pada komputer yang lebih besar.
		Pada 1940-an, sistem digital elektronik paling awal tidak memiliki sistem operasi. Sistem elektronik saat ini diprogram pada deretan switch mekanis atau oleh kabel jumper pada papan steker. Ini adalah sistem tujuan khusus yang, misalnya, menghasilkan tabel balistik untuk militer atau mengontrol pencetakan cek gaji dari data pada kartu kertas berlubang. Setelah komputer tujuan umum yang dapat diprogram diciptakan, bahasa mesin (terdiri dari string digit biner 0 dan 1 pada pita kertas berlubang) diperkenalkan yang mempercepat proses pemrograman.
		Pada awal 1950-an, komputer hanya dapat menjalankan satu program dalam satu waktu. Setiap pengguna memiliki satu-satunya penggunaan komputer untuk jangka waktu terbatas dan akan tiba pada waktu yang dijadwalkan dengan program dan data pada kartu kertas berlubang atau pita berlubang. Program akan dimuat ke dalam mesin, dan mesin akan diatur untuk bekerja sampai program selesai atau crash. Program umumnya dapat di-debug melalui panel depan menggunakan saklar beralih dan lampu panel. Dikatakan bahwa Alan Turing adalah seorang ahli dalam mesin Manchester Mark 1 awal ini, dan dia sudah mendapatkan konsep primitif dari sistem operasi dari prinsip-prinsip mesin Turing universal.
		Kemudian mesin datang dengan perpustakaan program, yang akan dikaitkan dengan program pengguna untuk membantu dalam operasi seperti input dan output dan menghasilkan kode komputer dari kode simbolik yang dapat dibaca manusia. Ini adalah awal dari sistem operasi modern. Namun, mesin masih menjalankan pekerjaan tunggal pada suatu waktu. Di Cambridge University di Inggris, antrian pekerjaan pada suatu waktu adalah garis pencucian (garis pakaian) dari mana kaset digantung dengan pakaian warna berbeda untuk menunjukkan prioritas pekerjaan. 
	\end{enumerate}
\subsection{Mikrokomputer}
	\begin{enumerate}
		Mikrokomputer pertama tidak memiliki kapasitas atau kebutuhan untuk sistem operasi yang rumit yang telah dikembangkan untuk mainframe dan miniis, sistem operasi minimalis dikembangkan, sering dimuat dari ROM dan dikenal sebagai monitor. Salah satu sistem operasi disk awal yang terkenal adalah CP / M, yang didukung oleh banyak mikrokomputer awal dan sangat ditiru oleh Microsoft MS-DOS, yang menjadi sangat populer sebagai sistem operasi yang dipilih untuk PC IBM . Pada 1980-an, Apple Computer Inc.  meninggalkan seri mikrokomputer Apple II yang populer untuk memperkenalkan komputer Apple Macintosh dengan antarmuka pengguna grafis inovatif  ke Mac OS.
		Pengenalan chip CPU Intel 80386 pada bulan Oktober 1985,  dengan arsitektur 32-bit dan kemampuan paging, menyediakan komputer pribadi dengan kemampuan untuk menjalankan sistem operasi multitasking seperti komputer minikomputer dan mainframe sebelumnya. Microsoft merespon perkembangan ini dengan mempekerjakan Dave Cutler, yang telah mengembangkan sistem operasi VMS untuk Digital Equipment Corporation. Dia dijadikan pemimpin dalam proses  pengembangan sistem operasi Windows NT, yang terus-menerus berfungsi sebagai dasar untuk jalur/way sistem operasi Microsoft. Steve Jobs, co-founder Apple Inc., memulai NeXT Computer Inc., yang mengembangkan sistem operasi NEXTSTEP. NEXTSTEP nantinya akan diakuisisi oleh Apple Inc. dan digunakan, bersama dengan kode dari FreeBSD sebagai inti dari Mac OS X.
		Proyek GNU dimulai oleh aktivis dan programmer Richard Stallman dengan tujuan menciptakan penggantian perangkat lunak gratis yang lengkap ke sistem operasi UNIX yang berpemilik. Meskipun proyek ini sangat berhasil dalam menduplikasi fungsionalitas berbagai bagian UNIX, pengembangan kernel GNU Hurd terbukti tidak produktif. Pada tahun 1991, mahasiswa ilmu komputer Finlandia Linus Torvalds, dengan kerja sama dari sukarelawan yang berkolaborasi melalui Internet, merilis versi pertama dari kernel Linux. Itu segera bergabung dengan komponen ruang pengguna GNU dan perangkat lunak sistem untuk membentuk sistem operasi yang lengkap. Sejak itu, kombinasi dari dua komponen utama biasanya hanya disebut Linux oleh industri perangkat lunak, konvensi penamaan yang Stallman dan Free Software Foundation tetap lawan, lebih memilih nama GNU / Linux. Berkeley Software Distribution, adalah bagian dari UNIX yang dirancang oleh University of California, Berkeley, dimulai pada thn 1970-an. Di distribusikan secara bebas dan diberikan ke banyak minikomputer, akhirnyapun mereka  memperoleh pengikut untuk digunakan pada PC, terutama sebagai FreeBSD, NetBSD dan OpenBSD.
	\end{enumerate}
\subsection{Berkeley Software Distribution}
	\begin{enumerate}
		Subkelompok keluarga Unix adalah keluarga Distribusi Perangkat Lunak Berkeley, yang mencakup FreeBSD, NetBSD, dan OpenBSD. Sistem operasi ini paling sering ditemukan di webservers, meskipun mereka juga dapat berfungsi sebagai OS komputer pribadi. Internet berutang banyak Kebijaksanaan untuk BSD, karena banyak dari protokol sekarang digunakan oleh komputer untuk menghubungkan, mengirim dan menerima data melalui jaringan diimplementasikan dan disempurnakan di BSD. World Wide Web Juga pertama kali dilakukan pada komputer yang menjalankan OS berdasarkan BSD yang disebut NeXTSTEP.
		Pada tahun 1974, University of California, Berkeley menciptakan sistem Unix awal. Seiring waktu, mahasiswa dan staf di departemen ilmu komputer telah mulai menambahkan program baru untuk menyederhanakan, seperti editor teks. Ketika Berkeley menerima komputer VAX baru pada tahun 1978 dengan Unixempat, para siswa di sekolah berbakat Unix lebih banyak menggunakan perangkat keras komputer. Departemen Pertahanan Advanced Defense Agency tertarik, dan memutuskan untuk mendanai proyek tersebut. Banyak sekolah, perusahaan, dan organisasi pemerintah menggunakan versi Berkeley dari Unix dan bukan yang resmi yang digunakan oleh AT & T.
		Steve Jobs, setelah Apple Inc. pada tahun 1985, membentuk NeXT Inc., perusahaan yang memproduksi komputer high-end yang berjalan di bawah kondisi yang sama seperti BSD yang disebut NeXTSTEP. Salah satu komputer ini oleh Tim Berners-Lee sebagai webserver pertama yang menciptakan World Wide Web.
		Pengembang seperti Keith Bostic mendorong proyek untuk mengurus kode non-bebas yang berasal dari Bell Labs. Setelah ini dilakukan, bagaimanapun, AT & T menuntut. Setelah dua tahun sengketa hukum, proyek BSD menghasilkan beberapa derivatif gratis, seperti NetBSD dan FreeBSD (keduanya pada tahun 1993), dan OpenBSD (dari NetBSD pada tahun 1995).
	\end{enumerate}
\subsection{Linux}
	\begin{enumerate}
		Kernel Linux berasal pada tahun 1991, sebagai proyek Linus Torvalds, sementara seorang mahasiswa di Finlandia. Dia memposting informasi tentang proyek-proyek di newsgroup untuk siswa komputer dan programer, dan Menerima dan membantu dari sukarelawan yang berhasil membuat kernel yang lengkap dan fungsional. Linux adalah Unix-like, tetapi dikembangkan tanpa kode Unix, tidak seperti BSD dan variannya. Karena model lisensi terbuka, kode kernel Linux tersedia untuk studi dan modifikasi, yang digunakan pada berbagai mesin dari superkomputer ke jam tangan pintar. Meskipun mereka menggunakan Linux pada 1,82% dari semua PC "desktop" (atau laptop), itu umumnya telah diadopsi untuk server embedded dan sistem seperti ponsel. Linux telah menemukan Unix pada banyak platform dan juga pada kebanyakan superkomputer termasuk 385 teratas. Banyak dari komputer yang sama juga menggunakan Green500 (tetapi dalam urutan yang berbeda), dan Linux berjalan di atas 10. Linux juga dapat digunakan pada perangkat kecil lainnya. komputer hemat energi, seperti smartphone dan jam pintar. Linux kernel dalam beberapa distribusi populer, seperti Red Hat, Debian, Ubuntu, Linux Mint dan Google Android, Chrome OS, dan Chromium OS.
	\end{enumerate}
\subsection{macOS}
	\begin{enumerate}
		macOS (sebelumnya Mac OS X dan kemudian OS X) juga merupakan sistem operasi grafis inti yang dikembangkan, dipasarkan dan dijual oleh Apple Inc., yang terbaru yang telah dimuat sebelumnya pada semua komputer Macintosh yang sedang dikirimkan. MacOS adalah penerus asli dari Mac OS klasik, yang telah menjadi sistem operasi utama Apple sejak 1984. Tidak seperti pendahulunya, ia telah dikembangkan di NeXT hingga tahun 1980-an dan sampai Apple membeli perusahaan tersebut awal tahun 1997. Mac OS X Server 1.0, diikuti pada Maret 2001 oleh versi klien (Mac OS X v10.0 Cheetah). Sejak itu, enam klien dan edisi server yang lebih baik dari mac OS telah dirilis, untuk hal yang sama di OS X 10.7 Lion. Sebelum bergabung dengan macOS, edisi server - MacOS Server - adalah sama seperti mitra desktop dan juga berjalan di lini perangkat Apple Macintosh. MacOS Server menyertakan manajemen grup dan perangkat lunak yang mencakup akses ke layanan jaringan utama, termasuk transfer surat, server Samba, server LDAP, server nama domain, dan banyak lagi. Dengan Mac OS X v10.7 Lion, semua aspek server Mac OS X Server telah diintegrasikan ke dalam versi klien dan produk tersebut bermerek kembali sebagai OS X (menjatuhkan "Mac" dari namanya). Alat server sekarang tersedia sebagai aplikasi.
	\end{enumerate}
\subsection{Microsoft Windows}
	\begin{enumerate}
		Microsoft Windows adalah sistem operasi yang dirancang oleh Microsoft Corporation dan khususnya untuk komputer arsitektur Intel, dengan dimensi 88,9 persen dari total pada komputer yang terhubung dengan Web. Versi terbaru adalah Windows 10.
		Pada 2011, Windows 7 mengambil alih Windows XP sebagai versi yang sangat umum. Microsoft Windows pertama kali dirilis pada tahun 1985, yang merupakan bagian dari MS-DOS, yang merupakan sistem operasi yang digunakan pada saat itu. Pada tahun 1995, Windows 95 dirilis hanya menggunakan MS-DOS sebagai bootstrap. Untuk menyelesaikan retret, Win9x dapat menjalankan driver MS-DOS dan Windows-3 Windows 16-bit yang nyata. Windows ME, dirilis pada tahun 2000, adalah versi terbaru dalam keluarga Win9x. Versi yang lebih baru sekarang didasarkan pada kernel Windows NT. Tanggung jawab Windows saat ini berjalan pada mikroprosesor ARM IA-32, x86-64, dan 32-bit. Selain itu Itanium masih didukung pada server lama versi Windows Server 2008 R2. Di masa lalu, Windows NT mendukung arsitektur tambahan. Server edisi Windows banyak digunakan. Dalam beberapa tahun terakhir, Microsoft telah mengeluarkan modal yang signifikan dalam upayanya ke Windows sebagai sistem operasi server. Namun, penggunaan Windows pada server tidak meluas seperti pada komputer pribadi karena Windows bersaing dengan Linux dan BSD untuk server pasar. ReactOS adalah sistem operasi Windows alternatif, yang sedang dikembangkan pada prinsip-prinsip Windows - tanpa menggunakan kode Microsoft.
	\end{enumerate}
	
	
\cite{silberschatz2014operating}
\cite{hoare1974monitors}
\cite{bach1986design}
\cite{love2005linux}
\cite{kukreja2006rui}
\cite{mckeown2009software}
\cite{russinovich2005microsoft}
\cite{van1994treecon}
\cite{mckusick1985performance}
\cite{higgins1988clustal}

\chapter[Proses OS]
{OS\\ Proses}
%kelompok 1 Sistem Operasi (Semaphore)
%Kelas D4 TI 1B
%Adam Noer Hidayatullah 1174097
%Ichsan Hizman
%Teddy
%Nisrina Aulia
%Irvan Rizkiansyah 1174043

\section{proses}

	\subsection{Proses}	
	Proses adalah sebuah  program yang sedang dieksekusi. Sedangkan program adalah kumpulan-kumpulan  suatu  instruksi yang sudah  ditulis ke dalam bahasa yang dimengerti sistem operasi.Proses berisi tentang  sebuah instruksi dan sebuah  data. program counter dan seluruh register pemroses, stack ini  berisi data sementara contoh  seperti parameter rutin, alamat pengiriman dan variabel lokal.Sistem operasi harus mengelola semua proses di dalam sistem tersebut dan mengalokasikan sumber daya ke sebuah  proses-proses sesuai dengan kebijaksanaan untuk memenuhi sasaran sistem
	
	\subsection{Istilah yang berkaitan dengan proses}
		\begin{itemize}
			\item Multiprogramming
			Multiprogramming (multitasking) adalah  istilah teknologi informasi dengan mengunakan bahasa inggris yang baik  mengacup kepada sebuah metode dimana banyak sebuah pekerjaan atau yang dikenal juga sebagai proses  dengan diolah dengan menggunakan sumber daya CPU yang sama.
			Contohnya sistem operasi jenis ini antaranya linux dan windows.
			\item Multiprocessing
			kemampuan komputer untuk melakukan beberapa proses dengan waktu yang bersamaan, dibantu dengan keberadaan teknologi yang berbasis multiprocessor.
			Contohnya seperti computer server.
			\item Distributed processing/computing
			Mengerjakan semua proses pengolahan data secara bersamaan antara komputer pusat dengan beberapa komputer yang lebih kecil dan saling berhubungan denan melalui jalur komunikasi.
			Contohnya komputer yang dirancang untuk melaksanakan tugas-tugas proyek.
		\end{itemize}
		
	\subsection{Status proses}
	Terdapat 5 macam jenis status yang mungkin dimiliki oleh suatu proses :
	\begin{enumerate}
		\item New, yaitu status yang dimiliki pada saat proses baru saja terjadi.
		\item Ready, yaitu status dimana proses siap untuk dieksekusi pada giliran berikutnya.
		\item Running, yaitu status dimana saat ini proses sedang dieksekusi oleh prosesor
		\item Waiting, yaitu status dimana proses yang tidak bisa dijalankan di saat prosesor sudah siap, status yang dimiliki pada saat proses menunggu suatu sebuah event seperti I/O
		\item Terminated, yaitu status yang dimiliki pada saat proses telah selesai dieksekusi
	\end{enumerate}
	
	Berikut ini adalah proses dari ke-5 status proses di atas :
	\begin{enumerate}
		\item New ke Ready
		Pertama Status dibuat lalu setelah itu , status akan memasuki proses ready dan siap untuk memasuki proses selanjutnya.
		\item Ready ke running
		Di saat sedang memilih proses yang akan dioperasikan, sistem operasi akan memilih salah satu proses yang berada didalam keadaan status ready.
		\item Running ke waiting
		Suatu proses dimasukkan dalam keadaan status waiting jika proses itu meminta sesuatu yang dapat menyebabkannya harus menunggu. Sebuah request ke sistem operasi pada umumnya merupakan bentuk panggilan dari layanan sistem (panggilan dari program yang sedang beroperasi ke prosedur yang sedang beroperasi ke prosedur yang merupakan bagian kode sistem operasi) misalnya seperti sebuah proses bisa meminta suatu layanan dari sistem operasi yang tidak siap dilakukan sistem opersi dengan segera. Atau proses dapat menginisiasi suatu aksi, misalnya operasi I/O, yang harus diselesaikan sebelum proses itu melanjutkan operasinya. Pada saat proses saling berkomunikasi dengan proses lainnya, suatu proses bisa diblokir jika sedang menunggu proses lainnya untuk menyediakan input atau sedang menunggu pesan dari proses lainnya.
		
	\end{enumerate}	

\chapter[Open Sourcei]
{OS\\ Open Source}
\section{Sistem Operasi Open Source}
\subsection{Definisi}
	Sistem Operasi Open Source yaitu sebuah sistem operasi yang source code dapat dibuka bebas oleh pengembangnya sehingga dapat dipelajari, diubah, dikembangkan, dan disebarluaskan lebih lanjut oleh setiap orang.

\subsection{Sejarah Sistem Operasi Open Source}
	Open source pertama kali digagas oleh  Eric S. Raymond, Christine Peterson, Todd Anderson, Larry Agustin, Jon Hall, dan Sam Ockman, yang dipimpin langsung oleh Richard Stallman pada tahun 1998. ini lah awal dari terbentuknya sistem operasi linux yang kita kenal saat ini.

\chapter[Linux]
{OS\\ Arsitektur Sistem Operasi Linux}
\section{Arsitektur Sistem Operasi Linux}
\subsection{Kernel}
	Kernel Linux merupakan kernel yang digunakan dalam sistem operasi GNU/Linux. Kernel ini merupakan turunan dari sistem operasi UNIX, yang mana di rilis menggunakan lisensi GNU \textit{General Public License} (GPL) dan dikembangkan oleh programmer di selururh dunia karena sifatnya yang \textit{open source}. Kernel ini merupakan inti dari sistem operasi komputer dengan memiliki kontrol penuh atas segala dalam sistem tersebut. Kernel ini menghubungkan antara perangkat lunak dan perangkat keras seperti pada gambar \textbf{\ref{kernel}}, salah satu program pertama yang memuat fungsi kernel ini yaitu dimuat didalam start-up setelah proses bootloader.

\begin{figure}[!htbp]
\centerline{\includegraphics[width=0.75\textwidth]{Figures/Kernel_Layout.png}}
\caption{Kernel connect}
\label{kernel}
\end{figure}

\lstinputlisting[caption=contoh dasar kode program kerne membuat hello world,label={lst:kode dasar}]{src/contoh.tex}

\subsection{struktur data kernel}
	Ketika kernel melakukan sebuah proses, data-data proses tersebut akan disimpan secara periodik ke dalam bentuk file-file. Untuk dapat melihat data kernel, maka file-file tersebut harus diparsing setiap saat dikarenakan datanya yang dinamis \cite{raharja2001pengenalan}. Cara termudah untuk melakukan hal tersebut yaitu menggunakan perintah \textbf{cat} \ref{lst:kode dasar2}

\lstinputlisting[caption=Perintah cat pada linux,label={lst:kode dasar2}]{src/cat.sh}
File-file ini akan tersimpan di dalam direktori yang tersetruktur dalam direktori /proc.

\chapter[OS Semaphore]
{OS\\ Semaphore}
\section{System Operasi Semaphore}

	\subsection{Definisi}
	
		\begin{figure}[ht]
			\centerline{\includegraphics[width=0.5\textwidth]{figures/sema.png}}
			\caption{Semaphore}
			\label{sema}
			\end{figure}
	
		Semaphore pada system operasi merupakan sebuah variabel bertipe integer. Di kehidupan nyata, semaphore merupakan sistem sinyal yang digunakan untuk memberi sinyal atau tanda dan berkomunikasi secara visual. 
		Semafor juga merupakan struktur data dalam bahasa komputer yang digunakan untuk menyinkronkan suatu proses, yaitu untuk memecahkan masalah di mana masalahnya lebih dari satu proses atau bisa 
		seperti thread yang akan dijalankan secara bersamaan dan harus diatur urutan kerja. Semaphore dibuat oleh Edsger Dijkstra dan pertama kali digunakan dalam sistem operasi.
		Nilai semaphore diinisialisasi dengan jumlah sumber daya yang dikontrol oleh pengguna. Dalam kasus khusus di mana ada sumber daya bersama, semaphore disebut semaphore biner. 
		Semaphore adalah solusi klasik dari dining philosophers problem, meskipun itu tidak mencegah deadlock.
		Pada software, semaphore merupakan sebuah variabel bertipe data integer yang selain saat inisialisasi, yang hanya dapat diakses melalui dua operasi standar, yaitu increment dan decrement. 
		Semaphore bisa digunakan untuk menyelesaikan masalah sinkronisasi secara umum, berdasarkan jenisnya. Semaphore hanya memiliki nilai 1 atau 0, atau lebih dari sama dengan 0. 
		Konsep semaphore pertama kali diajukan idenya oleh Edsger Dijkstra pada tahun 1967. Semaphore memiliki dua jenis, yaitu, Biner semaphore dan counting semaphore. 
		Biner semaphore tidak bisa memiliki semua jenis integer tetapi hanya memiliki 2 nilai yaitu 1 atau 0, Sering juga disebut sebagai semaphore primitive. Sedangkan Counting semaphore memiliki nilai 0, 1, sampai seterusnya atau integer lainnya. 
		Banyak sistem operasi yang tidak secara langsung menggunakan semaphore jenis ini, namun lebih banyak yang memanfaatkan semaphore jenis biner semaphore. 
		Pada semaphore ini harus diketahui bahwa, ada beberapa jenis dari counting semaphore yang salah satu jenisnya adalah semafor yang tidak bisa mencapai nilai negatif dan jenis yang lain adalah semaphore yang dapat mencapai nilai negatif. 
		Solusi dari Pembuatan Counting Semaphore adalah Binary Semaphore. Pembuatan counting semaphore banyak dilakukan para programmer untuk memenuhi alat sinkronisasi yang sesuai dengannya. 
		
	\subsection{Prinsip Semaphore}
	
		\begin{enumerate}

			\item Suatu proses yang berbeda bisa berkaitan dengan memanfaatkan sinyal - sinyal.
			\item Suatu proses dapat dihentikan oleh 
			\item Semaphore bertipe data integer yang diakses oleh dua operasi atomik standar (wait dan signal).
			\item Ada dua operasi di semaphore (Down dan UP). Yang nama aslinya : P dan V.
		
		\end{enumerate}

	\subsection{Kelemahan Semaphore}
	
		\begin{enumerate}

			\item Semaphore termasuk Low Level
			\item Dikarenaka semaphore tersebar di dalam seluruh program maka kita akan kesulitan dalam pemeliharaannya.
			\item Jika kita menghapus \"wait\" akan mengakibatkan \"nonmutual exclusion\"
			\item Jika kita menghapus \"signal\" akan mengakibatkan \"deadlock\"
			\item Jika terjadi deadlock akan sulti untuk dideteksi.

		\end{enumerate}
		
	\subsection{Semantik dan Impelementasi}
		Menghitung semaphores dilengkapi dengan dua operasi, secara historis dilambangkan sebagai P dan V (lihat § Nama operasi untuk nama alternatif). Operasi V menambahkan semaphore S, dan operasi P menurunkannya.

		Nilai semaphore S adalah jumlah unit sumber daya yang saat ini tersedia. Operasi P membuang waktu atau tidur sampai sumber daya yang dilindungi oleh semaphore menjadi tersedia, pada saat itu sumber daya segera diklaim. Operasi V adalah kebalikannya: ia membuat sumber daya tersedia lagi setelah proses selesai menggunakannya. Satu properti penting dari semaphore S adalah bahwa nilainya tidak dapat diubah kecuali dengan menggunakan operasi V dan P.

		Cara sederhana untuk memahami operasi tunggu (P) dan sinyal (V) adalah:
		
		\begin{itemize}
		
			\item menunggu: Jika nilai variabel semaphore tidak negatif, turunkan dengan 1. Jika variabel semaphore sekarang negatif, proses menunggu eksekusi diblokir (yaitu, ditambahkan ke antrian semaphore) sampai nilainya lebih besar atau sama dengan 1 Jika tidak, proses terus berjalan, setelah menggunakan satu unit sumber daya.
			\item sinyal: Menambah nilai semaphore variabel dengan 1. Setelah kenaikan, jika nilai pre-increment negatif (berarti ada proses menunggu sumber daya), ia mentransfer proses yang diblokir dari antrian menunggu semaphore ke antrean siap.
			
		\end{itemize}
		
		Banyak sistem operasi menyediakan primitif semaphore yang efisien yang membuka blokir proses menunggu ketika semaphore bertambah. Ini berarti bahwa proses tidak membuang waktu untuk memeriksa nilai semaphore yang tidak perlu.Konsep penghitungan semaphore dapat diperpanjang dengan kemampuan untuk mengklaim atau mengembalikan lebih dari satu \"unit\" dari semaphore, teknik yang diterapkan di Unix. Operasi V dan P yang dimodifikasi adalah sebagai berikut, menggunakan tanda kurung siku untuk menunjukkan operasi atom, yaitu operasi yang tampak terpisah dari perspektif proses lain:

		Semaphore pada system operasi merupakan sebuah variabel bertipe integer. Di kehidupan nyata, semaphore merupakan sistem sinyal 		yang digunakan untuk memberi sinyal atau tanda dan berkomunikasi secara visual. Pada software, semaphore merupakan sebuah variabel bertipe data integer yang selain saat inisialisasi, yang hanya dapat diakses melalui dua operasi standar, yaitu increment dan decrement.
		Semaphore bisa digunakan untuk menyelesaikan masalah sinkronisasi secara umum, berdasarkan jenisnya. Semaphore hanya memiliki nilai 1 atau 0, atau lebih dari sama dengan 0. Konsep semaphore pertama kali diajukan idenya oleh Edsger Dijkstra pada tahun 1967. Semaphore memiliki dua jenis, yaitu, Biner semaphore dan counting semaphore. Biner semaphore hanya memiliki nilai 1 atau 0, Sering juga disebut sebagai semaphore primitive. Sedangkan Counting semaphore memiliki nilai 0, 1, sampai seterusnya atau integer lainnya. Banyak sistem operasi yang tidak secara langsung menggunakan semaphore jenis ini, namun lebih banyak yang memanfaatkan semaphore jenis biner semaphore

	
	\subsection{Prinsip Semaphore}
		\begin{enumerate}

			\item Suatu proses yang berbeda bisa berkaitan dengan memanfaatkan sinyal - sinyal.
			\item Suatu proses dapat dihentikan oleh 
			\item Semaphore bertipe data integer yang diakses oleh dua operasi atomik standar (wait dan signal).
			\item Ada dua operasi di semaphore (Down dan UP). Yang nama aslinya : P dan V.
			
		\end{enumerate}
		
	\subsection{Kelemahan Semaphore}

		\begin{enumerate}

			\item Semaphore termasuk Low Level.
			\item Dikarenaka semaphore tersebar di dalam seluruh program maka kita akan kesulitan dalam pemeliharaannya.
			\item Jika kita menghapus \"wait\" akan mengakibatkan \"nonmutual exclusion\".
			\item Jika kita menghapus \"signal\" akan mengakibatkan \"deadlock\".
			\item Jika terjadi deadlock akan sulti untuk dideteksi.
			
		\begin{enumerate}
		
	\cite{luu1982apparatus}
	\cite{lauesen1975large}
	\cite{hoare1974monitors}




%{OS\\ Proses OS}
%%kelompok 1 Sistem Operasi (Semaphore)
%Kelas D4 TI 1B
%Adam Noer Hidayatullah 1174097
%Ichsan Hizman
%Teddy
%Nisrina Aulia
%Irvan Rizkiansyah 1174043

\section{proses}

	\subsection{Proses}	
	Proses adalah sebuah  program yang sedang dieksekusi. Sedangkan program adalah kumpulan-kumpulan  suatu  instruksi yang sudah  ditulis ke dalam bahasa yang dimengerti sistem operasi.Proses berisi tentang  sebuah instruksi dan sebuah  data. program counter dan seluruh register pemroses, stack ini  berisi data sementara contoh  seperti parameter rutin, alamat pengiriman dan variabel lokal.Sistem operasi harus mengelola semua proses di dalam sistem tersebut dan mengalokasikan sumber daya ke sebuah  proses-proses sesuai dengan kebijaksanaan untuk memenuhi sasaran sistem
	
	\subsection{Istilah yang berkaitan dengan proses}
		\begin{itemize}
			\item Multiprogramming
			Multiprogramming (multitasking) adalah  istilah teknologi informasi dengan mengunakan bahasa inggris yang baik  mengacup kepada sebuah metode dimana banyak sebuah pekerjaan atau yang dikenal juga sebagai proses  dengan diolah dengan menggunakan sumber daya CPU yang sama.
			Contohnya sistem operasi jenis ini antaranya linux dan windows.
			\item Multiprocessing
			kemampuan komputer untuk melakukan beberapa proses dengan waktu yang bersamaan, dibantu dengan keberadaan teknologi yang berbasis multiprocessor.
			Contohnya seperti computer server.
			\item Distributed processing/computing
			Mengerjakan semua proses pengolahan data secara bersamaan antara komputer pusat dengan beberapa komputer yang lebih kecil dan saling berhubungan denan melalui jalur komunikasi.
			Contohnya komputer yang dirancang untuk melaksanakan tugas-tugas proyek.
		\end{itemize}
		
	\subsection{Status proses}
	Terdapat 5 macam jenis status yang mungkin dimiliki oleh suatu proses :
	\begin{enumerate}
		\item New, yaitu status yang dimiliki pada saat proses baru saja terjadi.
		\item Ready, yaitu status dimana proses siap untuk dieksekusi pada giliran berikutnya.
		\item Running, yaitu status dimana saat ini proses sedang dieksekusi oleh prosesor
		\item Waiting, yaitu status dimana proses yang tidak bisa dijalankan di saat prosesor sudah siap, status yang dimiliki pada saat proses menunggu suatu sebuah event seperti I/O
		\item Terminated, yaitu status yang dimiliki pada saat proses telah selesai dieksekusi
	\end{enumerate}
	
	Berikut ini adalah proses dari ke-5 status proses di atas :
	\begin{enumerate}
		\item New ke Ready
		Pertama Status dibuat lalu setelah itu , status akan memasuki proses ready dan siap untuk memasuki proses selanjutnya.
		\item Ready ke running
		Di saat sedang memilih proses yang akan dioperasikan, sistem operasi akan memilih salah satu proses yang berada didalam keadaan status ready.
		\item Running ke waiting
		Suatu proses dimasukkan dalam keadaan status waiting jika proses itu meminta sesuatu yang dapat menyebabkannya harus menunggu. Sebuah request ke sistem operasi pada umumnya merupakan bentuk panggilan dari layanan sistem (panggilan dari program yang sedang beroperasi ke prosedur yang sedang beroperasi ke prosedur yang merupakan bagian kode sistem operasi) misalnya seperti sebuah proses bisa meminta suatu layanan dari sistem operasi yang tidak siap dilakukan sistem opersi dengan segera. Atau proses dapat menginisiasi suatu aksi, misalnya operasi I/O, yang harus diselesaikan sebelum proses itu melanjutkan operasinya. Pada saat proses saling berkomunikasi dengan proses lainnya, suatu proses bisa diblokir jika sedang menunggu proses lainnya untuk menyediakan input atau sedang menunggu pesan dari proses lainnya.
		
	\end{enumerate}


\chapter[Sistem Keamanan Linux]
{Sistem Keamanan Linux}
\section{Sistem Keamanan Linux}
\subsection{SELinux}
	SELinux \textit{(Security Enhaced Linux)} yang mana merupakan salah satu peningkatan keamanan dari sebuah sistem operasi berbasiskan linux, keamanan yang dimaksud disini yaitu untuk membedakan antara user root dan juga user yang sifatnya terbatas atau memiliki hak akses masing-masing. Aplikasi mendasar dari SELinux ini adalah layanan FTP \textit{(File Transfer Protocol)} dan HTTP \textit{(Hyper Text Transfer Protocol)}.
SELinux ini memiliki 3 mode yaitu :
\begin{enumerate}
\item Enforcing, merupakan pengaturan keamanan yang paling ketat
\item Permissive, merupakan pengaturan keamanan yang longgar
\item Disabled, merupakan pengaturan untuk memayikan SELinux
\end{enumerate}

pada linux terdapat SELinux, di windows pun sebenarnya ada yang seperti itu namun dengan nama yang berbeda yaitu \textit{User Account Control} atau UAC berfungsi untuk menjalankan aplikasi atau membuat, mengedit dan menghapus program yang penting.


\chapter[Starvation]
{OS\\ Starvation}
\section{Starvation} 
Suatu kondisi yang biasanya terjadi setelah deadlock. Progres deadlock yang terjadi dapat mengakibatkan kekurangan resource, maka  yang akan terjadi deadlock tidak akan pernah mendapat resource yang dibutuhkan
sehingga dapat mengakibatkan suatu kejadian, yaitu �starvation� atau kelaparan. Namun, starvation juga bisa terjadi tanpa deadlock.
Hal ini ketika terdapat kesalahan dalam sistem sehingga terjadi ketimpangan dalam pembagian resouce. Suatu proses selalu mendapatkan resource, sedangkan proses yang lain tidak pernah mendapatkannya
tarvation adalah kondisi yang biasanya terjadi setelah deadlock. Proses yang kekurangan resource (karena terjadi deadlock) tidak akan pernah mendapat resource yang dibutuhkan sehingga mengalami starvation (kelaparan). Namun, starvation juga bisa terjadi tanpa deadlock. Hal ini ketika terdapat kesalahan dalam sistem sehingga terjadi ketimpangan dalam pembagian resouce. Satu proses selalu mendapat resource, sedangkan proses yang lain tidak pernah mendapatkannya

\section{Algoritma Starvation}
Starvation terjadi pada proses proses penjadwalan yang menggunakan prinsip �gproses yang paling cepat diselesaikan didahulukan�h, seperti pada Shortest Job First atau yang biasa di singkat SJF dan Penjadwalan Prioritas.
Logikanya, Misalkan saya mempunyai banyak sekali kebutuhan, saya akan memilihnya mana yang didahulukan berdasarkan sesuatu.

\section {Menghindari Starvation}
Ada beberapa cara untuk mengatasi Starvation, salah satunya dengan Aging,  proses awal yang ada diberi urutan
(N) pemrosesan dengan rumus N = ( P+T ) / P. N maksimum akan mulai dikerjakan dan proses yang lain dinaikkan tingkat urutan prosesnya agar nanti jika ada proses lain yang masuk, proses terdahulu mendapatkan bagian resource dan dapat dikerjakan.

Round Robin , adalah proses yang akan dimasukkan ke dalam antrian menurut proses kedatangannya. Dalam penyelesainnya, suatu proses tidak akan langsung selesai jika waktu yang dibutuhkan melebihi waktu kuantum yang diberikan.

Waktu kuantum itu sendiri adalah waktu yang telah diberikan untuk menyelesaikan suatu proses. Ketika suatu proses telah mencapai batas waktu kuantum, sisa dari proses tersebut dikembalikan ke antrian paling belakang dan sumber-sumber data dipindahkan ke proses selanjutnya. Maka dengan cara ini, semua proses yang mengantri akan mendapatkan sumber-sumber data secara bergantian ( tidak ada proses yang memonopoli resource ) sehingga semua proses dapat diselesaikan.

\section {Menghambat Starvation dengan Disclosed}
Disclosed adalah menghambat proces Starvation dalam Sistem Operasi multitasking dengan menyediakan tipe pertama dari event penjadwalan pada interval waktu periodik, menyediakan tipe kedua dari event penjadwalan kedua sebagai tanggapan atas proses yang berjalan. Secara sukarela melepaskan prosesor dan, sebagai tanggapan atas acara penjadwalan menggantikan proses lama dengan yang baru, jika proses lama telah berjalan selama lebih dari satu jumlah waktu yang telah ditentukan. Sistem dijelaskan di sini menyediakan kernel kecil yang dapat dijalankan pada berbagai platform perangkat keras, seperti berbasis PowerPC Papan adaptor Symmetrix digunakan dalam Penyimpanan data Symmetrix perangkat yang disediakan oleh EMC Corporation of Hopkinton, Mass. Kode inti kernel dapat ditulis untuk target umum platform, seperti arsitektur PowerPC. Sejak Pow Modul spesifik implementasi erPC didefinisikan dengan baik, sistem mungkin cukup portabel antara prosesor PowerPC (seperti 8260 dan 750),
 

\chapter[DeadLock]
{OS\\ deadlock}
%Nama Kelompok: Sistem_Operasi_Deadlock
%Kelas: D4 TI 1B
%Alit Fajar Kurniawan(1174057) 
%Muhammad Iqbal Panggabean(1174063)
%Muhammad Afra Faris(1174041)
%Khadijah Hasanah Puteri Harahap(1174044)

\section {DEADLOCK}

\subsection {Deadlock}
\subsubsection {Pengertian Deadlock}
	Pada kesempatan ini saya akan menjelaskan tentang definisi Deadlock, Deadlock ialah suatu keadaan yang dimana dua proses atau lebih, saling menunggu proses untuk dapat melepaskan sumber daya yang sedang dijalankan. Misalnya proses A yang memperlukan suatu sumber daya, tetapi sumber saya tersebut sedang digunkana oleh proses lain. Untuk lebih paham mengenai pengertian dari deadlock dan bagaimana cara mengatasinya, anda dapat membandingkannya dengan situasi yang satu ini. Pertama, Dalam kehidupan kita tentu membutuhkan suatu pekerjaan, dan untuk memperoleh suatu pekerjaan, anda harus memiliki pengalaman yang baik, untuk dapat memiliki pengalaman yang baik anda harus bekerja.

	\begin{figure}[ht]
	\centerline{\includegraphics[width=1\textwidth]{figures/deadlock1.jpg}}
	\caption{Gambar Deadlock}
	\label{Gambar}
	\end{figure}
      
      Gambar \ref{Gambar} Contoh gambar pada saat terjadinya deadlock.

\subsection {Masalah Deadlock dan Metode Penanganan Deadlock}
\subsubsection {Masalah Deadlock}
	Deadlock merupakan dampak pengaruh dari sinkronisasi, yaitu dimana satu variabel yang digunakan oleh dua proses yang berbeda. Deadlock selalu tidak terlepas dari yang namanya sumber daya, karena hampir secara keseluruhan merupakan masalah mengenai sebuah sumber daya yang digunakan secara bersamaan. Sebuah Kelompok Proses yang diblok atau diblokir, dimana setiap proses memegang sebuah resource dan kemudian menunggu resource lain dari proses yang berada didalam proses yang sedang diBlok tersebut, biasanya dari semua proses-proses atau resource yang non preemptive..
	
\subsubsection {Metode Penanganan}
	Ada tiga Metode penanganan Deadlock:
	Yang Pertama yaitu, anda harus menggunakan satu protokol yang dapat membuat anda yakin bahwa sistem tersebut tidak akan pernah mengalami kejadian deadlock. Metode ini bisa disebut dengan Deadlock Prevention atau Avoidance.
	
	Yang Kedua, anda harus memberikan izin sistem untuk mengalami kejadian deadlock, namun setelah terjadinya deadlock anda harus dengan cepat segera untuk memperbaiki sistem yang mengalami deadlock tersebut. Metode ini biasanya disebut dengan Deadlock detection and recovery.
	
	Dan yang terakhir, anda hanya mengabaikan semua permasalahan yang terjadi secara bersamaan, dan kemudian menganggap bahwa deadlock tidak akan terjadi, metode ini digunakan dalam berbagai sistem operasi komputer, termasuk windows dan unix.

\begin{table}[H]
\begin{tabular}{|c|c|c|}
\hline
Proses & Jumlah Sumber Daya Digenggam & Maksimum Sumber Daya Dibutuhkan\\
\hline
X   & 2 & 10\\
\hline
Y   & 1 & 3\\
\hline
Z   & 3 & 7\\
\hline
Tersedia 4  &  &\\
\hline
\end{tabular}
\end{table}

\begin{table}[h]
\caption{Kondisi yang menyebabkan Deadlock}
\begin{tabular}{|c|c|}
\hline
Mutual exclusion&Hanya ada satu proses yang boleh memakai sumber daya, dan proses lain yang ingin memakai sumber daya tersebut harus menunggu hingga sumber daya tadi dilepaskan atau tidak ada proses yang memakai sumber daya tersebut.\\
\hline
Hold and wait&Proses yang sedang memakai sumber daya boleh meminta sumber daya lagi maksudnya menunggu hingga benar-benar sumber daya yang diminta tidak dipakai oleh proses lain, hal ini dapat menyebabkan kelaparan sumber daya sebab dapat saja sebuah proses tidak mendapat sumber daya dalam waktu yang lama.\\
\hline
No preemption&Sumber daya yang ada pada sebuah proses tidak boleh diambil begitu saja oleh proses lainnya. Untuk mendapatkan sumber daya tersebut, maka harus dilepaskan terlebih dahulu oleh proses yang memegangnya, selain itu seluruh proses menunggu dan mempersilahkan hanya proses yang memiliki sumber daya yang boleh berjalan. \\
\hline
Circular wait&Kondisi seperti rantai, yaitu sebuah proses membutuhkan sumber daya yang dipegang proses berikutnya.\\
\hline
\end{tabular}
\label{deadlock}
\end{table}

\subsection {Deadlock Detection}
\begin {enumerate}
\item
1. Pendeteksian secara Algoritma, yaitu dengan cara kita mengetahui jika terjadinya deadlock, deadlock terjadi jika suatu permintaan tidak dapat ditangani segera.
	
\item
2. Recovery atau Pemulihan, yaitu yang pertama menggagalkan semua proses deadlock, yang kedua mem backup semua proses yang deadlock dan kemudian silahkan melakukan restart di semua proses yang sedang terjadi, yang ketiga menggagalkan semua proses yang deadlock secara berurutan sehingga tidak akan terjadi lagi deadock, dan yang terakhir yaitu menggagalkan pengalokasian resource secara berurutan hingga tidak ada deadlock.

\end {enumerate}

\subsection {Beberapa hal yang terjadi ketika mendeteksi adanya deadlock}
\begin {enumerate}
\item
1. Permintaan sumber daya dikabulkan selama memungkinkan.
\item
2. Sistem operasi melakukan scanning apakah ada kondisi circular wait secara peiodik.
\item
3. Pemeriksaan dilakukan setiap ada sumber daya yang hendak digunakan.
\item
4. memeriksa dengan algoritma tertentu.
\end {enumerate}

\subsection {Beberapa jalan untuk kembali dari deadlock}
\begin {enumerate}
\item
1. Lewat Preemption, yaitu dengan jauhkan sumber daya dari pemakainya untuk sementara waktu, tujuannya untuk memberikannya pada proses lain. strategi dengan memberikannya kesempatan pada proses lain dengan tanpa diketahui oleh pemilik dari sumber daya itu dan tergantung juga dari sifat sumber daya itu sendiri.
\item
2. Lewat melacak kembali, setelah melakukan prosesn dari preemption tersebut maka secara otomatis proses utama yang diambil sumber dayanya akan stop dan tidak akan melanjutkan prosesnya, oleh karena itu dibutuhkan langkah untuk dapat kembali pada keadaan aman, tetapi untuk menentukan keadaan aman tersebut sangatlah susah.
\item
3. Mematikan proses yang menyebabkan deadlock, ini merupakan cara yang sangat umum digunakan yaitu dengan cara mematikan semua proses yang mengalami deadlock.
\item
4. Menghindari deadlock, pada sistem permintaan untuk sumberdaya biasanya hanya dilakukan sekali saja, sistem harus sudah dapat mengenali bahwa sistem itu aman atau tidak.
\end {enumerate}

\cite{siahaan2015penyelarasan}
\cite{fauzi2013perangkat}
\cite{silberschatz2014operating}


\chapter[Baudrate]
{OS\\ Baud Rate}
%Nama Kelompok: Sistem_Operasi_BaudRate
%Kelas: D4 TI 1B
%Anggota : 
%Kevin Natanael Nainggolan(1174059) 
%Luthfi Muhammad Nabil(1174035)
%Salwaa Tania(1174047)
%Surya Pandu Prananda(1174036)

\section{Konsep Baud Rate}
Komunikasi secara berturut-turut sudah tidak asing lagi di era teknologi ini, salah satunya dikarenakan jumlah penghantar yang digunakan bisa lebih efektif daripada melakukannya secara sejajar. Mengapa demikian? Karena kata Berturut-turut berarti mengirim satu bit data dan selanjutnya yang diikuti oleh bit-bit data yang lain pada jalur yang sama. Karena itulah kita dapat meringkas penggunaan kabel. Dikarenakan jalur yang dilalui bersamaan, maka kecepatan komunikasi berturut-turut tidak secepat kecepatan komunikasi sejajar. Komunikasi sejajar, dapat mengirim data secara bersamaan melalui beberapa jalur. Namun, untuk proses secara keseluruhan, sistem komunikasi berturut-turut memenuhi berbagai aplikasi microcontroler. Selain itu, sistem komunikasi berturut-turut sering digunakan pada modem, USB, RS-232, dan teman-temannya.


%\chapter[PySerial]
%{OS\\ PySerial}
%\section{Py Serial}

	\subsection{Python}
	Python adalah bahasa pemrograman yang dibuat oleh Guido van Rossum dan populer sebagai bahasa pemrograman scripting dan Web. Mengacu pada ide wikipedia, 
	Python adalah bahasa pemrograman interpretatif multiguna dengan filosofi desain yang berfokus pada keterbacaan kode. Python dikenal sebagai bahasa pemrograman 
	yang menggabungkan nilai nilai kapabilitas, kemampuan, dengan sintaks kode yang sangat begitu jelas, 

\chapter[Serial Communication di Linux]
{OS\\ Serial Comm}
%kelompok 1 Sistem Operasi (Semaphore)
%Kelas D4 TI 1B
%Adam Noer Hidayatullah 1174097
%Ichsan Hizman
%Teddy
%Nisrina Aulia
%Irvan Rizkiansyah 1174043

\section{Komunikasi Serial pada Linux}
	
	\subsection{Konsep Dasar Komunikasi Serial}
	Suatu komunikasi yang dilakukan dimana suatu pengiriman data dilakukan per bit ialah dinamakan komunikasi serial, sehingga akan lebih lambat jika dibandingkan dengan komunikasi parallel seperti yang ada pada port printer yang dapat mengirim 8 bit sekaligus dalam sekali detak.
	Terdapat 2 macam cara komunikasi data serial yaitu :
		\begin{enumerate}
			\item Komunikasi data serial sinkron
			\item Komunikasi data serial asinkron
		\end{enumerate}
	
	Terdapat 2 kelompok device pada komunikasi serial yaitu :
		\begin{enumerate}
			\item Data Communication Equipment (DCE)
			Contohnya seperti scanner, printer, modem dan yang lainnya.
			\item Data Terminal Equipment (DTE)
			Contohnya sepertia terminal yang ada pada komputer.
		\end{enumerate}
	
	Keuntungan menggunakan port serial
		\begin{itemize}
			\item Masalah cable loss tidak akan menjadi suatu masalah yang besar pada komunikasi dengan kabel yang panjang, dari pada menggunakan kabel paralel. Port paralel akan mentransmisikan 0 pada tegangan 0 volt dan 1 di tegangan 1 volt, sedangkan port serial akan mentransmisikan 1 di tegangan -3 - -25 volt dan 0 di tegangan +3 - +25 volt.
			\item Hanya membutuhkan jumlah kabel yang sedikit, menggunakan 3 kabel saja pun bisa yaitu saluran Ground, saluran Transmit Data, saluran Receive Data.
			\item Populernya penggunaan mikrokontroler dan kebanyakan mikrokontroler dilengkapi dengan Serial Communication Interface (SCI) yang bisa dipaki untuk melakukan komunikasi dengan port serial pada komputer.
		\end{itemize}
		
	\subsection{Koneksi Linux ke Serial Port}
	Untuk melakukan setting pada suatu perangkat, terkadang harus masuk terlebuh dahulu ke dalam console box. Biasanya akan menggunakan hyperterminal, namun software bawaan seperti itu tidak terdapat pada linux pada saat linux terinstall. Maka dari itu terdapat sebuah software yang dapat digunakan pada linux untuk melakukan komunikasi serial yaitu minicom untuk menggantikan hyperterminal.
	
		

\chapter[SERIALCOMWINDOWS]
{OS\\ SERIALCOMWINDOWS}
\section{Serial Com Windows}
	\subsection{Membuka Port}

		Dokumentasi SDK Platform menyatakan bahwa ketika membuka port komunikasi, panggilan ke CreateFile memiliki persyaratan berikut:

	

		\begin{enumerate} 
			
				\item  fdw Share Mode harus nol. Port komunikasi tidak dapat dibagikan dengan cara yang sama seperti file yang dibagikan. Aplikasi yang menggunakan TAPI dapat menggunakan fungsi TAPI untuk memfasilitasi berbagi sumber daya antar aplikasi. Untuk aplikasi yang tidak menggunakan TAPI, penanganan warisan atau duplikasi diperlukan untuk berbagi port komunikasi. Berurusan dengan duplikat berada di luar cakupan artikel ini, silakan merujuk ke dokumentasi Platform SDK untuk informasi lebih lanjut.
				\item  fdw Create harus menentukan bendera OPENEXISTING.
				\item  h Template File parameter harus NULL.

			\item Satu hal yang perlu diperhatikan tentang nama port adalah bahwa mereka secara tradisional telah COM1, COM2, COM3, atau COM4. Windows API tidak menyediakan mekanisme apa pun untuk menentukan port apa yang ada pada sistem. Beberapa sistem bahkan memiliki lebih banyak port daripada maksimum tradisional empat. Vendor perangkat keras dan pembuat perangkat serial-driver bebas memberi nama port apa pun yang mereka sukai. Untuk alasan ini, yang terbaik adalah pengguna memiliki kemampuan untuk menentukan nama port yang ingin mereka gunakan. Jika port tidak ada, kesalahan akan terjadi setelah mencoba membuka port, dan pengguna harus diberitahu bahwa port tidak tersedia.

			\item Satu hal yang perlu diperhatikan tentang nama port adalah bahwa mereka secara tradisional telah COM1, COM2, COM3, atau COM4. Windows API tidak menyediakan mekanisme apa pun untuk menentukan port apa yang ada pada sistem. Beberapa sistem bahkan memiliki lebih banyak port daripada maksimum tradisional empat. Vendor perangkat keras dan pembuat perangkat serial-driver bebas memberi nama port apa pun yang mereka sukai. Untuk alasan ini, yang terbaik adalah pengguna memiliki kemampuan untuk menentukan nama port yang ingin mereka gunakan. Jika port tidak ada, kesalahan akan terjadisetelah mencoba membuka port, dan pengguna harus diberitahu bahwa port tidak tersedia.

		\end{enumerate}
		
		\subsection{I / O tumpang tindih}
		\begin{enumerate}
		

				\item I / O yang tumpang tindih tidak sesederhana I / O non-tumpang tindih, tetapi memungkinkan lebih banyak fleksibilitas dan efisiensi. Sebuah port terbuka untuk operasi tumpang tindih memungkinkan beberapa utas untuk melakukan operasi I / O pada saat yang bersamaan dan melakukan pekerjaan lain ketika operasi sedang menunggu. Lebih jauh lagi, perilaku operasi yang tumpang tindih memungkinkan satu utas untuk mengeluarkan banyak permintaan yang berbeda dan bekerja di latar belakang sementara operasi masih menunggu.

				\item I/ O yang tumpang tindih tidak sesederhana I / O non-tumpang tindih, tetapi memungkinkan lebih banyak fleksibilitas dan efisiensi. Sebuah port terbuka untuk operasi tumpang tindih memungkinkan beberapa utas untuk melakukan operasi I / O pada saat yang bersamaan dan melakukan pekerjaan lain ketika operasi sedang menunggu. Lebih jauh lagi, perilaku operasi yang tumpang tindih memungkinkan satu utas untuk mengeluarkan banyak permintaan yang berbeda dan bekerja di latar belakang sementara operasi masih menunggu.


				\item Baik dalam aplikasi single-threaded maupun multithread, beberapa sinkronisasi harus dilakukan antara mengeluarkan permintaan dan memproses hasilnya. Satu utas harus diblokir sampai hasil operasi tersedia. Keuntungannya adalah I / O yang tumpang tindih memungkinkan utas untuk melakukan beberapa pekerjaan antara waktu permintaan dan penyelesaiannya. Jika tidak ada pekerjaan yang dapat dilakukan, maka satu-satunya kasus untuk I / O yang tumpang tindih adalah memungkinkan untuk respon pengguna yang lebih baik.

				\item I / O yang tumpang tindih adalah jenis operasi yang digunakan sampel MTTTY. Ini menciptakan sebuah thread yang bertanggung jawab untuk membaca data port dan membaca status port. Ini juga melakukan pekerjaan latar belakang secara berkala. Program ini menciptakan untaian lain secara eksklusif untuk menulis data di luar port.
				
				\item Catatan Aplikasi terkadang menyalahgunakan sistem multithreading dengan membuat terlalu banyak utas. Meskipun menggunakan beberapa utas dapat menyelesaikan banyak masalah yang sulit, membuat untaian yang berlebihan bukanlah penggunaan yang paling efisien dalam aplikasi. Thread kurang regangan pada sistem daripada proses tetapi masih memerlukan sumber daya sistem seperti waktu CPU dan memori. Aplikasi yang menciptakan untaian berlebihan dapat mempengaruhi kinerja keseluruhan sistem.

		\end{enumerate}

	\begin{figure}[ht]
		\centerline{\includegraphics[width=1\textwidth]{figures/seria.png}}
		\caption{Serial Com Windows}
		\label{seria}
	\end{figure}
	Gambar \ref{seria} Contoh gambar.
		
		\begin{verbatim}
			HANDLE hComm;
			hComm = CreateFile( gszPort,  
                    GENERIC_READ | GENERIC_WRITE, 
                    0, 
                    0, 
                    OPEN_EXISTING,
                    FILE_FLAG_OVERLAPPED,
                    0);
			if (hComm == INVALID_HANDLE_VALUE)
				// error opening port; abort
		\end{verbatim}
	\subsection{Membaca dan menulis}
		\begin{enumerate}
			\item Membaca dari dan menulis ke port komunikasi di Windows sangat mirip dengan file input / output  di Windows. Bahkan, fungsi yang melengkapi I / O file adalah fungsi yang sama yang digunakan untuk serial I / O. I / O dapat dilakukan dengan salah satu dari dua cara: tumpang tindih atau tidak tumpang tindih. Dokumentasi SDK Platform menggunakan istilah asinkron dan sinkron untuk mengkonotasikan jenis operasi I / O ini. Artikel ini, bagaimanapun, menggunakan istilah yang tumpang tindih dan tidak terabaikan.
			\item Nonoverlapped I / O akrab bagi kebanyakan pengembang karena ini adalah bentuk tradisional I / O, di mana operasi diminta dan diasumsikan lengkap ketika fungsi kembali. Dalam kasus I / O yang tumpang tindih, sistem dapat kembali ke pemanggil segera bahkan ketika operasi tidak selesai dan akan memberi sinyal kepada pemanggil ketika operasi selesai. Program ini dapat menggunakan waktu antara permintaan I / O dan penyelesaiannya untuk melakukan beberapa pekerjaan latar belakang.
		\end{enumerate}
				\subsubsection{Bacaan}
					\begin{enumerate}
						\item Fungsi ReadFile menerbitkan operasi baca. ReadFileEx juga mengeluarkan operasi baca, tetapi karena tidak tersedia pada Windows 95, itu tidak tercakup dalam artikel ini. Berikut adalah potongan kode yang merinci cara mempublikasikan permintaan baca. Perhatikan bahwa fungsi memanggil fungsi untuk memproses data jika ReadFile mengembalikan TRUE. Ini adalah fungsi yang sama yang disebut jika operasi menjadi tumpang tindih. Perhatikan flag fWaitingOnRead yang didefinisikan oleh kode; ini menunjukkan apakah operasi baca tumpang tindih atau tidak. Ini digunakan untuk mencegah penciptaan operasi baca baru jika mereka luar biasa.
					\end{enumerate}
				
				\begin{verbatim}
				DWORD dwRead;
					BOOL fWaitingOnRead = FALSE;
					OVERLAPPED osReader = {0};

						// Create the overlapped event. Must be closed before exiting
						// to avoid a handle leak.
						osReader.hEvent = CreateEvent(NULL, TRUE, FALSE, NULL);

						if (osReader.hEvent == NULL)
						// Error creating overlapped event; abort.

						if (!fWaitingOnRead) {
						// Issue read operation.
						if (!ReadFile(hComm, lpBuf, READ_BUF_SIZE, &dwRead, &osReader)) {
						if (GetLastError() != ERROR_IO_PENDING)     // read not delayed?
						// Error in communications; report it.
					else
						fWaitingOnRead = TRUE;
				}
					else {    
						// read completed immediately
						HandleASuccessfulRead(lpBuf, dwRead);
			}
		}
				\end{verbatim}

				\begin{enumerate}
						\item Bagian kedua dari operasi yang tumpang tindih adalah deteksi penyelesaiannya. Pegangan acara dalam struktur OVERLAPPED diteruskan ke fungsi WaitForSingleObject, yang akan menunggu hingga objek diberi isyarat. Setelah acara ditandai, operasi selesai. Ini tidak berarti bahwa itu berhasil diselesaikan, hanya saja itu selesai. Fungsi GetOverlappedResult melaporkan hasil operasi. Jika kesalahan terjadi, GetOverlappedResult mengembalikan FALSE dan GetLastError mengembalikan kode kesalahan. Jika operasi selesai dengan sukses, GetOverlappedResult akan mengembalikan TRUE.

						
						\item Catatan Get Overlapped Result dapat mendeteksi penyelesaian operasi, serta mengembalikan status kegagalan operasi. Get Overlapped Result mengembalikan FALSE dan Get Last Error mengembalikan ketika operasi tidak selesai. Selain itu, Get Overlapped Result dapat dibuat untuk memblokir hingga operasi selesai. Ini secara efektif mengubah operasi yang tumpang tindih menjadi operasi non-tumpang tindih dan dicapai dengan melewatkan TRUE sebagai parameter bWait.
				\end{enumerate}


			
		
			\subsubsection{Penulisan}
				\begin{enumerate}
					\item Pengarsipan data dari port komunikasi sangat mirip dengan membaca, karena menggunakan banyak API yang sama. Cuplikan kode di bawah ini menunjukkan cara menghapus dan menunggu operasi tulis selesai.
				\end{enumerate}
				\begin{verbatim}
				BOOL WriteABuffer(char * lpBuf, DWORD dwToWrite)
{
   OVERLAPPED osWrite = {0};
   DWORD dwWritten;
   DWORD dwRes;
   BOOL fRes;

   // Create this write operation's OVERLAPPED structure's hEvent.
   osWrite.hEvent = CreateEvent(NULL, TRUE, FALSE, NULL);
   if (osWrite.hEvent == NULL)
      // error creating overlapped event handle
      return FALSE;

   // Issue write.
   if (!WriteFile(hComm, lpBuf, dwToWrite, &dwWritten, &osWrite)) {
      if (GetLastError() != ERROR_IO_PENDING) { 
         // WriteFile failed, but isn't delayed. Report error and abort.
         fRes = FALSE;
      }
      else
         // Write is pending.
         dwRes = WaitForSingleObject(osWrite.hEvent, INFINITE);
         switch(dwRes)
         {
            // OVERLAPPED structure's event has been signaled. 
            case WAIT_OBJECT_0:
                 if (!GetOverlappedResult(hComm, &osWrite, &dwWritten, FALSE))
                       fRes = FALSE;
                 else
                  // Write operation completed successfully.
                  fRes = TRUE;
                 break;
            
            default:
                 // An error has occurred in WaitForSingleObject.
                 // This usually indicates a problem with the
                // OVERLAPPED structure's event handle.
                 fRes = FALSE;
                 break;
         }
      }
   }
   else
      // WriteFile completed immediately.
      fRes = TRUE;

   CloseHandle(osWrite.hEvent);
   return fRes;
}
\end{verbatim}	

	\subsection{Serial Status}
		\begin{enumerate}
			\item Ada dua metode untuk mengambil status port komunikasi. Yang pertama adalah dengan mengatur event mask yang menyebabkan pemberitahuan aplikasi ketika peristiwa yang diinginkan terjadi. Fungsi Set Comm Mask mengatur masker kejadian ini, dan fungsi Wait Comm Event menunggu kejadian yang diinginkan terjadi. Metode kedua untuk mengambil status port komunikasi adalah secara berkala memanggil beberapa fungsi status yang berbeda. Polling, tentu saja, tidak efisien dan tidak direkomendasikan.
		\end{enumerate}
			\subsection{Communications Events}
				\begin{enumerate}
				\item Komunikasi dapat terjadi kapan saja selama menggunakan port komunikasi. Dua langkah yang terlibat dalam menerima pemberitahuan acara komunikasi adalah sebagai berikut:

				\item Set Comm Mask menetapkan peristiwa yang diinginkan yang menyebabkan pemberitahuan.
			\item WaitCommEvent menerbitkan pemeriksaan status. Pemeriksaan status dapat berupa operasi tumpang-tindih atau non-tumpang tindih, seperti halnya operasi baca dan tulis.
				\item Catatan : Peristiwa kata dalam konteks ini merujuk pada acara komunikasi saja. Itu tidak mengacu pada objek peristiwa yang digunakan untuk sinkronisasi.


				\end{enumerate}
			
	\begin{verbatim}
	DWORD dwStoredFlags;

	dwStoredFlags = EV_BREAK | EV_CTS   | EV_DSR | EV_ERR | EV_RING |\
                EV_RLSD | EV_RXCHAR | EV_RXFLAG | EV_TXEMPTY ;
		if (!SetCommMask(hComm, dwStoredFlags))
   // error setting communications mask
	\end{verbatim}
\subsection{Flow Control}
	\begin{enumerate}
		\item Kontrol aliran dalam komunikasi serial menyediakan mekanisme untuk menangguhkan komunikasi sementara salah satu perangkat sibuk atau karena alasan tertentu tidak dapat melakukan komunikasi apa pun. Secara tradisional ada dua jenis kontrol aliran: perangkat keras dan perangkat lunak.

\item Masalah umum dengan komunikasi serial adalah operasi tulis yang sebenarnya tidak menulis data ke perangkat. Seringkali, masalah terletak pada kontrol aliran yang digunakan ketika program tidak menentukannya. Pemeriksaan dekat dari struktur DCB mengungkapkan bahwa satu atau lebih dari anggota berikut mungkin BENAR: fOutxCtsFlow, fOutxDsrFlow, atau fOutX. Mekanisme lain untuk mengungkapkan bahwa kontrol aliran diaktifkan adalah memanggil ClearCommError dan memeriksa struktur COMSTAT. Ini akan mengungkapkan ketika transmisi ditangguhkan karena kontrol aliran.

\item Sebelum membahas jenis-jenis pengendalian aliran, pemahaman yang baik tentang beberapa istilah sudah teratur. Komunikasi serial terjadi antara dua perangkat. Secara tradisional, ada PC dan modem atau printer. PC diberi label Data Terminal Equipment . DTE kadang-kadang disebut tuan rumah. Modem, printer, atau peralatan periferal lainnya diidentifikasi sebagai Peralatan Komunikasi Data . DCE kadang-kadang disebut sebagai perangkat.
\end{enumerate}

\cite{bai2004windows}
\cite{carvey2005tracking}
\cite{boling2003programming}			


%\chapter[usb to serial]
%{OS\\ usb to serial}
%%Nama Kelompok: Sistem_Operasi_Deadlock
%Kelas: D4 TI 1B
%Alit Fajar Kurniawan(1174057) 
%Muhammad Iqbal Panggabean(1174063)
%Muhammad Afra Faris(1174041)
%Khadijah Hasanah Puteri Harahap(1174044)

\section {USB TO SERIAL}

\subsection {Pengertian USB}
\subsubsection {Apa itu USB ?}
	Universal Serial Bus atau yang disingkat dengan USB adalah sebuah teknologi yang dapat memungkinkan para penggunanya untuk dapat menghubungkan hardware eksternal contohnya seperti printer, keyboard, harddisk, flashdisk dan perangkat keras lainnya. kecepatan trnasfer data yang didukung oleh USB sebesar 12 Mbps. pada saat ini semua PC sudah memiliki port USB sendiri minimal 2 buah port USB.

	

%\chapter[Instalasi PIP]
%{Arduino to Database\\ Instalasi PIP}
%%kelompok 1 Sistem Operasi (Proses Os)
%Kelas D4 TI 1B
%Adam Noer Hidayatullah 1174097
%Ichsan Hizman
%Teddy
%Nisrina Aulia
%Irvan Rizkiansyah 1174043

\section{PIP}

    \subsection{PIP}
	PIP adalah singkatan dari pip installs python.sebuah tool yang memudahkan programmer untuk menginstall library-library  atau disebut juga sebuah package manager untuk Python

	\subsection{cara install pip di windows}
	Sebelum melakukan peng-install-an pip, di wajibkan untuk menginstall python terlebih dahulu. Karena tanpa python tidak akan bisa mengeksekusi installer pip, dan pastikan environment variables python nya sudah ter-setting
		\begin{enumerate}
			\item download pip langsung ke website resminya di \url{<https://pip.pypa.io/en/latest/installing/>}
			\item letakan file get-pip.py ke direktori yang mudah di temukan.
			\item Buka CMD dan masuk ke direktori file get-pip.py yang tadi sudah diletakkan direktori.
			\item Pada saat di CMD langsung ketikkan python get-pip.py .
			\item tunggu proses nya hingga selesai.
			\item Setelah menginstall, kini saatnya mensetting Environment Variables supaya mudah dan dapat menjalankan pip lewat CMD tanpa harus masuk ke dalam folder hasil instalasi pip.
			\item Masuk ke dalam Control Panel - System And Security - System - advanced System Settings
			\item Lalu setelah muncul windows baru, klik pada Environment variables, terhadap sesi System variables pilih path lalu klik edit dan tambahkan
				\begin{table}[H]
					\begin{tabular}{|c|}
						\hline
						variables value ;C:/Python27/Scripts\\
					\end{tabular}
				\end{table}
				
	\subsection{Virtualenv}
	virtualenv yaitu sebuah tool yang berguna untuk mengisolasi pada lingkungan python.
	
	\subsection{lingkungan pada PIP}
	lingkungan pada PIP yaitu melingkupi binary atau executable, library, dan semua package.
	
	\subsection{kegunaan virtualenv}
	kegunaan virtualenv adalah untuk menginstall aplikasi atau pada lebrary python. jika tidak menggunakan virtualenv dan menggunakan user selain root,
	akan terjadi error, karena user lain pada root tidak punya akses untuk menuju ke foldedr python yang berada di system.
				  
	\begin{figure} [ht]
		\centerline{\includegraphics[width=1\textwidth]{figures/setting-env.png}}
		\caption{Gambar Setting Environment pip}
		\label{setting-env}
	\end{figure}
	
	\ref{setting-env}
	
	Harus sesuai dengan direktori hasil instalasi pip. Lalu untuk melakukan pengetesan dari berhasil atau tidak berhasilnya instalasi pip, dengan cara buka CMD lalu ketikkan perintah pip, maka akan muncul seperti gambar dibawah ini.
	
	\begin{figure} [ht]
		\centerline{\includegraphics[width=1\textwidth]{figures/pip-terinstall.png}}
		\caption{Gambar pip yang Sudah Ter-install}
		\label{pip-terinstall}
	\end{figure}
	
	\ref{pip-terinstall}
				  
		\end{enumerate}

	\subsection{cara upgrade pip}
		\begin{item}
			\item python -m pip install -U pip
		\end{itemize}
		
	
	Dirangkum dari makalah \cite{feautrierpip}
	Dirangkum dari makalah \cite{nation2011qutip}
	Dirangkum dari artikel \cite{jewett2016moltemplate}


%\chapter[MySQL]
%{Arduino to Database\\ MySQL}
%\section{Cara Insert Database Mysql}

	\subsection{Database Mysql}}
	Basis data merupakan istilah yang mengacu kepada kumpulan data yang saling berhubungan satu sama lain, dan perangkat lunak harus mengacu pada sistem manajemen basis data DBMS. Jika konteksnya jelas, banyak administrator dan programer menggunakan istilah basis data untuk kedua makna, dan juga  merupakan sekumpulan data yang membentuk file yang saling berhubungan dengan format tertentu untuk membentuk data atau informasi baru. Atau database adalah kumpulan data yang saling berhubungan satu sama lain yang diatur berdasarkan skema atau struktur tertentu. Di komputer, basis data disimpan di perangkat penyimpanan perangkat keras, dan dengan perangkat lunak tertentu yang dimanipulasi oleh minat atau minat tertentu. 
	Hubungan atau hubungan data biasanya ditunjukkan oleh kunci dari setiap file. MySQL adalah sistem manajemen basis data perangkat lunak atau perangkat lunak SQL atau DBMS Multithread dan multi user. mysql sebenarnya merupakan turunan dari salah satu konsep utama dalam database untuk seleksi atau seleksi dan entri data yang memungkinkan operasi data dilakukan dengan mudah dan otomatis. 
	Michael Widenius merupakan seseorang yang menciptakan mysql pada tahun 1979, seorang programmer komputer swedia yang mengembangkan sistem database sederhana yang disebut unireg yang menggunakan koneksi mesin database isam tingkat rendah dengan pengindeksan. mysql adalah salah satu jenis server basis data yang paling populer.mysql menggunakan bahasa sql untuk mengakses database-nya. lisensi mysql adalah lisensi foss exception dan ada juga versi komersial. 
	Mysql tag adalah database sumber terbuka paling populer di dunia, dan sebenarnya merupakan turunan dari salah satu konsep utama dalam database yang sudah ada di sql. konsep operasi pada database terutama sql adalah untuk menseleksi atau seleksi dan entri data yang memungkinkan pengoperasian data dilakukan secara otomatis dengan mudah. 
	
	\subsection{Cara Penulisan Dasar Query INSERT}
	
	jika kita mengutip dari manual resmi MySQL sendiri, penulisan dasar perintah INSERT adalah seperti dibawah ini :
	
	\begin{verbatim}
		INSERT [LOW_PRIORITY | DELAYED | HIGH_PRIORITY] [IGNORE]
		[INTO] tbl_name [(col_name,...)]
		{VALUES | VALUE} ({expr | DEFAULT},...),(...),...
		[ ON DUPLICATE KEY UPDATE
		col_name=expr
		[, col_name=expr] ... ]
	\end{verbatim}
	
	\subsection{Cara Penggunaan Query INSERT...VALUES}
	Disini kita akan membahas perintah INSERT yang paling sederhana, yakni:
	
	\begin{verbatim}
		INSERT INTO nama_table VALUES (nilai_kolom1, nilai_kolom2,...);
	\end{verbatim}
	
	nama_table merupakan nama tabel yang akan kita input, dan nilai_kolom1 merupakan nilai yang akan kita imputkan ke dalam tabel tersebut, dan juga seterusnya untuk nilai kolom. Di nilai kolom haris berada dalam tanda kurung dan dipisah oleh tanda koma.
	
	Gambar dibawah adalah contoh memasukkan sebaris data :
	
	\begin{figure}[ht]
			\centerline{\includegraphics[width=0.5\textwidth]{figures/insert.png}}
			\caption{Insert}
			\label{insert}
			\end{figure}
			
	Kita juga bisa langsung memasukkan dua baris data ataupun lebih secara langsung hanya dengan satu perintah query INSERT, kita hanya butuh untuk memasukkan data di baris selanjutnya di belakang perintah, dengan bentuk seperti ini :
	
	\begin{verbatim}
		INSERT INTO nama_table VALUES (nilai_kolom1a, nilai_kolom2a,...), 
		(nilai_kolom1b, nilai_kolom2b,...);
	\end{verbatim}
	
	Gambar dibawah ini adalah contoh penambahan data dua baris sekaligus :
	

%\chapter[PostgreSQL]
%{Arduino to Database\\ PostgreSQL}
%\section{Cara Menghubungkan Python Dengan PostgreSQL}

\subsection{Definisi PostgreSQL}
\cite{momjian2001postgresql}PostgreSQL adalah database server dengan basis open source yang paling maju. PostgreSQL sendiri sudah dipakai oleh banyak kalangan di luar negeri karena faktor keamanan yang cukup terjamin. Seperti database server lainnya, PostgreSQL memakai bahasa pada umumnya yaitu standard SQL tetapi beberapa syntax diubah untuk kepentingan keamanan.Dalam PostgreSQL dapat menyimpan titik geografis dari aplikasi Sistem Informasi Geografis yang membuat bahasa ini popular dalam kalangan programmer Sistem Informasi Geografis. Selain titik geografis, PostgreSQL dapat menyimpan objek geometric diantaranya titik, garis, dan area. Selain itu, PostgreSQL dapat beroperasi seperti layaknya bahasa lainnya seperti manipulasi data atau membuat table. Hal tersebut memungkinkan PostgreSQL menjadi bahasa sql terbaik untuk digunakan pada Sistem Informasi Geografis karena sifatnya yang Spatially-enabled. Untuk mengunduh PostgreSQL dapat langsung dicari dan diunduh di website resminya yaitu di www.postgresql.org, cara instalasinya dapat terbilang cukup rumit karena untuk instalasi memakan banyak waktu dan tidak direkomendasikan untuk pemula, diantaranya harus memasuki regedit pada windows. 

\subsection{Integrasi Python Dengan PostgreSQL} 
Integrasi Python dengan PostgreSQL bisa dibilang sama dengan integrasi Python dengan bahasa lainnya.  Karena kita tinggal mendownload hal yang dibutuhkan dan membuat kode untuk dikoneksikan ke database. Salah satunya juga yaitu mengkoneksikan PostgreSQL dimana kita harus instalasi PostgreSQL itu sendiri. Berikut hal yang perlu disiapkan sebelum mengkoneksikan : 
\begin{itemize}
\item Connector Python to PostgreSQL : Fungsi connector yaitu penghubung antara python dengan postgresqlnya sendiri. 
\item Python (Semua Versi) : Sebelum mendownload piranti yang lain, pastikan python sudah terinstall di komputer
\item PostgreSQL DB : Sebagai media yang menyimpan data, aplikasi ini harus sudah terinstal juga di computer anda.
\end{itemize}

\subsection{Insert Data Di PostgreSQL}
Insert data pada umumnya memiliki sistem yang sama, bahasa yang sama. di PostgreSQL sendiri sama seperti bahasa yang umum yaitu MySQL yang memiliki syntax Insert seperti berikut
\begin{verbatim}
insert into db_postgre(kol1, kol2, kol3) values('kol1', 'kol2', 'kolek');
\end{verbatim}
Tetapi PostgreSQL memiliki aturan ketat mengenai cara insert database. Jadi tidak sembarang teknik menginsert dapat digunakan dalam metode insert data ke postgres.

\subsection{Pengujian}
Menguji POSTGRESQL pada platform UNIX sebagai berikut: 
beberapa kompilasi ini membutuhkan gcc
\begin {itemize} 
	\item aix - IBM pada AIX 3.2.5 atau 4.x
	\item alpha - DEC Alpha AXP pada Digital Unix 2.0, 3.2, 4.0
	\item BSD44derived - OS yang berasal dari 4.4-lite BSD (NetBSD, FreeBSD)
	\item bsdi - BSD / OS 2.x, 3.x, 4.x
	\item dgux - DG / UX 5.4R4.11
	\item hpux - HP PA-RISC pada HP-UX 9. *, 10. *
	\item i386solaris - i386
	\item Irix5 - SGI MIPS
	\item MIPS pada IRIX 5.3
	\item linux - Intel i86 Alpha SPARC PPC M68k
	\item sco - SCO 3.2v5
	\item Unixware
	\item sparcsolaris - SUN SPARC pada Solaris 2.4, 2.5, 2.5.1
	\item sunos4 - SUN SPARC pada SunOS 4.1.3
	\item svr4 - Intel x86 pada Intel SVR4 dan MIPS
	\item ultrix4 - DEC MIPS di Ultrix 4.4
\end {itemize}

Port selain sistem operasi Unix yang tersedia dan memungkinkan. 
Untuk mengkompilasi libpq C library, psql, dan interface lain dan binari untuk dijalankan pada MS Platform Windows. Dalam hal ini, klien berjalan di MS Windows, dan berkomunikasi melalui TCP / IP ke server yang berjalan di salah satu platform Unix kami yang didukung.Sebuah file win31.mak termasuk dalam distribusi untuk membuat libpq librq Win32 dan psql. 
Server database sekarang bekerja pada Windows NT menggunakan pustaka porting Cygnus Unix / NT

\subsection {Bahasa Pemrograman yang tersedia}
Bahasa yang tersedia untuk berkomunikasi dengan POSTGRESQL ada
 \begin {itemize}
	\item C atau libpq
	\item C ++ atau libpq ++
	\item Embedded C atau ecpg
	\item Java atau jdbc
	\item Perl atau perl5
	\item ODBC atau odbc
	\item Python atau PyGreSQL
	\item TCL atau libpgtcl
	\item C Easy API atau libpgeasy
	\item Embedded HTML atau PHP
\end {itemize}

\subsection{Instalasi} 
\begin {itemize}
	\item Dapatkan POSTGRESQL
	\item Membuat POSTGRESQL user
	\item Konfigurasi
	\item Kompilasi
	\item Instalasi
	\item Inisialisasi
	\item Memulai Server
	\item Membuat Database
\end {itemize}

\subsubsection{Akses selain host asli}
Cara untuk mencegah host lain mengakses
database POSTGRESQL, Secara default, POSTGRESQL hanya mengizinkan koneksi dari komputer lokal menggunakan soket domain Unix. Mesin lain tidak akan dapat terhubung kecuali Anda menambahkan flag -i ke postmaster, dan mengaktifkan hostbased otentikasi. Ini akan memungkinkan TCP / IP koneksi. Perangkat lunak POSTGRESQL didistribusikan dalam beberapa format:
\begin {itemize}
	\item File -Tar-gzip dengan ekstensi file .tar.gz
	\item File pra-paket dengan ekstensi file .rpm
	\item Format lain yang sudah dikemas
	\item CD ROM
\end {itemize}
Karena begitu banyak format yang ada, apendiks ini hanya akan membahas langkah-langkah umum yang perlu dipasang POSTGRESQL. Setiap distribusi dilengkapi dengan file INSTALL atau README dengan lebih spesifik instruksi.



%\chapter[MongoDB]
%{Arduino to Database\\ MongoDBL}
%%Nama Kelompok: Sistem_Operasi_Deadlock
%Kelas: D4 TI 1B
%Alit Fajar Kurniawan(1174057) 
%Muhammad Iqbal Panggabean(1174063)
%Muhammad Afra Faris(1174041)
%Khadijah Hasanah Puteri Harahap(1174044)

\section {Insert Database MongoDB}

\subsection {MongoDB}}
\subsubsection {Pengertian MongoDB}}
MongoDB (dari " humongous ") adalah platform lintas platform untuk sistem database yang berorientasi  dokumen. 
MongoDB diklasifikasikan sebagai database NoSQL dengan menghindari struktur basis data relasional berbasis tabel tradisional yang mendukung JSON sebagai dokumen dinamis (MongoDB menyebutnya dalam format BSON), membuatnya lebih mudah dan lebih cepat untuk mengintegrasikan beberapa data aplikasi. 
Dirilis di bawah Lisensi Publik Umum GNU Affero dan kombinasi lisensi Apache, MongoDB adalah perangkat lunak bebas dan sumber terbuka.

\subsubsection {Sejarah MongoDB}
Perusahaan ini dibangun pada Oktober 2007 oleh sebuah perusahaan New York City, 10gen (sekarang MongoDB Inc.) sebagai bagian dari platform yang direncanakan sebagai produk layanan.
MongoDB telah diadopsi sebagai perangkat lunak backend melalui banyak situs web dan layanan, termasuk Craigslist, eBay, Square, SourceForge, dan The New York Times. MongoDB adalah sistem basis data NoSQL yang paling populer. 
Perusahaan telah mengembangkan model pengembangan open source pada tahun 2009, dengan 10gen menawarkan dukungan komersial dan layanan lainnya.

%\chapter[SQLite]
%{Arduino to Database\\ SQLite}
%\section{Cara insert SQLite}
	\subsection{Definisi}
		\begin{enumerate}
			\item SQLite adalah sistem manajemen basis data relasional yang terkandung dalam pustaka pemrograman C. Tidak seperti banyak sistem manajemen basis data lainnya, SQLite bukan mesin basis data client-server. Semuanya, dicetak dalam program terakhir.
			\item SQLite adalah ACID-compliant dan mengimplementasikan sebagian besar standar SQL, menggunakan sintaks SQL dinamis dan yang tidak dapat menjamin integritas domain..
		\end{enumerate}
	\subsection{Sejarah}
		\begin{enumerate}
			\item D. Richard Hipp mendesain SQLite pada musim semi tahun 2000 ketika bekerja untuk General Dynamics dalam kontrak dengan Angkatan Laut Amerika Serikat. Hipp merancang perangkat lunak yang digunakan pada kapal perusak rudal, yang semula menggunakan HP-UX dengan database back-end IBM Informix. SQLite dimulai sebagai ekstensi Tcl.
			\item Tujuan desain dari SQLite adalah untuk memungkinkan program untuk beroperasi tanpa menginstal sistem manajemen basis data atau membutuhkan administrator basis data. Hipp berdasarkan sintaks dan semantik pada PostgreSQL 6.5. Pada bulan Agustus 2000, versi 1.0 dari SQLite dirilis, dengan penyimpanan berdasarkan gdbm (GNU Database Manager). SQLite 2.0 menggantikan gdbm dengan implementasi B-tree khusus, menambahkan kemampuan transaksi. SQLite 3.0, sebagian didanai oleh America Online, menambahkan internasionalisasi, pengetikan manifes, dan peningkatan besar lainnya.
			\item Pada tahun 2011 Hipp mengumumkan rencana untuk menambahkan antarmuka NoSQL (mengelola dokumen yang dinyatakan dalam JSON) ke database SQLite dan mengembangkan UnQLite, database berorientasi dokumen yang dapat disematkan. UnQLite dirilis sebagai database independen.
		\end{enumerate}
	\subsection{Desain}
		\begin{enumerate}
			\item Tidak seperti sistem manajemen basis data client-server, mesin SQLite tidak memiliki proses mandiri yang digunakan aplikasi untuk berkomunikasi. Sebaliknya, pustaka SQLite terhubung dan dengan demikian menjadi bagian integral dari program aplikasi. Menautkan dapat berupa statis atau dinamis. Program aplikasi menggunakan fungsi SQLite melalui panggilan fungsi sederhana, yang mengurangi latensi dalam akses database: panggilan fungsi dalam satu proses lebih efisien daripada komunikasi antar-proses. SQLite menyimpan seluruh basis data (definisi, tabel, indeks, dan datanya sendiri) sebagai file lintas platform tunggal pada mesin host. Ini mengimplementasikan desain sederhana ini dengan mengunci seluruh file database saat menulis. Operasi baca SQLite dapat multitasked, meskipun menulis hanya dapat dilakukan secara berurutan.
			\item Karena desainnya tanpa server, aplikasi SQLite memerlukan lebih sedikit konfigurasi daripada database client-server. SQLite disebut nol-conf karena tidak memerlukan manajemen layanan (seperti skrip startup) atau kontrol akses berdasarkan GRANT dan kata sandi. Kontrol akses ditangani dengan cara izin sistem file yang diberikan ke file database itu sendiri. Database dalam sistem client-server menggunakan hak akses file sistem yang memberikan akses ke file database hanya ke proses daemon.
			\item Implikasi lain dari desain tanpa server adalah bahwa beberapa proses mungkin tidak dapat menulis ke file database. Dalam basis data berbasis server, beberapa penulis akan terhubung ke daemon yang sama, yang mampu menangani kunci secara internal. Di sisi lain, SQLite harus bergantung pada kunci sistem file. Ini memiliki sedikit pengetahuan tentang proses lain yang mengakses database pada saat yang bersamaan. Oleh karena itu, SQLite bukan pilihan yang lebih disukai untuk penyebaran intensif. Namun, untuk pertanyaan sederhana dengan sedikit konkurensi, keuntungan kinerja SQLite dari menghindari overhead yang meneruskan datanya ke proses lain.
			\item SQLite menggunakan PostgreSQL sebagai platform referensi. Apa yang akan dilakukan PostgreSQL digunakan untuk memahami standar SQL. Salah satu ketidakberesan utama adalah bahwa, dengan pengecualian kunci primer, SQLite tidak memaksakan jenis pemeriksaan; jenis nilai bersifat dinamis dan tidak dibatasi oleh skema (meskipun skema akan memicu konversi saat menyimpan, jika konversi semacam itu berpotensi reversibel). SQLite berusaha mengikuti Aturan Postel.
		\end{enumerate}
		
		\subsection{Syntax Perintah Insert}
\begin{verbatim}

INSERT INTO nama_tabel [(kolom1, kolom2, kolom3, … kolomn)]
VALUES (nilai1, nilai2, nilai3,… nilaiN);
Kolom1, kolom2, kolom3 merupakan nama kolom yang ada didalam tabel sqlite yang akan kamu tambahkan data kedalam masing masing kolom tersebut.

INSERT INTO nama_tabel VALUES (nilai1, nilai2, nilai3,… nilaiN);
Contoh Perintah Insert Di SQLite


INSERT INTO siswa (id, nama, umur, alamat)
VALUES (1, 'Firdan Ardiansyah',27,'JL. KH. Atim II');
INSERT INTO siswa (id, nama, umur, alamat)
VALUES (2, 'Muhammad Ammar',25,'BTN. Palaton');
INSERT INTO siswa (id, nama, umur, alamat)
VALUES (3, 'Bilal Ardiansyah',23,'BTN. Depag');
INSERT INTO siswa (id, nama, umur, alamat)
VALUES (4, 'Rafi Syabani',25,'BTN. Sumur Buang');

INSERT INTO siswa VALUES (5, 'Muhammad Bintang',23,'Pandeglang'); 
Melihat Data Didalam Tabel Siswa
\end{verbatim}


	\subsection{Melihat Data}
		\begin{verbatim}
SELECT * FROM siswa;
		\end{verbatim}
		
	\begin{figure}[ht]
		\centerline{\includegraphics[width=1\textwidth]{figures/Sql.png}}
		\caption{Hasilnya}
		\label{Sql}
	\end{figure}
	Gambar \ref{Sql} Hasil gambar.
	
	\subsection{Fiture}
		\begin{enumerate}
			\item SQLite mengimplementasikan sebagian besar standar SQL-92 untuk SQL tetapi tidak memiliki beberapa fitur. Misalnya, beberapa menyediakan pemicu, dan tidak dapat menulis ke tampilan (tetapi menyediakan pemicu INSTEAD OF yang menyediakan fungsi ini). Meskipun menyediakan query kompleks, masih memiliki fungsi ALTER TABLE terbatas, karena tidak dapat memodifikasi atau menghapus kolom.
			\item SQLite menggunakan sistem tipe yang tidak biasa untuk DBMS yang kompatibel dengan SQL: alih-alih menetapkan jenis ke kolom seperti pada kebanyakan sistem basis data SQL, mereka ditetapkan ke nilai individual: dalam bahasa yang diketik secara dinamis. Selain itu, itu lemah diketik dalam beberapa cara yang sama bahwa Perl adalah: satu dapat memasukkan string ke dalam kolom integer (meskipun SQLite akan mencoba untuk mengubah string ke integer pertama, jika jenis kolom yang diinginkan adalah bilangan bulat). Ini menambah fleksibilitas ke kolom, terutama ketika terikat ke bahasa scripting yang diketik secara dinamis. Namun, teknik ini tidak portabel untuk produk SQL lainnya. Kritik umum adalah bahwa sistem tipe SQLite tidak memiliki mekanisme integritas data yang disediakan oleh kolom-kolom yang diketik secara statis dalam produk lain. Situs web SQLite menggambarkan mode afinitas yang ketat, tetapi fitur ini belum ditambahkan. Namun, itu dapat diimplementasikan dengan pembatasan semacam itu. Tabel biasanya menyertakan kolom index rowid tersembunyi yang memberikan akses lebih cepat. Jika database termasuk kolom Integer Primary Key, SQLite biasanya akan mengoptimalkannya dengan memperlakukannya sebagai alias untuk rowid, menyebabkan kontennya disimpan sebagai integer 64-bit yang diketik-ketat dan mengubah perilakunya menjadi agak seperti penambahan otomatis kolom. Versi masa depan dari SQLite mungkin termasuk perintah untuk mengintrospeksi apakah sebuah kolom memiliki perilaku seperti itu dari rowid untuk membedakan kolom ini dari Kunci Primer Integer yang tidak terkunci secara otomatis.
			\item SQLite dengan fungsi Unicode penuh adalah opsional.
			\item Beberapa proses komputer atau utas dapat mengakses basis data yang sama secara bersamaan. Beberapa akses dapat dibaca secara paralel. Akses hanya dapat digunakan jika ada. Jika tidak, akses tulis gagal dengan kode kesalahan (atau dapat dicoba kembali secara otomatis sampai batas waktu dapat dikonfigurasi untuk kedaluwarsa). Akses data ini akan berubah ketika berhadapan dengan tabel sementara. Pembatasan ini dilonggarkan dalam versi 3,7 dari kompilasi write-ahead (WAL) diaktifkan untuk memungkinkan membaca dan pencurian simultan. SQLite versi 3.7.4 pertama kali melihat modul FTS4 tambahan (pencarian teks lengkap), menampilkan perbaikan dalam modul FTS3 yang lebih lama. FTS4 memungkinkan pengguna untuk melakukan teks lengkap pada dokumen yang mirip dengan bagaimana mesin pencari mencari halaman web. Versi 3.8.2 Tambah untuk membuat tabel tanpa rowid, yang dapat memberikan ruang dan peningkatan kinerja. Penguatan tabel umum ditambahkan ke SQLite di versi 3.8.3.
		\end{enumerate}
	\subsection{Perkembangan dan Distribusi}
		\begin{enumerate}
			\item Kode SQLite di-host dengan Fossil, sistem kontrol versi terdistribusi yang dibangun di database SQLite.
			\item Program baris perintah yang berdiri sendiri disediakan dalam distribusi SQLite. Ini dapat digunakan untuk membuat database, menentukan tabel, menyisipkan dan mengubah baris, menjalankan kueri dan mengelola file database SQLite. Ini juga berfungsi sebagai contoh untuk menulis aplikasi yang menggunakan pustaka SQLite.
			\item SQLite menggunakan pengujian regresi otomatis sebelum setiap rilis. Lebih dari 2 juta tes dijalankan sebagai bagian dari verifikasi rilis. Dimulai dengan rilis 10 Agustus 2009 dari SQLite 3.6.17, rilis SQLite memiliki cakupan cabang cakupan 100%, salah satu komponen cakupan kode. Tes dan tes menggunakan domain publik dan kepemilikan parsial.
		\end{enumerate}
	\subsection{SQLite Serverless}
		\begin{enumerate}
			\item Sebagian besar mesin database SQL diimplementasikan sebagai proses server terpisah. Program yang ingin mengakses database berkomunikasi dengan server menggunakan beberapa jenis komunikasi interprocess (biasanya TCP / IP) untuk mengirim permintaan ke server dan menerima hasilnya kembali. SQLite tidak berfungsi dengan cara ini. Dengan SQLite, proses yang ingin mengakses database membaca dan menulis langsung dari file database pada disk. Tidak ada proses server menengah.
			\item Ada kelebihan dan kekurangan menjadi serverless. Keuntungan utama adalah tidak ada proses server terpisah untuk menginstal, mengkonfigurasi, mengkonfigurasi, menginisialisasi, mengelola, dan memecahkan masalah. Ini adalah salah satu alasan mengapa SQLite adalah mesin database "konfigurasi-nol". Program yang menggunakan SQLite tidak memerlukan dukungan administratif untuk menyiapkan mesin basis data sebelum dijalankan. Setiap program yang dapat mengakses disk dapat menggunakan database SQLite.
			\item Di sisi lain, mesin database yang menggunakan server dapat memberikan perlindungan yang lebih baik dari bug di aplikasi klien - pointer liar pada klien tidak dapat merusak memori di server. Dan karena server adalah proses persisten tunggal, ia mampu mengontrol akses basis data dengan lebih presisi, memungkinkan penguncian yang lebih baik dan konkurensi yang lebih baik.
			\item Sebagian besar mesin klien / server database berbasis SQL. Dari mereka tanpa server, SQLite adalah satu-satunya yang diketahui dari penulis ini yang memungkinkan beberapa aplikasi untuk mengakses database yang sama pada saat yang bersamaan.
			\item Biasanya, RDBMS seperti MySQL, PostgreSQL, dll, membutuhkan proses server terpisah untuk beroperasi. Aplikasi yang ingin mengakses server database menggunakan protokol TCP / IP untuk mengirim dan menerima permintaan. Ini disebut arsitektur client / server.
		\end{enumerate}
	\subsection{Neo Serverless vs Classic Serverless}
		\begin{enumerate}
			\item Baru-baru ini, orang mulai menggunakan kata serverless berarti sesuatu yang sangat berbeda dari makna yang dimaksudkan dalam dokumen ini. 
			\item Berikut dua definisi yang mungkin tanpa server:
			\item Classic Serverless: Mesin database berjalan dalam proses, thread, dan ruang alamat yang sama dengan aplikasi. Tidak ada pesan yang lewat atau aktivitas jaringan.
			\item Neo-Serverless: Mesin database berjalan di ruang nama aplikasi terpisah, mungkin pada mesin yang terpisah, tetapi database disediakan sebagai layanan turn-key oleh penyedia hosting, yang tidak memerlukan manajemen atau administrasi dari pemilik aplikasi, dan sangat mudah digunakan bahwa pengembang dapat mengasumsikan database sebagai serverless bahkan jika server benar-benar menggunakan di bawah selimut.
			\item SQLite adalah contoh mesin database tanpa server klasik. Dengan SQLite, tidak ada proses, thread, mesin, atau mekanisme lain (selain OS host komputer dan sistem file) untuk membantu menyediakan layanan atau implementasi database. Sama sekali tidak ada server.
			\item Microsoft Azure Cosmo DB dan Amazon S3 adalah contoh dari database neo-serverless. Database ini diimplementasikan oleh proses server yang berjalan secara terpisah di cloud. Tetapi server dikelola dan dikelola oleh ISP, bukan oleh pengembang aplikasi. Pengembang aplikasi hanya menggunakan layanan ini. Pengembang tidak harus menyediakan, mengkonfigurasi, atau mengelola instance server basis data, karena semua pekerjaan ditangani secara otomatis oleh penyedia layanan. Server database memang ada, mereka hanya tersembunyi dari para pengembang.
			\item Penting untuk memahami dua definisi yang berbeda ini untuk serverless. Ketika sebuah database mengklaim sebagai serverless, pastikan untuk mengetahui apakah itu berarti serverless klasik atau neo-serverless.
		\end{enumerate}

		\subsection{Mandiri}
		\begin{enumerate}
		\item SQLite adalah alat mandiri yang memerlukan dukungan minimal dari sistem operasi atau pustaka eksternal. Ini membuat SQLite dapat digunakan di lingkungan apa pun terutama di perangkat yang tertanam seperti iPhone, ponsel Android, konsol game, pemutar media genggam, dll.
		\item SQLite dikembangkan menggunakan ANSI-C. Kode sumber tersedia sebagai sqlite3.c besar dan file header-nya sqlite3.h. Jika Anda ingin mengembangkan aplikasi yang menggunakan SQLite, Anda hanya perlu memasukkan file-file ini ke dalam proyek Anda dan kompilasi dengan kode Anda.
		\end{enumerate}
		
		\subsection{Kesimpulan}
			\begin{enumerate}
				\item SQLite adalah pusat dalam pemrosesan yang mengimplementasikan server di database SQL sendiri, tanpa server, nol - konfigurasii, transakssional. Kode untuk SQLite ada di domain publik dan dengan demikian bebas digunakan untuk tujuan apa pun, komersial atau pribadi. SQLite adalah database yang paling banyak digunakan di dunia dengan aplikasi lebih dari yang dapat kita hitung, termasuk beberapa proyek profil tinggi.
				\item SQLite adalah mesin basis data SQL tertanam. Tidak seperti kebanyakan database SQL lainnya, SQLite tidak memiliki proses server terpisah. SQLite membaca dan menulis langsung ke file disk biasa. Database SQL lengkap dengan banyak tabel, indeks, pemicu, dan tampilan, terdapat dalam satu file disk. Format file database adalah cross-platform - Anda dapat dengan bebas menyalin database antara sistem 32-bit dan 64-bit atau antara arsitektur big-endian dan little-endian. Fitur-fitur ini menjadikan SQLite pilihan populer sebagai Format File Aplikasi. Pikirkan SQLite bukan sebagai pengganti Oracle tetapi tidak fopen ().
				\item SQLite adalah pustaka yang ringkas. Dengan semua fitur diaktifkan, ukuran pustaka bisa kurang dari 500KiB, tergantung pada platform target dan pengaturan optimasi kompilator. (Kode 64-bit lebih besar, dan beberapa optimisasi kompilator seperti inlining fungsi agresif dan dekomposisi loop dapat menyebabkan kode objek menjadi jauh lebih besar.) Ada tradeoff antara penggunaan memori dan kecepatan. SQLite umumnya berjalan lebih cepat, semakin banyak memori yang Anda berikan. Namun demikian, kinerja biasanya cukup baik bahkan dalam lingkungan memori yang rendah. Tergantung pada bagaimana itu digunakan, SQLite bisa lebih cepat daripada filesystem langsung I atau O.
				\item SQLite sangat hati-hati diuji sebelum rilis masing-masing dan memiliki reputasi sebagai sangat dapat diandalkan. Sebagian besar kode sumber SQLite spesifik untuk pengujian dan verifikasi. Sebuah rangkaian uji otomatis menjalankan jutaan dan jutaan kasus uji coba yang melibatkan ratusan juta pernyataan SQL individual dan mencapai cakupan cakupan 100%. SQLite merespon dengan anggun untuk kegagalan alokasi memori dan disk I / O kesalahan. Transaksi adalah ACID bahkan jika terganggu oleh sistem crash atau gangguan listrik. Semua ini diverifikasi oleh tes otomatis menggunakan alat uji khusus yang mensimulasikan kegagalan sistem. Tentu saja, bahkan dengan semua tes ini, masih ada bug. Tapi tidak seperti beberapa proyek serupa (terutama pesaing komersial) SQLite terbuka dan jujur ​​tentang semua bug dan memberikan daftar bug dan kronologi menit-demi-menit dari perubahan kode.
				\item Basis kode SQLite didukung oleh tim pengembangan internasional yang bekerja pada SQLite penuh waktu. Pengembang terus memperluas kemampuan SQLite dan meningkatkan keandalan dan kinerja mereka sambil mempertahankan kompatibilitas mundur dengan spesifikasi antarmuka yang dipublikasikan, sintaks SQL, dan format file database. Kode sumbernya benar-benar gratis bagi siapa saja yang menginginkannya, tetapi dukungan profesional juga tersedia.
				\item Proyek SQLite dimulai pada 2000-05-09. Masa depan selalu tidak dapat diprediksi, tetapi maksud dari pengembang adalah untuk mendukung SQLite sepanjang tahun 2050. Keputusan desain dibuat dengan mengingat tujuan tersebut.
				\item Kami pengembang berharap bahwa Anda menemukan SQLite berguna dan kami meminta Anda untuk menggunakannya dengan baik: untuk membuat produk cantik dan indah yang cepat, andal, dan mudah digunakan. Carilah pengampunan untuk diri sendiri saat Anda memaafkan orang lain. Dan sama seperti Anda telah menerima SQLite gratis, jadi berikan juga gratis, bayar utang ke depan.
			\end{enumerate}
			
\cite{owens2010sqlite}
\cite{newman2004sqlite}
\cite{kreibich2010using}
\cite{kang2013x}
\cite{jeon2012recovery}
\cite{lee2012creating}

\subsection{insert sqlite language}
\begin{enumerate}
\itemSisipan Bahasa Query Sqlite
Pernyataan INSERT hadir dalam tiga bentuk dasar.
1. INSERT INTO table VALUES (...);
Bentuk pertama (dengan kata kunci "VALUES") membuat satu atau lebih baris baru dalam tabel yang ada. Jika daftar nama kolom setelah nama tabel dihilangkan, maka jumlah nilai yang dimasukkan ke setiap baris harus sama dengan jumlah kolom dalam tabel. Dalam hal ini, hasil evaluasi ekspresi paling kiri dari setiap istilah daftar VALUES dimasukkan ke dalam kolom paling kiri dari setiap baris baru, dan seterusnya untuk setiap ekspresi berikutnya. Jika daftar kolom-nama ditentukan, maka jumlah nilai dalam setiap jangka waktu daftar VALUE harus sesuai dengan jumlah kolom yang ditentukan. Setiap kolom bernama dari baris baru diisi dengan hasil mengevaluasi ekspresi VALUES yang sesuai. Kolom tabel yang tidak muncul dalam daftar kolom diisi dengan nilai kolom default (ditetapkan sebagai bagian dari pernyataan CREATE TABLE), atau dengan NULL jika tidak ada nilai default yang ditentukan.
2. INSERT INTO table SELECT ...;
Bentuk kedua dari pernyataan INSERT berisi pernyataan SELECT bukannya klausa VALUES. Entri baru dimasukkan ke dalam tabel untuk setiap baris data yang dikembalikan dengan mengeksekusi pernyataan SELECT. Jika daftar kolom ditentukan, jumlah kolom dalam hasil SELECT harus sama dengan jumlah item dalam daftar-kolom. Jika tidak, jika tidak ada daftar kolom yang ditentukan, jumlah kolom dalam hasil SELECT harus sama dengan jumlah kolom dalam tabel. Setiap pernyataan SELECT, termasuk pernyataan gabungan SELECT dan SELECT dengan klausa ORDER BY dan / atau LIMIT, dapat digunakan dalam pernyataan INSERT dari formulir ini.
3. INSERT INTO table DEFAULT VALUES;
Bentuk ketiga dari pernyataan INSERT adalah dengan NILAI DEFAULT. Pernyataan INSERT ... DEFAULT VALUES menyisipkan satu baris baru ke dalam tabel bernama. Setiap kolom dari baris baru diisi dengan nilai default-nya, atau dengan NULL jika tidak ada nilai default yang ditetapkan sebagai bagian dari definisi kolom dalam pernyataan CREATE TABLE.
\end{enumerate}
			
			



%\chapter[DeadLock]
%{OS\\ DeadLock}
%\input{section/pullrequestdeadlock.tex}

%\chapter[Internet]
%{Definisi\\ Internet}
%\input{section/1internet.tex}

%\chapter[Web]
%{Definisi\\ Web}
%\input{section/1web.tex}

%\chapter[Backend]
%{Definisi\\ Backend}
%\input{section/1Backend.tex}

%\chapter[Frontend]
%{Definisi\\ Frontend}
%\input{section/1Frontend.tex}

% contoh aplikasi web service
% web service
% protokol
% port

% HTTP
% URL
% POST
% GET


\bibliographystyle{IEEEtran}
\bibliography{references, kelompok31A}

\printindex

\end{document}

