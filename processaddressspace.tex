 \section manajemen memori
Konsep dasar memori
Memori adalah sebagai suatu pusat dari operasi yang pada setiap komputer modern , memori berfungsi untuk tempat penyimpan sebuah informasi dan harus diatur dan dijaga sebaik mungkin . memori adalah array besar dari word atau byte yang bisa disebut alamat CPU untuk mengambil instruksi instruksi dari sebuah memory
CPU mengambil instruksi instruksi dari memori berdasarkan pada nilai dari penghitung program  program.
Sedangkan manajemen memori adalah kegiatan untuk mengatur memori komputer. Proses ini akan menyediakan cara untuk mengalokasikan memori untuk memproses atas sebuah  permintaan mereka, membebaskannya untuk digunakan kembali ketika tidak lagi diperlukan dan mempertahankan alokasi ruang memori untuk proses tersebut. Manajemen memori utama sangat penting untuk sistem komputer  penting untuk mengolah dan menginput fasilitas atau keluaran secara efisien, sehingga memori dapat ditampung sebanyak mungkin proses dan dalam upaya pemogram atau proses tidak terbatas kapasitas memori fisik dalam sistem komputer (Eko, 2009).

Memory manager adalah salah satu bagian dari sistem operasi yang sangat berpengaruh dalam menentukan proses proses yang ditempatkan pada antrian.




\subsection
Ketika kita bekerja dengan program aplikasi bagdi maka kita akan menghasilkan data. Data akan disimpan sementara di RAM yang tersisa. Data yang disimpan dalam RAM adalah voletile, yang berarti bahwa data hanya dapat bertahan selama daya komputer AKTIF. Untuk menyimpan kebiasaan menyimpan data ke disk dalam waktu tidak terlalu lama, misalnya setiap 5 menit

\subsection Fair Physical Memory Allocation
Alokasi Memori Fisik yang Adil
Alokasi Memori Fisik yang Adil Biasanya Digunakan dalam Manajemen Memori untuk membagi penggunaan memori fisik secara merata ke dalam setiap proses yang berjalan pada suatu sistem.

\subsection Shared Virtual Memory.
Meskipun dalam Memori Virtual Bersama setiap proses menggunakan ruang alamat yang berbeda dari memori virtual, tetapi ada kalanya proses dihadapkan untuk berbagi penggunaan memori.

C. Ruang Alamat Logika dan Fisik
Alamat logika adalah alamat yang dihasilkan dalam CPU, juga disebut alamat virtual. Alamat fisik adalah alamat yang terlihat oleh memori. Untuk mengubah dari alamat logis ke alamat fisik perangkat keras yang disebut MMU (Memory Management Unit) diperlukan. Mengubah dari alamat logis ke alamat fisik adalah pusat manajemen memori. Alamat yang dihasilkan oleh CPU disebut alamat logis di mana alamat muncul sebagai memori uni yang disebut alamat fisik. Tujuan utama dari manajemen memori adalah konsep menempatkan ruang alamat logis ke ruang alamat fisik (Ama, 2003).

Hasil dari skema waktu gabungan dan waktu pengikatan alamat pada alamat yang logis dan alamat yang ada pada memori adalah serupa. Tetapi skema waktu pengikatan waktu untuk eksekusi alamat berbeda. dalam hal ini, alamat logis disebut alamat virtual. Himpunan semua alamat logika yang dihasilkan oleh program disebut ruang alamat logis; himpunan semua alamat fisik yang terkait dengan alamat logis disebut ruang alamat fisik.
Unit Manajemen Memori (MMU) adalah perangkat perangkat keras yang memetakan alamat virtual ke alamat fisik. Dalam skema MMU, nilai register relokasi ditambahkan ke setiap alamat yang dihasilkan oleh proses pengguna ketika dikirim ke memori.

\subsection Multiprogramming Dengan Swapping
manajemen memori adalah kegiatan mentransfer gambar proses antara memori utama dan disk selama eksekusi, atau dengan kata lain itu adalah pengalihan manajemen proses dari memori utama ke disk dan kembali lagi (swapping). Manajemen ini terdiri dari:
Multiprogramming Dengan Partisi Dinamis
Jumlah, lokasi, dan ukuran proses dalam memori dapat bervariasi dari waktu ke waktu secara dinamis. Kelemahan: a) Lubang memori dapat terjadi di antara partisi yang digunakan; b) mempersulit alokasi dan alokasi memori.

\subsection Swapping
salah satu contoh untuk mengilustrasikan teknik swapping adalah Algoritma Round-Robin yang digunakan dalam lingkungan multiprogamming menggunakan waktu kuantum dalam pelaksanaan prosesnya. ketika waktu kuantum berakhir, pengelola memori akan mengeluarkan proses lain ke dalam memori bebas.. Pada saat yang sama, penjadwal CPU akan mengalokasikan waktu ke proses lain dalam memori. Perhatiannya adalah bahwa waktu kuantum harus cukup lama sehingga waktu penggunaan CPU dapat lebih optimal bila dibandingkan dengan proses pertukaran yang terjadi antara memori dan disk.
Teknik swapping bergulir keluar, roll in menggunakan algoritma berbasis prioritas di mana ketika proses prioritas yang lebih tinggi tiba maka manajer memori akan mengeluarkan proses prioritas yang lebih rendah dan memuat proses dengan prioritas yang lebih tinggi.Ketika proser prioritas tinggi telah selesai mengeksekusi feed, proses yang memiliki prioritas lebih rendah dapat dimasukkan kembali ke dalam memori dan kembali dalam eksekusi.
Waktu swapping adalah waktu transer, misalnya kita melihat ilustrasi berikut. Proses pengguna memiliki 5 mb, sementara penyimpanan sementara / kecepatan hard drive 20 mb maka waktu yang diperlukan untuk mentransfer proses 5 MB adalah 250 ms.Karena ada dua contoh di mana satu proses melucuti proses dan yang lainnya adalah proses memasukkan proses ke dalam memori, total waktu swap menjadi 252 + 252 = 504 ms.
Untuk teknik swapping agar lebih efisien, lebih baik menukar proses yang benar-benar diperlukan sehingga mengurangi waktu swap. Karena itu, sistem harus selalu tahu setiap perubahan yang terjadi dalam pemenuhan kebutuhan memori

\subsection Solusi:
Lubang-lubang kecil di antara blok-blok memori yang digunakan dapat diatasi dengan pemadatan memori yaitu menggabungkan semua lubang kecil menjadi satu lubang besar dengan memindahkan semua proses agar saling berdekatan.
 
Strategi Alokasi Memori
a) First fit algorithm : memory  manager men-scan list untuk  menemukan hole yg cukup untuk menampung proses yg baru. Proses akan menempati hole pertama yg ditemuinya yg cukup untuk dirinya.
b) Fit algoritma berikutnya: sama seperti fit pertama, tetapi pencarian lubang dimulai dari lubang yang ditemui dari pemindaian sebelumnya.
c) Algoritme fit terbaik: mencari lubang yang akan menghasilkan residu paling sedikit setelah proses masuk.
d) Algoritma fit terburuk: kebalikan dari fit terbaik.
e) Quick Fit algorithm: group holes dan membuat daftar mereka sendiri. Misalnya, ada daftar untuk 4K lubang, satu daftar untuk 8K, dan seterusnya.

\subsection Random Access Memory
Selain menyimpan data tidak lenyap, simpan juga ke disk tujuan untuk mengosongkan RAM agar tidak cepat penuh.
Dalam sistem ini kita juga dapat melihat bahwa sistem operasi terletak berdekatan dengan program lain dalam RAM sehingga kemungkinan sistem operasi terganggu atau diubah oleh proses yang berjalan sangat sangat besar. Seharusnya tidak terjadi.Untuk dapat  mencegah terganggunya sistem operasi tersebut maka alamat yang tertinggi dari sistem operasi yang diletakkan pada register batas dalam CPU tersebut. Jika ada proses yang mengacu ke alamat itu atau yang lebih rendah dari itu maka proses tersebut akan di hentikan dan program tersebut akan menampilkan pesan kesalahan yang terjadi.