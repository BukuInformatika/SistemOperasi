 \section manajemen memori
Konsep dasar memori
Memori adalah sebagai suatu pusat dari operasi yang pada setiap komputer modern , memori berfungsi untuk tempat penyimpan sebuah informasi dan harus diatur dan dijaga sebaik mungkin . memori adalah array besar dari word atau byte yang bisa disebut alamat CPU untuk mengambil instruksi instruksi dari sebuah memory


\subsection
Ketika kita bekerja dengan program aplikasi bagdi maka kita akan menghasilkan data. Data akan disimpan sementara di RAM yang tersisa. Data yang disimpan dalam RAM adalah voletile, yang berarti bahwa data hanya dapat bertahan selama daya komputer AKTIF. Untuk menyimpan kebiasaan menyimpan data ke disk dalam waktu tidak terlalu lama, misalnya setiap 5 menit

\subsection Fair Physical Memory Allocation
Alokasi Memori Fisik yang Adil
Alokasi Memori Fisik yang Adil Biasanya Digunakan dalam Manajemen Memori untuk membagi penggunaan memori fisik secara merata ke dalam setiap proses yang berjalan pada suatu sistem.

\subsection Shared Virtual Memory.
Meskipun dalam Memori Virtual Bersama setiap proses menggunakan ruang alamat yang berbeda dari memori virtual, tetapi ada kalanya proses dihadapkan untuk berbagi penggunaan memori.

\subsection Multiprogramming Dengan Swapping
manajemen memori adalah kegiatan mentransfer gambar proses antara memori utama dan disk selama eksekusi, atau dengan kata lain itu adalah pengalihan manajemen proses dari memori utama ke disk dan kembali lagi (swapping). Manajemen ini terdiri dari:
Multiprogramming Dengan Partisi Dinamis
Jumlah, lokasi, dan ukuran proses dalam memori dapat bervariasi dari waktu ke waktu secara dinamis. Kelemahan: a) Lubang memori dapat terjadi di antara partisi yang digunakan; b) mempersulit alokasi dan alokasi memori.

\subsection Swapping
salah satu contoh untuk mengilustrasikan teknik swapping adalah Algoritma Round-Robin yang digunakan dalam lingkungan multiprogamming menggunakan waktu kuantum dalam pelaksanaan prosesnya. ketika waktu kuantum berakhir, pengelola memori akan mengeluarkan proses lain ke dalam memori bebas.. Pada saat yang sama, penjadwal CPU akan mengalokasikan waktu ke proses lain dalam memori. Perhatiannya adalah bahwa waktu kuantum harus cukup lama sehingga waktu penggunaan CPU dapat lebih optimal bila dibandingkan dengan proses pertukaran yang terjadi antara memori dan disk.
Teknik swapping bergulir keluar, roll in menggunakan algoritma berbasis prioritas di mana ketika proses prioritas yang lebih tinggi tiba maka manajer memori akan mengeluarkan proses prioritas yang lebih rendah dan memuat proses dengan prioritas yang lebih tinggi.

\subsection Solusi:
Lubang-lubang kecil di antara blok-blok memori yang digunakan dapat diatasi dengan pemadatan memori yaitu menggabungkan semua lubang kecil menjadi satu lubang besar dengan memindahkan semua proses agar saling berdekatan.
 
Strategi Alokasi Memori
a)      First fit algorithm : memory  manager men-scan list untuk  menemukan hole yg cukup untuk menampung proses yg baru. Proses akan menempati hole pertama yg ditemuinya yg cukup untuk dirinya.

