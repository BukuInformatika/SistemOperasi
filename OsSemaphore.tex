\section{System Operasi Semaphore}
\subsection{Definisi}
Semaphore pada system operasi merupakan sebuah variabel bertipe integer. Di kehidupan nyata, semaphore merupakan sistem sinyal yang
digunakan untuk memberi sinyal atau tanda dan berkomunikasi secara visual. Pada software, semaphore merupakan sebuah variabel bertipe data
integer yang selain saat inisialisasi, yang hanya dapat diakses melalui dua operasi standar, yaitu increment dan decrement. Semaphore bias
digunakan untuk menyelesaikan masalah sinkronisasi secara umum, berdasarkan jenisnya. Semaphore hanya memiliki nilai 1 atau 0, atau lebih
dari sama dengan 0. Konsep semaphore pertama kali diajukan idenya oleh Edsger Dijkstra pada tahun 1967. Semaphore memiliki dua jenis,
yaitu, Biner semaphore dan counting semaphore. Biner semaphore hanya memiliki nilai 1 atau 0, Sering juga disebut sebagai semaphore
primitive. Sedangkan Counting semaphore memiliki nilai 0, 1, sampai seterusnya atau integer lainnya. Banyak sistem operasi yang tidak
secara langsung menggunakan semaphore jenis ini, namun lebih banyak yang memanfaatkan semaphore jenis biner semaphore

\subsection{Prinsip Semaphore}
1. Suatu proses yang berbeda bisa berkaitan dengan memanfaatkan sinyal - sinyal.
2. Suatu proses dapat dihentikan oleh 
3. Semaphore bertipe data integer yang diakses oleh dua operasi atomik standar (wait dan signal).
4. Ada dua operasi di semaphore (Down dan UP). Yang nama aslinya : P dan V.

\subsection{Kelemahan Semaphore}
1. Semaphore termasuk Low Level
2. Dikarenaka semaphore tersebar di dalam seluruh program maka kita akan kesulitan dalam pemeliharaannya.
3. Jika kita menghapus \"wait\" akan mengakibatkan \"nonmutual exclusion\"
4. Jika kita menghapus \"signal\" akan mengakibatkan \"deadlock\"
5. Jika terjadi deadlock akan sulti untuk dideteksi.
