/section{Fragmentasi}
/subsection{Pengertian fragmentasi}	
	Penyimpanan di komputer, fragmentasi adalah sebuah fenomena di ruang penyimpanan yang digunakan secara tidak efisien, mengurangi kapasitas penyimpanan. Istilah ini juga digunakan untuk menunjukkan tempat yang gersang itu sendiri.
	Ada tiga bentuk yang terkait dengan fragmentasi: fragmentasi eksternal, internal fragmentasi, dan data fragmentasi. Berbagai skema alokasi penyimpanan pameran satu atau beberapa kelemahan. Fragmentasi dapat diterima di kembali untuk peningkatan kecepatan atau kesedeahan
	
	/subsection{fragmentasi internal)
	Fragmentasi internal terjadi saat penyimpanan dialokasikan tanpa pernah ingin menggunakannya. [1] Ini adalah ruang-siakan. Sementara ini tampaknya bodoh, sering diterima dalam kembali untuk meningkatkan efisiensi atau kesederhanaan. Istilah “internal” merujuk pada kenyataan bahwa unusable penyimpanan yang dialokasikan di dalam wilayah namun tidak sedang digunakan.
	Misalnya, dalam banyak sistem file, setiap file selalu dimulai pada awal sebuah cluster, karena ini simplifies organisasi dan memudahkan tumbuh file. Setiap ruang kiri atas antara terakhir byte dari file yang pertama dan byte berikutnya dari cluster adalah bentuk internal disebut fragmentasi file atau kendur kendur ruang. [2] [3]

    /subsection{fragmentasi eksternal}
    Fragmentasi eksternal adalah fenomena yang gratis menjadi dibagi menjadi beberapa bagian kecil dari waktu ke waktu. [1] Ini adalah kelemahan dari beberapa algoritma alokasi penyimpanan, terjadi ketika aplikasi dan mengalokasikan deallocates ( “frees”) dari daerah penyimpanan berbagai ukuran, dan alokasi oleh algoritma merespon meninggalkan dialokasikan dan deallocated daerah interspersed. Hasilnya adalah bahwa, walaupun gratis tersedia, maka secara efektif unusable karena dibagi menjadi potongan-potongan yang terlalu kecil untuk memenuhi kebutuhan dari aplikasi. Istilah “eksternal” merujuk pada kenyataan bahwa unusable penyimpanan yang dialokasikan di luar daerah.

    /subsection{Segmentasi}
    Konsep segmentasi adalah user atau programmer tidak memikirkan sejumlah rutin program yang dipetakan ke main memori sebagai array linier dalam byte tetapi memori dilihat sebagai kumpulan segmen dengan ukuran berbeda-beda, tidak perlu berurutan diantara segment tersebut.
    Segmentasi adalah skema manajemen memori yang memungkinkan user untuk melihat memori tersebut. Ruang alamat logika adalah kumpulan segmen. Setiap segmen mempunyai nama dan panjang. Spesifikasi alamat berupa nama segmen dan offset. Segment diberi nomor dan disebut dengan nomor segmen (bukan nama segmen) atau segment number. Segmen dibentuk secara otomatis oleh compiler.


    /subsection{Memori virtual}
    Dalam ilmu komputer, memori virtual adalah teknik manajemen memori yang dikembangkan untuk kernel multitugas. Teknik ini divirtualisasikan dalam berbagai bentuk arsitektur komputer dari komputer penyimpanan data (seperti memori akses acak dan cakram penyimpanan), yang memungkinkan sebuah program harus dirancang seolah-olah hanya ada satu jenis memori, memori “virtual”, yang bertindak secara langsung beralamat memori baca/tulis (RAM).
    Sebagian besar sistem operasi modern yang mendukung memori virtual juga menjalankan setiap proses di ruang alamatkhususnya sendiri. Setiap program dengan demikian tampaknya memiliki akses tunggal ke memori virtual. Namun, beberapa sistem operasi yang lebih tua (seperti OS/VS1 dan OS/VS2 SVS) dan bahkan yang modern yang (seperti IBM i) adalah sistem operasi ruang alamat tunggal yang menjalankan semua proses dalam ruang alamat tunggal yang terdiri dari memori virtual.

