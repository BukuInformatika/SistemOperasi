\section SISTEM PAGING
Sistem Paging Adalah sistem manajemen pada sistem operasi dalam mengatur program yang sedang berjalan. Program yang berjalan harus dimuat di memori utama. Kendala yang terjadi apabila suatu program lebih besar dibandingkan dengan memori utama yang tersedia.

Untuk mengatasi hal tersebut Sistem Paging mempunyai 2 solusi, yaitu:

\subsection Konsep Overlay
Dimana program yang dijalankan dipecah menjadi beberapa bagian yang dapat dimuat memori (overlay). Overlay yang belum diperlukan pada saat program berjalan (tidak sedang di eksekusi) disimpan di disk, dimana nantinya overlay tersebut akan dimuat ke memori begitu diperlukan dalam eksekusinya.

\section sistem paging 
pengertian sistem paging 
Sistem Paging Merupakan sistem yang memanajemen pada sistem operasi dalam mengelola program program yang sedang berjalan. Program program yang berjalan harus dimuat didalam memori utama. Batasan batasan yang terjadi saat program lebih besar dari memori utama yang telah tersedia.
\subsection pembagian solusi sistem paging 
a.Konsep Overlay
b.Konsep Memori Maya (virtual Memory)


\section Masalah Penggantian Halaman
 
Pada dasarnya, kesalahan halaman (page fault) sudah tidak lagi menjadi masalah yang terlalu  dianggap serius. Hal ini disebabkan karena masing-masing halaman pasti akan mengalami paling tidak satu kali kesalahan dalam pemberian halaman, yakni ketika halaman ini ditunjuk untuk pertama kalinya. Representasi seperti ini sebenarnya tidaklah terlalu akurat. 