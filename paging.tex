\section SISTEM PAGING
Sistem Paging Adalah sistem manajemen pada sistem operasi dalam mengatur program yang sedang berjalan. Program yang berjalan harus dimuat di memori utama. Kendala yang terjadi apabila suatu program lebih besar dibandingkan dengan memori utama yang tersedia.

Untuk mengatasi hal tersebut Sistem Paging mempunyai 2 solusi, yaitu:

\subsection Konsep Overlay
Dimana program yang dijalankan dipecah menjadi beberapa bagian yang dapat dimuat memori (overlay). Overlay yang belum diperlukan pada saat program berjalan (tidak sedang di eksekusi) disimpan di disk, dimana nantinya overlay tersebut akan dimuat ke memori begitu diperlukan dalam eksekusinya.
<<<<<<< HEAD
- Konsep Memori Maya (virtual Memory)
Adalah kemampuan mengalamati ruang memori melebihi memori utama yang tersedia. Konsep ini pertama kali dikemukakan Fotheringham pada tahun 1961 untuk sistem komputer Atlas di Universitas Manchester, Inggris.

Gagasan Memori Maya adalah ukuran gabungan program, data dan stack melampaui jumlah memori fisik yang tersedia. Sistem operasi menyimpan bagian-bagian proses yang sedang digunakan di memori utama dan sisanya di disk. Begitu bagian di disk diperlukan maka bagian memori yang tidak diperlukan disingkirkan dan diganti bagian disk yang diperlukan.
=======

<<<<<<< HEAD

\section Masalah Penggantian Halaman
disini punya mu ham

Berdasarkan pertimbangan tersebut, sebenarnya proses-proses yang memiliki 10 halaman hanya akan menggunakan setengah dari jumlah seluruh halaman yang dimilikinya. Kemudian demand paging akan menyimpan I/O yang dibutuhkan untuk mengisi 5 halaman yang belum pernah digunakan. Kita juga dapat meningkatkan derajat multiprogramming dengan menjalankan banyak proses sebanyak 2 kali.

Jika kita meningkatkan derajat multiprogramming, itu sama artinya dengan melakukan over-allocating terhadap memori
=======
\section sistem paging 
pengertian sistem paging 
Sistem Paging Merupakan sistem yang memanajemen pada sistem operasi dalam mengelola program program yang sedang berjalan. Program program yang berjalan harus dimuat didalam memori utama. Batasan batasan yang terjadi saat program lebih besar dari memori utama yang telah tersedia.

\subsection pembagian solusi sistem paging 
a. Konsep Overlay
Dimana program yang dijalankan dipecah menjadi beberapa bagian yang dapat dimuat memori (overlay). Overlay yang belum diperlukan pada saat program berjalan (tidak sedang di eksekusi) disimpan di disk, dimana nantinya overlay tersebut akan dimuat ke memori begitu diperlukan dalam eksekusinya.

b. Konsep Memori Maya (virtual Memory)
Adalah kemampuan mengalamati ruang memori melebihi memori utama yang tersedia. Konsep ini pertama kali dikemukakan Fotheringham pada tahun 1961 untuk sistem komputer Atlas di Universitas Manchester, Inggris.


