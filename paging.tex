\section SISTEM PAGING
Sistem Paging Adalah sistem manajemen pada sistem operasi dalam mengatur program yang sedang berjalan. Program yang berjalan harus dimuat di memori utama. Kendala yang terjadi apabila suatu program lebih besar dibandingkan dengan memori utama yang tersedia.

Untuk mengatasi hal tersebut Sistem Paging mempunyai 2 solusi, yaitu:

\subsection Konsep Overlay
Dimana program yang dijalankan dipecah menjadi beberapa bagian yang dapat dimuat memori (overlay). Overlay yang belum diperlukan pada saat program berjalan (tidak sedang di eksekusi) disimpan di disk, dimana nantinya overlay tersebut akan dimuat ke memori begitu diperlukan dalam eksekusinya.
<<<<<<< HEAD
- Konsep Memori Maya (virtual Memory)
Adalah kemampuan mengalamati ruang memori melebihi memori utama yang tersedia. Konsep ini pertama kali dikemukakan Fotheringham pada tahun 1961 untuk sistem komputer Atlas di Universitas Manchester, Inggris.

Gagasan Memori Maya adalah ukuran gabungan program, data dan stack melampaui jumlah memori fisik yang tersedia. Sistem operasi menyimpan bagian-bagian proses yang sedang digunakan di memori utama dan sisanya di disk. Begitu bagian di disk diperlukan maka bagian memori yang tidak diperlukan disingkirkan dan diganti bagian disk yang diperlukan.
=======

<<<<<<< HEAD

\section Masalah Penggantian Halaman
disini punya mu ham

Berdasarkan pertimbangan tersebut, sebenarnya proses-proses yang memiliki 10 halaman hanya akan menggunakan setengah dari jumlah seluruh halaman yang dimilikinya. Kemudian demand paging akan menyimpan I/O yang dibutuhkan untuk mengisi 5 halaman yang belum pernah digunakan. Kita juga dapat meningkatkan derajat multiprogramming dengan menjalankan banyak proses sebanyak 2 kali.

Jika kita meningkatkan derajat multiprogramming, itu sama artinya dengan melakukan over-allocating terhadap memori
=======
\section sistem paging 
pengertian sistem paging 
Sistem Paging Merupakan sistem yang memanajemen pada sistem operasi dalam mengelola program program yang sedang berjalan. Program program yang berjalan harus dimuat didalam memori utama. Batasan batasan yang terjadi saat program lebih besar dari memori utama yang telah tersedia.

\subsection pembagian solusi sistem paging 
<<<<<<< HEAD
a.Konsep Overlay
b.Konsep Memori Maya (virtual Memory)


\section Masalah Penggantian Halaman
 
Pada dasarnya, kesalahan halaman (page fault) sudah tidak lagi menjadi masalah yang terlalu  dianggap serius. Hal ini disebabkan karena masing-masing halaman pasti akan mengalami paling tidak satu kali kesalahan dalam pemberian halaman, yakni ketika halaman ini ditunjuk untuk pertama kalinya. Representasi seperti ini sebenarnya tidaklah terlalu akurat. 
=======
a. Konsep Overlay
Dimana program yang dijalankan dipecah menjadi beberapa bagian yang dapat dimuat memori (overlay). Overlay yang belum diperlukan pada saat program berjalan (tidak sedang di eksekusi) disimpan di disk, dimana nantinya overlay tersebut akan dimuat ke memori begitu diperlukan dalam eksekusinya.

b. Konsep Memori Maya (virtual Memory)
Adalah kemampuan mengalamati ruang memori melebihi memori utama yang tersedia. Konsep ini pertama kali dikemukakan Fotheringham pada tahun 1961 untuk sistem komputer Atlas di Universitas Manchester, Inggris.

\section penggantian page 
Pada dasarnya, kesalahan halaman tidak lagi menjadi masalah serius. Ini karena setiap halaman akan mengalami setidaknya satu kesalahan dalam pengiriman halaman, yaitu ketika halaman ini ditujukan untuk pertama kalinya. Representasi seperti ini sebenarnya tidak terlalu akurat. Berdasarkan pertimbangan ini, sebenarnya proses yang memiliki 10 halaman hanya akan menggunakan setengah dari jumlah total halaman yang dimilikinya. Kemudian permintaan paging akan menyimpan I / O yang diperlukan untuk mengisi 5 halaman yang belum pernah digunakan. Kami juga dapat meningkatkan tingkat multiprogramming dengan menjalankan beberapa proses 2 kali.

Jika kita meningkatkan derajat multiprogramming, itu sama artinya dengan melakukan over-allocating terhadap memori. Jika kita menjalankan 6 proses, dengan masing-masing mendapatkan 10 halaman, walau pun sebenarnya yang digunakan hanya 5 halaman, kita akan memiliki utilisasi CPU dan throughput yang lebih tinggi dengan 10 frame yang masih kosong.

Lebih jauh lagi, kita harus mempertimbangkan bahwa sistem memori tidak hanya digunakan untuk menangani pengalamatan suatu program. Buffer untuk I / O juga menggunakan beberapa memori. Penggunaan ini dapat meningkatkan penggunaan algoritma dalam penempatan di memori.
Beberapa sistem mengalokasikan persis beberapa persen dari memori mereka untuk menyangga I / O, di mana keduanya,


>>>>>>> bb5b1b8fb5ec356edc5fac3a7fdb84aeb4100dc3

Beberapa sistem mengalokasikan tepat beberapa persen dari memori mereka untuk buffer I / O, keduanya, baik proses pengguna dan subsistem I / O race untuk mengambil keuntungan dari seluruh sistem memori.

\section skema dasar pemindahan halaman 
Pemindahan halaman menggunakan pendekatan berikut. Jika tidak ada frame yang kosong, kami mencari frame yang tidak digunakan dan mengosongkannya. Kita dapat mengosongkan bingkai dengan menulis isinya ke dalam ruang swap, dan mengubah tabel halaman (serta tabel lainnya) untuk menunjukkan bahwa halaman tidak akan lama dalam memori.

\subsection cara memindahkan halaman
1. Cari lokasi dari halaman yang diinginkan pada disk

2. Cari frame kosong
