\section{Fragmentasi}
\subsection{Fragmentasi Data}
	Berdasarkan artikel yang berjudul Materi manajemen memori \cite{tawarmateri} Fragmentasi data adalah suatu peristiwa  yang terjadi ketika sebuah bagian dari data dalam memori rusak dan terbagi ke dalam banyak potongan-potongan yang tidak saling berdekatan. Hal ini biasanya terjadi setelah mencoba untuk memasukkan benda yang besar ke dalam penyimpanan yang telah mengalami fragmentasi eksternal.
    
    Contohnya, file yang berada dalam file sistem telah diatur dalam unit yang disebut blok atau kelompok. Pada saat sebuah file sistem yang dibuat, ada suatu ruang untuk menyimpan blok bersama contiguously. Situasi semacam ini memungkinkan membaca dan menulis file secara cepat dam berurut. Namun, jika file ditambahkan, dihapus, dan ukurannya diubah, ruang bagi menjadi eksternal dan hanya meninggalkan lubang kecil di tempat yang tepat untuk data baru. Jika file yang baru ditambahkan, atau jika file yang sudah ada diperpanjang, maka data baru blok akan tersebar. Hal ini terjadi karena perlambatan akses untuk mencari waktu dan pemutaran penundaan dari membaca / menulis head, dan overhead incurring tambahan untuk mengelola tambahan lokasi. Hal ini disebut fragmentasi file system.
